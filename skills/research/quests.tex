\documentclass{article}

\usepackage{amsfonts, amsmath, amssymb, amsthm}
\usepackage{hyperref, graphicx, verbatim, listings, multirow}
\usepackage{algorithm, algorithmic}
% \usepackage{fancyhdr}

% \ifdefined\makeindex
%   \message{\string\makeindex\space is defined}\usepackage{natbib}\setcitestyle{super}%
% \else
%   \message{no command \string\makeindex}%
% \fi

% To remove excessive vertical spacing while using itemize etc.
\usepackage{enumitem}
\setlist{nolistsep}

% Include and configure tikz
\usepackage{tikz}
\usetikzlibrary{arrows,decorations,fit,backgrounds}

\tikzstyle{gm-var-constant}=[rectangle, draw, fill=black!30, thick, minimum size=1mm]
\tikzstyle{gm-var-hidden}=[circle, draw, thick, minimum size=1mm]
\tikzstyle{gm-var-seen}=[gm-var-hidden, fill=black!30, minimum size=1mm]
\tikzstyle{gm-plate} = [thick,draw=black,rounded corners=2mm]

% For clique trees.
\tikzstyle{ct-shared-nodes}=[rectangle, draw, fill=black!30, thick, minimum size=1mm]
\tikzstyle{ct-clique}=[circle, draw, thick, minimum size=1mm]


% \usepackage[bottom]{footmisc}

% COnfigure the listings package: break lines.
\lstset{breaklines=true}

\setcounter{tocdepth}{3}

% Lets verbatim and verb environments automatically break lines.
\makeatletter
\def\@xobeysp{ }
\makeatother
% \lstset{breaklines=true,basicstyle=\ttfamily}

% Packages not included:
% For multiline comments, use caption package. But this conflicts with hyperref while making html files.
% subfigure conflicts with use with memoir style-sheet.

% Configuration for the memoir class.
\renewcommand{\cleardoublepage}{}
% \renewcommand*{\partpageend}{}
\renewcommand{\afterpartskip}{}
\maxsecnumdepth{subsubsection} % number subsections
\maxtocdepth{subsubsection}

\addtolength{\parindent}{-5mm}
% Packages not included:
% For multiline comments, use caption package. But this conflicts with hyperref while making html files.
% subfigure conflicts with use with memoir style-sheet.

% Use something like:
% % Use something like:
% % Use something like:
% \input{../../macros}

% groupings of objects.
\newcommand{\set}[1]{\ensuremath{\left\{ #1 \right\}}}
\newcommand{\seq}[1]{\ensuremath{\left(#1\right)}}
\newcommand{\ang}[1]{\ensuremath{\langle#1\rangle}}
\newcommand{\tuple}[1]{\ensuremath{\left(#1\right)}}
\newcommand{\size}[1]{\ensuremath{\left| #1\right|}}

% numerical shortcuts.
\newcommand{\abs}[1]{\ensuremath{\left| #1\right|}}
\newcommand{\floor}[1]{\ensuremath{\left\lfloor #1 \right\rfloor}}
\newcommand{\ceil}[1]{\ensuremath{\left\lceil #1 \right\rceil}}

% linear algebra shortcuts.
\newcommand{\change}{\ensuremath{\Delta}}
\newcommand{\norm}[1]{\ensuremath{\left\| #1\right\|}}
\newcommand{\dprod}[1]{\ensuremath{\langle#1\rangle}}
\newcommand{\linspan}[1]{\ensuremath{\langle#1\rangle}}
\newcommand{\conj}[1]{\ensuremath{\overline{#1}}}
\newcommand{\der}{\ensuremath{\frac{d}{dx}}}
\newcommand{\lap}{\ensuremath{\Delta}}
\newcommand{\kron}{\ensuremath{\otimes}}
\newcommand{\nperp}{\ensuremath{\nvdash}}

\newcommand{\mat}[1]{\left[ \begin{smallmatrix}#1 \end{smallmatrix} \right]}

% derivatives and limits
\newcommand{\partder}[2]{\ensuremath{\frac{\partial #1}{\partial #2}}}
\newcommand{\partdern}[3]{\ensuremath{\frac{\partial^{#3} #1}{\partial #2^{#3}}}}
\newcommand{\gradient}{\ensuremath{\nabla}}
\newcommand{\subdifferential}{\ensuremath{\partial}}

% Arrows
\newcommand{\diverge}{\ensuremath{\nearrow}}
\newcommand{\notto}{\ensuremath{\nrightarrow}}
\newcommand{\up}{\ensuremath{\uparrow}}
\newcommand{\down}{\ensuremath{\downarrow}}
% gets and gives are defined!

% ordering operators
\newcommand{\oleq}{\preceq}
\newcommand{\ogeq}{\succeq}

% programming and logic operators
\newcommand{\dfn}{:=}
\newcommand{\assign}{:=}
\newcommand{\co}{\ co\ }
\newcommand{\en}{\ en\ }


% logic operators
\newcommand{\xor}{\ensuremath{\oplus}}
\newcommand{\Land}{\ensuremath{\bigwedge}}
\newcommand{\Lor}{\ensuremath{\bigvee}}
\newcommand{\finish}{\ensuremath{\Box}}
\newcommand{\contra}{\ensuremath{\Rightarrow \Leftarrow}}
\newcommand{\iseq}{\ensuremath{\stackrel{_?}{=}}}


% Set theory
\newcommand{\symdiff}{\ensuremath{\Delta}}
\newcommand{\union}{\ensuremath{\cup}}
\newcommand{\inters}{\ensuremath{\cap}}
\newcommand{\Union}{\ensuremath{\bigcup}}
\newcommand{\Inters}{\ensuremath{\bigcap}}
\newcommand{\nullSet}{\ensuremath{\phi}}


% graph theory
\newcommand{\nbd}{\Gamma}

% Script alphabets
% For reals, use \Re

% greek letters
\newcommand{\eps}{\ensuremath{\epsilon}}
\newcommand{\del}{\ensuremath{\delta}}
\newcommand{\ga}{\ensuremath{\alpha}}
\newcommand{\gb}{\ensuremath{\beta}}
\newcommand{\gd}{\ensuremath{\del}}
\newcommand{\gp}{\ensuremath{\pi}}
\newcommand{\gf}{\ensuremath{\phi}}
\newcommand{\gh}{\ensuremath{\eta}}
\newcommand{\gF}{\ensuremath{\Phi}}
\newcommand{\gl}{\ensuremath{\lambda}}
\newcommand{\gm}{\ensuremath{\mu}}
\newcommand{\gn}{\ensuremath{\nu}}
\newcommand{\gr}{\ensuremath{\rho}}
\newcommand{\gs}{\ensuremath{\sigma}}
\newcommand{\gt}{\ensuremath{\theta}}
\newcommand{\gx}{\ensuremath{\xi}}

\newcommand{\sw}{\ensuremath{\sigma}}
\newcommand{\SW}{\ensuremath{\Sigma}}
\newcommand{\ew}{\ensuremath{\lambda}}
\newcommand{\EW}{\ensuremath{\Lambda}}

\newcommand{\Del}{\ensuremath{\Delta}}
\newcommand{\gD}{\ensuremath{\Delta}}
\newcommand{\gG}{\ensuremath{\Gamma}}
\newcommand{\gO}{\ensuremath{\Omega}}
\newcommand{\gS}{\ensuremath{\Sigma}}

% Bold english letters.
\newcommand{\bA}{\ensuremath{\mathbf{A}}}
\newcommand{\bB}{\ensuremath{\mathbf{B}}}
\newcommand{\bC}{\ensuremath{\mathbf{C}}}
\newcommand{\bD}{\ensuremath{\mathbf{D}}}
\newcommand{\bE}{\ensuremath{\mathbf{E}}}
\newcommand{\bF}{\ensuremath{\mathbf{F}}}
\newcommand{\bG}{\ensuremath{\mathbf{G}}}
\newcommand{\bH}{\ensuremath{\mathbf{H}}}
\newcommand{\bI}{\ensuremath{\mathbf{I}}}
\newcommand{\bJ}{\ensuremath{\mathbf{J}}}
\newcommand{\bK}{\ensuremath{\mathbf{K}}}
\newcommand{\bL}{\ensuremath{\mathbf{L}}}
\newcommand{\bM}{\ensuremath{\mathbf{M}}}
\newcommand{\bN}{\ensuremath{\mathbf{N}}}
\newcommand{\bO}{\ensuremath{\mathbf{O}}}
\newcommand{\bP}{\ensuremath{\mathbf{P}}}
\newcommand{\bQ}{\ensuremath{\mathbf{Q}}}
\newcommand{\bR}{\ensuremath{\mathbf{R}}} % PROSPER defines \R
\newcommand{\bS}{\ensuremath{\mathbf{S}}}
\newcommand{\bT}{\ensuremath{\mathbf{T}}}
\newcommand{\bU}{\ensuremath{\mathbf{U}}}
\newcommand{\bV}{\ensuremath{\mathbf{V}}}
\newcommand{\bW}{\ensuremath{\mathbf{W}}}
\newcommand{\bX}{\ensuremath{\mathbf{X}}}
\newcommand{\bY}{\ensuremath{\mathbf{Y}}}
\newcommand{\bZ}{\ensuremath{\mathbf{Z}}}
\newcommand{\bba}{\ensuremath{\mathbf{a}}}
\newcommand{\bbb}{\ensuremath{\mathbf{b}}}
\newcommand{\bbc}{\ensuremath{\mathbf{c}}}
\newcommand{\bbd}{\ensuremath{\mathbf{d}}}
\newcommand{\bbe}{\ensuremath{\mathbf{e}}}
\newcommand{\bbf}{\ensuremath{\mathbf{f}}}
\newcommand{\bbg}{\ensuremath{\mathbf{g}}}
\newcommand{\bbh}{\ensuremath{\mathbf{h}}}
\newcommand{\bbk}{\ensuremath{\mathbf{k}}}
\newcommand{\bbl}{\ensuremath{\mathbf{l}}}
\newcommand{\bbm}{\ensuremath{\mathbf{m}}}
\newcommand{\bbn}{\ensuremath{\mathbf{n}}}
\newcommand{\bbp}{\ensuremath{\mathbf{p}}}
\newcommand{\bbq}{\ensuremath{\mathbf{q}}}
\newcommand{\bbr}{\ensuremath{\mathbf{r}}}
\newcommand{\bbs}{\ensuremath{\mathbf{s}}}  % TIPA defines \s and LaTeX \ss!
\newcommand{\bbt}{\ensuremath{\mathbf{t}}}
\newcommand{\bbu}{\ensuremath{\mathbf{u}}}
\newcommand{\bbv}{\ensuremath{\mathbf{v}}}
\newcommand{\bbw}{\ensuremath{\mathbf{w}}}
\newcommand{\bbx}{\ensuremath{\mathbf{x}}}
\newcommand{\bby}{\ensuremath{\mathbf{y}}}
\newcommand{\bbz}{\ensuremath{\mathbf{z}}}
\newcommand{\0}{\ensuremath{\mathbf{0}}}
\newcommand{\1}{\ensuremath{\mathbf{1}}}


% Formatting shortcuts
\newcommand{\red}[1]{\textcolor{red}{#1}}
\newcommand{\green}[1]{\textcolor{green}{#1}}
\newcommand{\magenta}[1]{\textcolor{magenta}{#1}}
\newcommand{\orange}[1]{\textcolor{orange}{#1}}
\newcommand{\gray}[1]{\textcolor{gray}{#1}}
\newcommand{\blue}[1]{\textcolor{blue}{#1}}
\newcommand{\htext}[2]{\texorpdfstring{#1}{#2}}

% Statistics
\newcommand{\distr}{\ensuremath{\sim}}
\newcommand{\stddev}{\ensuremath{\sigma}}
\newcommand{\covmatrix}{\ensuremath{\Sigma}}
\newcommand{\mean}{\ensuremath{\mu}}
\newcommand{\param}{\ensuremath{\gt}}
\newcommand{\ftr}{\ensuremath{\phi}}

% General utility
\newcommand{\todo}[1]{\textbf{[TODO]}] \footnote{TODO: #1}}
\newcommand{\exclaim}[1]{{\textbf{\textit{#1}}}}
\newcommand{\tbc}{[\textbf{Incomplete}]}
\newcommand{\chk}{[\textbf{Check}]}
\newcommand{\oprob}{[\textbf{OP}]:}
\newcommand{\core}[1]{\textbf{Core Idea:}}
\newcommand{\why}{[\textbf{Find proof}]}
\newcommand{\opt}[1]{\textit{#1}}


\renewcommand{\~}{\htext{$\sim$}{~}}

\DeclareMathOperator*{\argmin}{arg\,min}
\DeclareMathOperator*{\argmax}{arg\,max}

% Items pertaining to the headed list.
\newcommand{\headeditem}[1]{\item\textbf{#1}}
\newcommand{\headedsubitem}[1]{\subitem\textbf{#1}}



% groupings of objects.
\newcommand{\set}[1]{\ensuremath{\left\{ #1 \right\}}}
\newcommand{\seq}[1]{\ensuremath{\left(#1\right)}}
\newcommand{\ang}[1]{\ensuremath{\langle#1\rangle}}
\newcommand{\tuple}[1]{\ensuremath{\left(#1\right)}}
\newcommand{\size}[1]{\ensuremath{\left| #1\right|}}

% numerical shortcuts.
\newcommand{\abs}[1]{\ensuremath{\left| #1\right|}}
\newcommand{\floor}[1]{\ensuremath{\left\lfloor #1 \right\rfloor}}
\newcommand{\ceil}[1]{\ensuremath{\left\lceil #1 \right\rceil}}

% linear algebra shortcuts.
\newcommand{\change}{\ensuremath{\Delta}}
\newcommand{\norm}[1]{\ensuremath{\left\| #1\right\|}}
\newcommand{\dprod}[1]{\ensuremath{\langle#1\rangle}}
\newcommand{\linspan}[1]{\ensuremath{\langle#1\rangle}}
\newcommand{\conj}[1]{\ensuremath{\overline{#1}}}
\newcommand{\der}{\ensuremath{\frac{d}{dx}}}
\newcommand{\lap}{\ensuremath{\Delta}}
\newcommand{\kron}{\ensuremath{\otimes}}
\newcommand{\nperp}{\ensuremath{\nvdash}}

\newcommand{\mat}[1]{\left[ \begin{smallmatrix}#1 \end{smallmatrix} \right]}

% derivatives and limits
\newcommand{\partder}[2]{\ensuremath{\frac{\partial #1}{\partial #2}}}
\newcommand{\partdern}[3]{\ensuremath{\frac{\partial^{#3} #1}{\partial #2^{#3}}}}
\newcommand{\gradient}{\ensuremath{\nabla}}
\newcommand{\subdifferential}{\ensuremath{\partial}}

% Arrows
\newcommand{\diverge}{\ensuremath{\nearrow}}
\newcommand{\notto}{\ensuremath{\nrightarrow}}
\newcommand{\up}{\ensuremath{\uparrow}}
\newcommand{\down}{\ensuremath{\downarrow}}
% gets and gives are defined!

% ordering operators
\newcommand{\oleq}{\preceq}
\newcommand{\ogeq}{\succeq}

% programming and logic operators
\newcommand{\dfn}{:=}
\newcommand{\assign}{:=}
\newcommand{\co}{\ co\ }
\newcommand{\en}{\ en\ }


% logic operators
\newcommand{\xor}{\ensuremath{\oplus}}
\newcommand{\Land}{\ensuremath{\bigwedge}}
\newcommand{\Lor}{\ensuremath{\bigvee}}
\newcommand{\finish}{\ensuremath{\Box}}
\newcommand{\contra}{\ensuremath{\Rightarrow \Leftarrow}}
\newcommand{\iseq}{\ensuremath{\stackrel{_?}{=}}}


% Set theory
\newcommand{\symdiff}{\ensuremath{\Delta}}
\newcommand{\union}{\ensuremath{\cup}}
\newcommand{\inters}{\ensuremath{\cap}}
\newcommand{\Union}{\ensuremath{\bigcup}}
\newcommand{\Inters}{\ensuremath{\bigcap}}
\newcommand{\nullSet}{\ensuremath{\phi}}


% graph theory
\newcommand{\nbd}{\Gamma}

% Script alphabets
% For reals, use \Re

% greek letters
\newcommand{\eps}{\ensuremath{\epsilon}}
\newcommand{\del}{\ensuremath{\delta}}
\newcommand{\ga}{\ensuremath{\alpha}}
\newcommand{\gb}{\ensuremath{\beta}}
\newcommand{\gd}{\ensuremath{\del}}
\newcommand{\gp}{\ensuremath{\pi}}
\newcommand{\gf}{\ensuremath{\phi}}
\newcommand{\gh}{\ensuremath{\eta}}
\newcommand{\gF}{\ensuremath{\Phi}}
\newcommand{\gl}{\ensuremath{\lambda}}
\newcommand{\gm}{\ensuremath{\mu}}
\newcommand{\gn}{\ensuremath{\nu}}
\newcommand{\gr}{\ensuremath{\rho}}
\newcommand{\gs}{\ensuremath{\sigma}}
\newcommand{\gt}{\ensuremath{\theta}}
\newcommand{\gx}{\ensuremath{\xi}}

\newcommand{\sw}{\ensuremath{\sigma}}
\newcommand{\SW}{\ensuremath{\Sigma}}
\newcommand{\ew}{\ensuremath{\lambda}}
\newcommand{\EW}{\ensuremath{\Lambda}}

\newcommand{\Del}{\ensuremath{\Delta}}
\newcommand{\gD}{\ensuremath{\Delta}}
\newcommand{\gG}{\ensuremath{\Gamma}}
\newcommand{\gO}{\ensuremath{\Omega}}
\newcommand{\gS}{\ensuremath{\Sigma}}

% Bold english letters.
\newcommand{\bA}{\ensuremath{\mathbf{A}}}
\newcommand{\bB}{\ensuremath{\mathbf{B}}}
\newcommand{\bC}{\ensuremath{\mathbf{C}}}
\newcommand{\bD}{\ensuremath{\mathbf{D}}}
\newcommand{\bE}{\ensuremath{\mathbf{E}}}
\newcommand{\bF}{\ensuremath{\mathbf{F}}}
\newcommand{\bG}{\ensuremath{\mathbf{G}}}
\newcommand{\bH}{\ensuremath{\mathbf{H}}}
\newcommand{\bI}{\ensuremath{\mathbf{I}}}
\newcommand{\bJ}{\ensuremath{\mathbf{J}}}
\newcommand{\bK}{\ensuremath{\mathbf{K}}}
\newcommand{\bL}{\ensuremath{\mathbf{L}}}
\newcommand{\bM}{\ensuremath{\mathbf{M}}}
\newcommand{\bN}{\ensuremath{\mathbf{N}}}
\newcommand{\bO}{\ensuremath{\mathbf{O}}}
\newcommand{\bP}{\ensuremath{\mathbf{P}}}
\newcommand{\bQ}{\ensuremath{\mathbf{Q}}}
\newcommand{\bR}{\ensuremath{\mathbf{R}}} % PROSPER defines \R
\newcommand{\bS}{\ensuremath{\mathbf{S}}}
\newcommand{\bT}{\ensuremath{\mathbf{T}}}
\newcommand{\bU}{\ensuremath{\mathbf{U}}}
\newcommand{\bV}{\ensuremath{\mathbf{V}}}
\newcommand{\bW}{\ensuremath{\mathbf{W}}}
\newcommand{\bX}{\ensuremath{\mathbf{X}}}
\newcommand{\bY}{\ensuremath{\mathbf{Y}}}
\newcommand{\bZ}{\ensuremath{\mathbf{Z}}}
\newcommand{\bba}{\ensuremath{\mathbf{a}}}
\newcommand{\bbb}{\ensuremath{\mathbf{b}}}
\newcommand{\bbc}{\ensuremath{\mathbf{c}}}
\newcommand{\bbd}{\ensuremath{\mathbf{d}}}
\newcommand{\bbe}{\ensuremath{\mathbf{e}}}
\newcommand{\bbf}{\ensuremath{\mathbf{f}}}
\newcommand{\bbg}{\ensuremath{\mathbf{g}}}
\newcommand{\bbh}{\ensuremath{\mathbf{h}}}
\newcommand{\bbk}{\ensuremath{\mathbf{k}}}
\newcommand{\bbl}{\ensuremath{\mathbf{l}}}
\newcommand{\bbm}{\ensuremath{\mathbf{m}}}
\newcommand{\bbn}{\ensuremath{\mathbf{n}}}
\newcommand{\bbp}{\ensuremath{\mathbf{p}}}
\newcommand{\bbq}{\ensuremath{\mathbf{q}}}
\newcommand{\bbr}{\ensuremath{\mathbf{r}}}
\newcommand{\bbs}{\ensuremath{\mathbf{s}}}  % TIPA defines \s and LaTeX \ss!
\newcommand{\bbt}{\ensuremath{\mathbf{t}}}
\newcommand{\bbu}{\ensuremath{\mathbf{u}}}
\newcommand{\bbv}{\ensuremath{\mathbf{v}}}
\newcommand{\bbw}{\ensuremath{\mathbf{w}}}
\newcommand{\bbx}{\ensuremath{\mathbf{x}}}
\newcommand{\bby}{\ensuremath{\mathbf{y}}}
\newcommand{\bbz}{\ensuremath{\mathbf{z}}}
\newcommand{\0}{\ensuremath{\mathbf{0}}}
\newcommand{\1}{\ensuremath{\mathbf{1}}}


% Formatting shortcuts
\newcommand{\red}[1]{\textcolor{red}{#1}}
\newcommand{\green}[1]{\textcolor{green}{#1}}
\newcommand{\magenta}[1]{\textcolor{magenta}{#1}}
\newcommand{\orange}[1]{\textcolor{orange}{#1}}
\newcommand{\gray}[1]{\textcolor{gray}{#1}}
\newcommand{\blue}[1]{\textcolor{blue}{#1}}
\newcommand{\htext}[2]{\texorpdfstring{#1}{#2}}

% Statistics
\newcommand{\distr}{\ensuremath{\sim}}
\newcommand{\stddev}{\ensuremath{\sigma}}
\newcommand{\covmatrix}{\ensuremath{\Sigma}}
\newcommand{\mean}{\ensuremath{\mu}}
\newcommand{\param}{\ensuremath{\gt}}
\newcommand{\ftr}{\ensuremath{\phi}}

% General utility
\newcommand{\todo}[1]{\textbf{[TODO]}] \footnote{TODO: #1}}
\newcommand{\exclaim}[1]{{\textbf{\textit{#1}}}}
\newcommand{\tbc}{[\textbf{Incomplete}]}
\newcommand{\chk}{[\textbf{Check}]}
\newcommand{\oprob}{[\textbf{OP}]:}
\newcommand{\core}[1]{\textbf{Core Idea:}}
\newcommand{\why}{[\textbf{Find proof}]}
\newcommand{\opt}[1]{\textit{#1}}


\renewcommand{\~}{\htext{$\sim$}{~}}

\DeclareMathOperator*{\argmin}{arg\,min}
\DeclareMathOperator*{\argmax}{arg\,max}

% Items pertaining to the headed list.
\newcommand{\headeditem}[1]{\item\textbf{#1}}
\newcommand{\headedsubitem}[1]{\subitem\textbf{#1}}



% groupings of objects.
\newcommand{\set}[1]{\ensuremath{\left\{ #1 \right\}}}
\newcommand{\seq}[1]{\ensuremath{\left(#1\right)}}
\newcommand{\ang}[1]{\ensuremath{\langle#1\rangle}}
\newcommand{\tuple}[1]{\ensuremath{\left(#1\right)}}
\newcommand{\size}[1]{\ensuremath{\left| #1\right|}}

% numerical shortcuts.
\newcommand{\abs}[1]{\ensuremath{\left| #1\right|}}
\newcommand{\floor}[1]{\ensuremath{\left\lfloor #1 \right\rfloor}}
\newcommand{\ceil}[1]{\ensuremath{\left\lceil #1 \right\rceil}}

% linear algebra shortcuts.
\newcommand{\change}{\ensuremath{\Delta}}
\newcommand{\norm}[1]{\ensuremath{\left\| #1\right\|}}
\newcommand{\dprod}[1]{\ensuremath{\langle#1\rangle}}
\newcommand{\linspan}[1]{\ensuremath{\langle#1\rangle}}
\newcommand{\conj}[1]{\ensuremath{\overline{#1}}}
\newcommand{\der}{\ensuremath{\frac{d}{dx}}}
\newcommand{\lap}{\ensuremath{\Delta}}
\newcommand{\kron}{\ensuremath{\otimes}}
\newcommand{\nperp}{\ensuremath{\nvdash}}

\newcommand{\mat}[1]{\left[ \begin{smallmatrix}#1 \end{smallmatrix} \right]}

% derivatives and limits
\newcommand{\partder}[2]{\ensuremath{\frac{\partial #1}{\partial #2}}}
\newcommand{\partdern}[3]{\ensuremath{\frac{\partial^{#3} #1}{\partial #2^{#3}}}}
\newcommand{\gradient}{\ensuremath{\nabla}}
\newcommand{\subdifferential}{\ensuremath{\partial}}

% Arrows
\newcommand{\diverge}{\ensuremath{\nearrow}}
\newcommand{\notto}{\ensuremath{\nrightarrow}}
\newcommand{\up}{\ensuremath{\uparrow}}
\newcommand{\down}{\ensuremath{\downarrow}}
% gets and gives are defined!

% ordering operators
\newcommand{\oleq}{\preceq}
\newcommand{\ogeq}{\succeq}

% programming and logic operators
\newcommand{\dfn}{:=}
\newcommand{\assign}{:=}
\newcommand{\co}{\ co\ }
\newcommand{\en}{\ en\ }


% logic operators
\newcommand{\xor}{\ensuremath{\oplus}}
\newcommand{\Land}{\ensuremath{\bigwedge}}
\newcommand{\Lor}{\ensuremath{\bigvee}}
\newcommand{\finish}{\ensuremath{\Box}}
\newcommand{\contra}{\ensuremath{\Rightarrow \Leftarrow}}
\newcommand{\iseq}{\ensuremath{\stackrel{_?}{=}}}


% Set theory
\newcommand{\symdiff}{\ensuremath{\Delta}}
\newcommand{\union}{\ensuremath{\cup}}
\newcommand{\inters}{\ensuremath{\cap}}
\newcommand{\Union}{\ensuremath{\bigcup}}
\newcommand{\Inters}{\ensuremath{\bigcap}}
\newcommand{\nullSet}{\ensuremath{\phi}}


% graph theory
\newcommand{\nbd}{\Gamma}

% Script alphabets
% For reals, use \Re

% greek letters
\newcommand{\eps}{\ensuremath{\epsilon}}
\newcommand{\del}{\ensuremath{\delta}}
\newcommand{\ga}{\ensuremath{\alpha}}
\newcommand{\gb}{\ensuremath{\beta}}
\newcommand{\gd}{\ensuremath{\del}}
\newcommand{\gp}{\ensuremath{\pi}}
\newcommand{\gf}{\ensuremath{\phi}}
\newcommand{\gh}{\ensuremath{\eta}}
\newcommand{\gF}{\ensuremath{\Phi}}
\newcommand{\gl}{\ensuremath{\lambda}}
\newcommand{\gm}{\ensuremath{\mu}}
\newcommand{\gn}{\ensuremath{\nu}}
\newcommand{\gr}{\ensuremath{\rho}}
\newcommand{\gs}{\ensuremath{\sigma}}
\newcommand{\gt}{\ensuremath{\theta}}
\newcommand{\gx}{\ensuremath{\xi}}

\newcommand{\sw}{\ensuremath{\sigma}}
\newcommand{\SW}{\ensuremath{\Sigma}}
\newcommand{\ew}{\ensuremath{\lambda}}
\newcommand{\EW}{\ensuremath{\Lambda}}

\newcommand{\Del}{\ensuremath{\Delta}}
\newcommand{\gD}{\ensuremath{\Delta}}
\newcommand{\gG}{\ensuremath{\Gamma}}
\newcommand{\gO}{\ensuremath{\Omega}}
\newcommand{\gS}{\ensuremath{\Sigma}}

% Bold english letters.
\newcommand{\bA}{\ensuremath{\mathbf{A}}}
\newcommand{\bB}{\ensuremath{\mathbf{B}}}
\newcommand{\bC}{\ensuremath{\mathbf{C}}}
\newcommand{\bD}{\ensuremath{\mathbf{D}}}
\newcommand{\bE}{\ensuremath{\mathbf{E}}}
\newcommand{\bF}{\ensuremath{\mathbf{F}}}
\newcommand{\bG}{\ensuremath{\mathbf{G}}}
\newcommand{\bH}{\ensuremath{\mathbf{H}}}
\newcommand{\bI}{\ensuremath{\mathbf{I}}}
\newcommand{\bJ}{\ensuremath{\mathbf{J}}}
\newcommand{\bK}{\ensuremath{\mathbf{K}}}
\newcommand{\bL}{\ensuremath{\mathbf{L}}}
\newcommand{\bM}{\ensuremath{\mathbf{M}}}
\newcommand{\bN}{\ensuremath{\mathbf{N}}}
\newcommand{\bO}{\ensuremath{\mathbf{O}}}
\newcommand{\bP}{\ensuremath{\mathbf{P}}}
\newcommand{\bQ}{\ensuremath{\mathbf{Q}}}
\newcommand{\bR}{\ensuremath{\mathbf{R}}} % PROSPER defines \R
\newcommand{\bS}{\ensuremath{\mathbf{S}}}
\newcommand{\bT}{\ensuremath{\mathbf{T}}}
\newcommand{\bU}{\ensuremath{\mathbf{U}}}
\newcommand{\bV}{\ensuremath{\mathbf{V}}}
\newcommand{\bW}{\ensuremath{\mathbf{W}}}
\newcommand{\bX}{\ensuremath{\mathbf{X}}}
\newcommand{\bY}{\ensuremath{\mathbf{Y}}}
\newcommand{\bZ}{\ensuremath{\mathbf{Z}}}
\newcommand{\bba}{\ensuremath{\mathbf{a}}}
\newcommand{\bbb}{\ensuremath{\mathbf{b}}}
\newcommand{\bbc}{\ensuremath{\mathbf{c}}}
\newcommand{\bbd}{\ensuremath{\mathbf{d}}}
\newcommand{\bbe}{\ensuremath{\mathbf{e}}}
\newcommand{\bbf}{\ensuremath{\mathbf{f}}}
\newcommand{\bbg}{\ensuremath{\mathbf{g}}}
\newcommand{\bbh}{\ensuremath{\mathbf{h}}}
\newcommand{\bbk}{\ensuremath{\mathbf{k}}}
\newcommand{\bbl}{\ensuremath{\mathbf{l}}}
\newcommand{\bbm}{\ensuremath{\mathbf{m}}}
\newcommand{\bbn}{\ensuremath{\mathbf{n}}}
\newcommand{\bbp}{\ensuremath{\mathbf{p}}}
\newcommand{\bbq}{\ensuremath{\mathbf{q}}}
\newcommand{\bbr}{\ensuremath{\mathbf{r}}}
\newcommand{\bbs}{\ensuremath{\mathbf{s}}}  % TIPA defines \s and LaTeX \ss!
\newcommand{\bbt}{\ensuremath{\mathbf{t}}}
\newcommand{\bbu}{\ensuremath{\mathbf{u}}}
\newcommand{\bbv}{\ensuremath{\mathbf{v}}}
\newcommand{\bbw}{\ensuremath{\mathbf{w}}}
\newcommand{\bbx}{\ensuremath{\mathbf{x}}}
\newcommand{\bby}{\ensuremath{\mathbf{y}}}
\newcommand{\bbz}{\ensuremath{\mathbf{z}}}
\newcommand{\0}{\ensuremath{\mathbf{0}}}
\newcommand{\1}{\ensuremath{\mathbf{1}}}


% Formatting shortcuts
\newcommand{\red}[1]{\textcolor{red}{#1}}
\newcommand{\green}[1]{\textcolor{green}{#1}}
\newcommand{\magenta}[1]{\textcolor{magenta}{#1}}
\newcommand{\orange}[1]{\textcolor{orange}{#1}}
\newcommand{\gray}[1]{\textcolor{gray}{#1}}
\newcommand{\blue}[1]{\textcolor{blue}{#1}}
\newcommand{\htext}[2]{\texorpdfstring{#1}{#2}}

% Statistics
\newcommand{\distr}{\ensuremath{\sim}}
\newcommand{\stddev}{\ensuremath{\sigma}}
\newcommand{\covmatrix}{\ensuremath{\Sigma}}
\newcommand{\mean}{\ensuremath{\mu}}
\newcommand{\param}{\ensuremath{\gt}}
\newcommand{\ftr}{\ensuremath{\phi}}

% General utility
\newcommand{\todo}[1]{\textbf{[TODO]}] \footnote{TODO: #1}}
\newcommand{\exclaim}[1]{{\textbf{\textit{#1}}}}
\newcommand{\tbc}{[\textbf{Incomplete}]}
\newcommand{\chk}{[\textbf{Check}]}
\newcommand{\oprob}{[\textbf{OP}]:}
\newcommand{\core}[1]{\textbf{Core Idea:}}
\newcommand{\why}{[\textbf{Find proof}]}
\newcommand{\opt}[1]{\textit{#1}}


\renewcommand{\~}{\htext{$\sim$}{~}}

\DeclareMathOperator*{\argmin}{arg\,min}
\DeclareMathOperator*{\argmax}{arg\,max}

% Items pertaining to the headed list.
\newcommand{\headeditem}[1]{\item\textbf{#1}}
\newcommand{\headedsubitem}[1]{\subitem\textbf{#1}}



%opening
\title{Research quests}
\author{vishvAs vAsuki}

\begin{document}
\maketitle
\part{General thoughts}

For charactarization of important themes, modes and efforts of various types of research, see the relevant reference sheets.
Potential thesis topics:

\part{Computer science}
\section{Theoretical computer science}
Problems dealt with in theoretical computer science ultimately have their origin in practical problems. But, some of these problems are so hard that many subsidiary theoretical questions arise. Eg: Computational complexity research.

Our descendents, equipped with genetically engineered brains and better interface with machines, may be much better equipped to do deep theory. Perhaps theory is better served by accelerating their arrival.

\section{Experimental computer science}

Often, researchers in such areas use experiments when it is difficult to theoretically prove a conjecture. Eg: The study of wireless and wired networks, distributed systems, computational biology. The ideas are mathematically well motivated but unproved. Eg: Dimensionality reduction using Principal component analysis.

A rich source of problems is found in the act of building systems.
Characterization of research effort

\section{Algorithms}
Advantage of being an algorithmicist: can play in everyone's backyard.

Natural algorithms: Algorithmic models of biological systems.

\section{Numerical analysis}
Enter the field of continuous mathematics algorithms from the perspective of theoretical CS.

\section{Graphs}
Analytically prove power law group size distribution in KDD affiliation networks paper.

\subsection{Graph generation with certain community structure}
Given the k*k q(xhat, yhat) matrix, which specifies the edge densities between various node clusters, make a generative model which produces graphs with that clustering structure, such that it possesses various static and temporal properties of real world graphs.

\begin{itemize}
\item A trivial solution seems to exist if the diagonal entries heavily dominate others. Is this well motivated?
\subitem This was implemented in R using igraph. Power law degree distribution was experimentally verified. Yet to be verified: cluster structure.
\subitem Analytical proof of cluster structure pending. Is this an (a, b) cluster in the Tarjan sense?
\item Verify the matrix generated by the implemented code by applying information theoretic coclustering. Obstruction: inability to write sparse matreces due to lack of necessary library in old R version.
\end{itemize}

\subsection{Graph generation with hierarchical community structure}
Construct a graph generative model which, besides having good static and temporal properties displayed by graphs evolved using other models, also shows hierarchical community structure.

\begin{itemize}
\item This can be done if the idea to use affiliation networks algorithm to generate clusters and then interconnect them works.
\end{itemize}

\subsection{Co-Clustering affiliation networks}
credits: pratIka.

Time Complexity of the algorithm should be small: social networks tend to be large.

Community clusters defined by common members vs community clusters defined by dense subgraphs.

Use special properties of these networks to improve clustering. Try to use information from corresponding social network A to better cocluster the affiliation nw B.

\subsubsection{Ideas}
Look at users $\set{u_{i}}$ as features, then community c is a point in feature space: a binary vector. Want to map it into a kernel space where similarities between users is taken into account while measuring $\dprod{c_{1}, c_{2}} = k(c_{1}, c_{2}) = \sum_{i,j} s(u_{i}, u_{j})$ for similarity fn s.
\subitem Credits: pratIk.
\subitem (jengdo.Ng) For adj matrix A, any s() can be considered as $\sum w_{i}A^{i}$.

Using LSA or its Bayesian version LDA.
\subitem Find latent factors: $\min \norm{B-P^{T}Q} + g\norm{A-P^{T}P}$, where P and Q are latent factor matreces for rows(A) and cols(A). Thence finding latent factors P to account for both B and A. Then use LDA for coclustering.
\subitem Credits: jhengdong.

Try using a modified version of the Von Neumann kernel pair.

Minimize a normalized cut objective using established algorithms: spectral clustering, or graclus, which does not involve ev computation.

Merge user-user and user-group graphs giving different weights to different kinds of edges. Then try various clustering algorithms.

\paragraph*{Using (a, b) clustering}
(Tarjan et al) Cluster notion defined by internal density and external sparsity. Can find overlapping clusters. $\gr$ champion of a cluster: no more than $\gr |C|$ edges go out of the cluster.

Cluster among users can define community clusters and vice versa.

The (a, b) clustering notion is not suitable for clustering affiliation: actors must be connected to $b|C|$ communities: $|C|$ restricted by number of communities.

New notion: $(a_{u}, b_{u}, a_{g}, b_{g})$ cluster of $|U|$ users and $|G|$ groups with $(\gr_{u}, \gr_{g})$ champion pair. (credits: pratIk.) A small modification to .

This is not equivalent to doing (a, b) clustering after folding the affiliation network to get a user-user graph. When you fold the graph, you loose some information: $(a_{u}, b_{u}, a_{g}, b_{g})$ cluster becomes $(b_u a_g +a_{u}|G|/|U|, b_{u}b_{g})$ cluster.

Extend to use social network.

From experimental data, time required seems to be $O(n^{2})$. Make it faster, or parallelize it.

\subsubsection{Actions}
\begin{itemize}
 \item Consider previous work on co-clustering bipartite graphs.
 \subitem Read Collaborative Filtering for Orkut Communities: \href{http://www.stat.washington.edu/raftery/Research/PDF/Handcock2007.pdf}{Model-based clustering for social networks by Handcock, Raftery and Tantrum}.
 \subitem Note previous work mentioned by Prof. Dhillon.

 \item Consider previous work on clustering social networks in general. May give you ideas.
\end{itemize}

Read up on the LDA/ LSA approach:
 \subitem Read \href{http://www.cs.cmu.edu/~lafferty/pub/ctm.pdf}{Correlated topic models: David Blei and John Lafferty}.
 \subitem Read \\
{http://www.cs.cmu.edu/~cohn/papers/nips00.pdf}{The Missing Link - A Probabilistic Model of Document Content and Hypertext Connectivity: Cohn and Hofmann} : PLSA + PHits

\subsubsection{Experiments}
\begin{itemize}
\item Try different weighted merges of the user-user and user-group networks and use the graclus algorithm.

\item Try applying ITCC. vai implementation did not work. Try hyuk implementation: understand the required input files.

\item Try implementing the simple dhillon bipartite graph spectral partitioning alg.

\item Try applying the Trajan algorithm.

\item Try the algorithm where ye cluster users and clusters by iteratively folding the affiliation nw to get user nw/ community nw and then using clusters in one nw to cluster the other nw.
\end{itemize}

\subsubsection{Discuss}

\subsubsection{pragatiH}
kimapi na chakAra.

\subsection{Clustering nodes in social networks}
Cluster nodes in social networks. Use special properties of these networks to improve clustering. Try to use affiliation nw here.

credits: pratIka.

\subsection{Partitioning/ clustering bi-partite graphs in general}
Extend current graph partitioning algorithms to have soft clusters.

Try extending clustering algorithms by Brian and Trajan to co-cluster bi-partitite graphs.

credits: pratIka.



\section{AI}
Expert systems which write text books and papers to report findings to other machines and men.

Agents which learn from each other.

Vision is the current bottleneck in robotics. Bad vision ability is stopping massive robotic deployment in the world.

\section{Pattern recognition}
Understand where Naive bayes regressor works better than other regression algorithms.

\subsection{Learning theory}
Consider the problem of learning a decision tree with a 1/n correlated parity, given a noisy parity learner which only works with constant noise.

Theoretical characterization of the protein structure prediction algorithms: How accurate can a certain algorithm be?

\section{Human machine interface}
Thought to speech system.

\part{Biology}
What is the mechanism of generation of antibodies?

\section{Biostatistics}
The actual expected amount of DNA and genes shared by siblings.

\section{Biomedical engineering}
Sequence human genome within 10 minutes

\section{Neuroscience}
Neuroscience of primitive animals

\section{Structural biology}
Protein structure prediction as a Constraint satisfaction problem:
          o In iterative formulation:
                + backtracking search
          o In complete formulation:
                + Minimum conflict heuristic

Use the fact that a significant part of the protein is disordered, with the "fail first" heuristic to undertake more efficient conformation space search.

Ensuring proper sampling of the conformation space:
          o Start from the structure of a denatured protein.
          o Cotranslational folding: Fold a protein as it folds.

predict changes in shape from point mutations : use this to predict structure!

Explore the use heirarchial task network planning in finding protein structure.

Consider the peptides as robots in a swarm. Use this to predict structure.

the problem of deciding whether a molecule in one conformation can spontaneously go to another conformation.

Structural genomics: what is the optimal set of proteins to me modelled?

MBS in protein-ligand docking.

What heuristics may be used in protein structure prediction?

Map the structure of the conformation space.

\part{Mathematics}
\section{Recreational maths}


Minimum information required for deterministic sudoku gameplay.

\part{Physics}
\section{Mechanics}



Derive the principle of conservation of momentum (and energy) for a system with non conservative forces such as friction by showing that the momentum lost by the sliding body is converted to heat in the perturbed molecules where friction occurs.


A molecular mechanics explanation for elasticity?


A statistical mechanics explanation for the omni-directional nature of a liquid's pressure, though it is influenced by gravity.




% \bibliographystyle{plain}
% \bibliography{colt}

\end{document}
