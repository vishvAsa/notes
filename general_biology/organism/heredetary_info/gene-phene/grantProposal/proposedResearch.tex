% Proposed research Section.
We now briefly discuss the technical details of our plan to develop algorithms to identify gene-disease connections. In contrast to most previous research, we propose to leverage the power of informatics to exploit all the information available in solving this problem. Our approaches, by and large, focus on using multiple sources of information in novel ways that not only better explain the current observations but also help in making better predictions. We begin by describing the problem from the network link prediction perspective. We then provide varied formulations for the problem in Section \ref{section:problemFormulation}. In Section \ref{section:proposedApproaches}, we detail the various approaches that we propose to tackle the problem. In the following subsections, we discuss the evaluation strategies which are critical to measure the success of our proposed approaches. The strategies include both analytical and practical (wetlab experiments) evaluation. We conclude with a proposition of engineering tools for bioinformatics.

\paragraph*{The network link identification perspective.}
Given a phenotype (thyroid disease, for example), the problem is to identify genes affecting the expression of the phenotype. To solve this problem, one can use a variety of information sources, some of which were described in Section \ref{sec:dataDescription}.

One can view the above problem from the perspective of link identification on bipartite graphs. If one were to consider the bipartite graph composed of nodes corresponding to genes and phenotypes, the problem is to predict new links between genes and a given phenotype. Alternate sources of information, such as the gene-phenotype networks corresponding to other species, together with the information of homology relationships between genes of various species, may be similarly modelled as graphs. The question then is one of exploiting information from these graphs to predict new links between human genes and human diseases.

Such network link identification problems have been explored in the field of social network analysis \cite{KleinbergLinkPred}. This field has seen active research in recent years due to the explosion of online social networks, such as Facebook, LinkedIn and Orkut. In the social network link prediction problem, given the current state of a network among users, the task is to predict new links among these users. The community recommendation problem in social networks with explicitly defined communities of users is to accomplish a similar task in the prediction of user-community links \cite{GoogleCCF, GoogleCFLatent}.

The problem of how best to use information from secondary networks in order to make predictions or identify links in the network of interest (the human gene-phenotype network, in our case) is of critical importance in this project. PI Dhillon's research group has conducted concerted research in this direction, in the context of link prediction in social networks\cite{berkantSupervised} and affiliation networks \cite{vasukiNatarajan}.

\subsection{Varied problem formulations}
\label{section:problemFormulation}
We have described and modeled the problem using network analysis methodologies. In this section, we present a few different problem formulations.

\subsubsection{Formulation as a ranking problem}
\label{forumulationRankingProblem}
The problem of identifying genes associated with a certain disease may be viewed as one of ranking genes. For a given phenotype, the task of a gene-phenotype link predictor can be viewed as one of ranking genes in the order of their likelihood of influencing the phenotype.

Ranking a set of items in the order of relevance is a central problem in the field of recommender systems. Consider the famous example of the Netflix competition \cite{yehudaMillionDollar}. In this problem, ratings given to some movies by some users is known. Based on this information, for a particular user, the task is to sort all movies according to how much a user will like these movies. This information can then be used to recommend movies to users, which plays an important role in the movie rental company's revenue. For our purposes, phenotypes or diseases may be considered to play the role of users, whereas genes may be considered to play the role of movies.

This perspective is very convenient because most approaches to identifying \textit{relevant} items function using scores, a concept we describe next. A relevance function $relevant:\set{items} \to \set{0, 1}$ is a binary function which is 1 for any \textit{relevant} item, and 0 otherwise. In practice, for many problems including the problem of identifying genes relevant to a certain phenotype, this relevance function is unknown. Many algorithms designed for identifying relevant items function by assigning \textit{relevance scores} to every candidate item. Then, for a threshold score, one can identify all items with scores higher than the threshold to be 'relevant'.

\subsubsection{Probabilistic formulations}
\label{section:outlier}
One can also attempt to solve the problem of identifying genes relevant to a particular phenotype by modeling $Pr(relevant(g|p))$, the probability of gene-phenotype associations. These probabilities may then be used to play the role of the \textit{relevance scores} described earlier. This approach has been taken by Marcotte et al \cite{McGaryOrthologousPhenotypes} with some reasonable early success.

A related formulation is one of one-class classification \cite{Scholkopf99, ProteinOneClass, OutlierDetection, VillalbaOneClass}. The problem here, given a set of examples belonging to a certain class, is to identify whether a previously unseen item belongs to that class. This problem may also be viewed as one of determining the support of a distribution (which is $Pr(relevant(g|p))$ in our case). The problem can also be formulated as one of learning the graphical model G describing a probability distribution over the vector space spanned by all human genes. Various human diseases are then viewed as partially observed samples from this distribution. The task then would be to learn the graphical model G and model the gene-phenotype prediction problem as one of inferring conditional probabilities in G.

These different ways of modeling the problem suggest various approaches to solving the gene-disease link identification problem, which are described next.

\subsection{Proposed Approaches}
\label{section:proposedApproaches}
We now describe different methods for tackling the varied problem formulations discussed in Section \ref{section:problemFormulation}. In the remainder of the section, we will use $genes(Hs)$ to represent the set of human genes, and $phenotypes(Hs)$ to represent the set of human diseases under consideration.

\subsubsection{Matrix completion approaches}
\label{section:matrixComp}
Consider the ranking formulation of the gene-phenotype link identification problem described in Section \ref{forumulationRankingProblem}. Recall that the task, for a given phenotype $p$, may be viewed as one of ranking genes in the order of their likelihood of influencing $p$. As mentioned earlier, many algorithms designed for identifying relevant items function by assigning \textit{relevance scores} to every candidate item. In other words, every pair $(g, p) \in genes(Hs) \times phenotypes(Hs)$ is assigned a score $score(g,p) \in \Re$.

For gene-disease connections that are already known, denoted by the set $\Omega$, suppose that $score(g, p) = 1$. The scoring problem is then to assign scores to $\set{(g, p) \in genes(Hs) \times phenotypes(Hs) - \Omega}$. The closer score(g, p) is to exceeding or equalling 1, the more confidence one has about there being a connection between g and $p$.

The problem can be viewed as a \textbf{matrix completion problem} in the following manner. Form a \\
$\size{genes(Hs)} \times \size{phenotypes(Hs)}$ matrix S, in which only entries $\set{S_{g, p} = 1 : \forall (g, p) \in \Omega}$ are known. Note that one can think of S as a part of the adjacency matrix corresponding to the network formed by gene-disease interactions. The task is to estimate the remaining entries of S, so that the estimated values are meaningful as scores which can be used for gene-phenotype link prediction.

If one were to further assume that all genes and phenotypes have a low rank representation, one can then model the problem as one of \textbf{low rank matrix approximation}. Here, one imposes the model $\bS \approx \bG*\bP^{T}$, where G and P have dimensions $\size{genes(Hs)} \times k, \size{phenotypes(Hs)} \times k$ for some small rank $k$. So, one tries to find a low rank approximation for $\bS$.

Here, the modeling assumption is that, for a gene-phenotype pair $(g, p)$, $\dprod{v_g, v_p} \approx S_{g, p}$, where $v_g$ and $v_p$ are low dimensional representations of the gene g and the phenotype $p$ respectively. This assumption is appealing to the intuition. For example, in the Netflix movie recommendation example, the low dimensional representation of movies and users may be interpreted as having meaning related to various features of a movie or a person's taste, like the theme, the music and so on. This approach has been enormously influential in recommender systems research, and the PI has developed fast algorithms for such problems \cite{PrateekSVP}.

Alternate sources of information such as the gene-phenotype networks from other species, and the gene-gene orthologous relationships can also be viewed as incomplete matrices. One may then extend matrix completion techniques to discern low dimensional representations of genes and phenotypes which are good at accounting for these secondary observations, besides accounting for observed entries of S. The idea here is that such low dimensional representations of genes and phenotypes better reflect reality, and have greater predictive ability. Thus the matrix completion problem itself is a novel extension of the classical problem: infer the missing values given secondary data sources. Preliminary results are reported in Section \ref{sec:preliminaryResearch}.

We now describe other mathematical challenges that arise in the matrix completion formulation. Let $S$ be the adjacency matrix corresponding to a network, and suppose that only some entries $\Omega$ of $S$ are observed. Also, suppose that we want to derive a low rank approximation $X$ of $S$. Firstly, $S$ is binary matrix. Secondly, most of the unknown entries in $S$ are expected to be zero: so, a good predictor must have a heavy bias towards predicting that the unknown entries are zero. Thirdly, all the known entries in $S$ are 1. Matrix completion for binary matrices, especially for the purpose of binary classification presents challenges that have not been adequately addressed in the literature. For example, while using mathematical programming to find X to approximate S, one classical way to penalize deviation from the known entries in S is to use the quadratic penalty, $\sum_{(i, j) \in \Omega} (S_{i, j} - X_{i,j})^{2}$. Considering the fact that we are dealing with a binary matrix, using a hinge loss or exponential loss may be more appropriate. Furthermore, the requirement that most inferred entries should be zero implies a sparsity structure on $X$, which may be enforced using an L1-norm penalty function. Combining the L1-norm penalty function with the trace norm regularization would be a challenging optimization problem (which is different from the low-rank + outlier problems considered in \cite{SanghaviChandrasekaran,candesRPCA09}). The gene-gene network would additionally enforce smoothness in the low-rank, sparse approximation to be computed. The proposed research, besides tackling the biological problem of gene-disease link prediction will also involve the development of general techniques to solve such novel mathematical problems.

\subsubsection{Graph proximity approaches}
\label{section:graphProximityIntro}
An alternative way of modeling connections between genes and phenotypes can be obtained from looking at the problem from a more graph-theoretic perspective. Consider the nodes in a graph --- various ways of measuring similarity between nodes have been proposed and tested in the context of social networks analysis\cite{KleinbergLinkPred}. The basic idea is that, the more \textit{similar} two nodes are, the greater the probability of a link arising between those nodes.

Consider the adjacency matrix A of a graph. One well known similarity measure between nodes in a graph is the \textit{common neighbors} measure, where the number of neighbors shared by a pair of nodes is used as a similarity measure. This similarity measure between nodes may be concisely expressed using the matrix $S = A^{2}$, where $S_{i, j}$ denotes the similarity between nodes $i$ and $j$. $S_{i,j}$ can be interpreted as counting the number of paths of length two between nodes $i$ and $j$. Similarly, similarity measures can be defined to take into account paths of arbitrary length. For example, the Katz measure, defined by $S(\gb) = \sum_{i=1}^{\infty} \gb^{i}A^{i}$. The utility of such similarity measures in link prediction on social networks has been an active area of research. The PI has recently worked on producing better similarity measures which assign weights that can be learned using supervised approaches\cite{berkantSupervised}, which are described later in this section.

Such notions of node similarity can be extended to the case of bipartite graphs of genes and phenotypes. A challenge of using this approach is to find a way of exploiting secondary sources of information, such as the gene-phenotype interaction networks in other species, and the homology information between genes of various species. Preliminary work in this direction is reported in Section \ref{sec:preliminaryResearch}.

\paragraph*{Supervised graph proximity methods.}
Another way of estimating new proximities between genes and phenes is to extend the above graph proximity methods into a supervised framework. For instance, if we adopt a linear regression model, a linear function by combining various path counting features (extracted from gene-phene bipartite graphs based on other species) can be used to approximate the target gene-phene bipartite graph. The model parameters, i.e., weights for different path counting features, can be obtained adaptively through linear regression by fitting the data. More sophisticated models such as generalized linear models can be applied by extending the linear function to the nonlinear case. For example, we can transform the output of the linear model through a logistic function. This will lead to the logistic regression model, which can capture more complex intrinsic properties of the data than the linear regression model. 

Multiple gene-phene interactions in species other than humans can be naturally treated as multiple sources of information characterizing various connections between genes and phenes. It would be interesting to find out whether gene-phene interactions in multiple species can collectively predict potential gene-phene connections in human diseases. With multiple proximity graphs, we have a much richer choice of graph topological features (hybrid path counting features) by allowing cross routes between graphs. Let us take an example for better understanding. Note that multiple gene-gene connections can be formed based on gene-phene networks from multiple species. In the context of \emph{multigraphs}\cite{harary94}, between any two genes there can be an edge from source $A$ and an edge from source $B$. Both $A$ and $B$ are in the form of bipartite graphs. By traversing edges in both $A$ and $B$, we can construct hybrid path counting features across multiple sources in addition to pure path counting features within just one source. To control the model complexity due to the increase of the number of path counting features, we can enforce some sparsity constraints on the feature weights through L1 regularization\cite{lasso}. This will lead to the Lasso or the grouped Lasso problem (following the hierarchical structure formed by multiple sources)\cite{jenatton09,bach08,zhao09}. In the spirit of selectively combining hybrid path counting features from multiple sources, the PI's group has sucessfully applied the idea to the link prediction problem in social network analysis, which yields promising results on real world application data\cite{berkantSupervised}.  


\subsubsection{Other approaches}
We also propose to explore various other novel approaches to solving the problem. In particular, we plan to explore the use of probabilistic modeling of the relationships between genes and phenotypes. Consider the matrix completion approaches described in Section \ref{section:matrixComp}, for example, where the objective was to compute a low rank approximation $\bA$ to an incomplete matrix $\bS$. One can design a statistical model parameterized by a low rank matrix $\bA$ for the process generating $\bS$ and then learn the model parameters which are most likely to have generated the observations in $\bS$.

In Section \ref{section:outlier}, we formulated the gene-phenotype link identification problem as one of one-class classification. Here, previously known gene-phenotype connections serve as examples coming from the distribution D of ``positive examples''. But, there are no ``negative examples'', or cases where connections between genes and phenotypes are certainly known not to exist. The application of one-class classification to the problem of link prediction would also be explored as part of the proposed research.

In Section \ref{section:outlier}, we also formulated the problem as one of learning a graphical model describing a probability distribution $P$ over a binary vector space spanned by the set of human genes. Prior gene-disease links, for a given disease, are viewed as samples drawn from this distribution. This task presents a new challenge in research about probabilistic graphical models --- generate a model so that most conditional probabilities are close to zero, reflecting the belief that most genes do not contribute to a human disease. The proposed research will involve tackling this theoretical problem and applying it to the gene-disease link prediction.

In the context of link prediction and recommender systems, research has shown that a combination of many predictors is capable of better performance than any single predictor. For example, the winners in the recent Netflix competition used a non-trivial combination of many predictors. The task of identifying connections between genes and diseases is one of great importance. In proposed research, we plan to employ such a combination of predictors in identifying gene-disease relationships.

\subsection{Identifying new information sources}
A predictor is only as good as the information it gets, and the more information that is available, the greater our freedom to develop powerful predictors which exploit this information. As part of the proposed research, we plan to explore the use of new sources of information in addition to the data described in Section \ref{section:dataset}. There are other interesting sources of information like the gene-gene ortholog information. An orthologous relationship is a mapping from the subset of human genes to a subset of genes of other species. The mapping is many-to-many, i.e., one or more human genes can be mapped to the same ortholog or a human gene can be mapped to more than one ortholog. This relationship is represented as a binary matrix. This information can potentially be used for link prediction in biological network, and has not been exploited by any previous work on this problem. Also, there is the interesting aspect of determining association between phenotypes. These associations can be derived from from the medical literature or from medical records of patients, e.g., if some diseases are known to co-occur, it is suggestive of a relationship between them.


\subsection{Evaluation Strategies}
\label{section:evaluation}
Evaluation is critical and complementary to the development of novel methods. Quantifying the success of the proposed techniques presents a challenge by itself. Theoretical evaluation of the methods is one way of measuring the goodness of the results. However, the nature of the problem at hand calls for direct confirmation of the results from biologists. Our goal is to use analytical evaluation strategies to measure the performance of the individual methods proposed, while ultimately enabling wetlab experiments to confirm the role of predicted genes in human diseases of concern. Considering the costly and time consuming nature of wet-lab experiments, development of good evaluation techniques is essential for identification of good predictors. This is also important in combining predictions from multiple predictors: one needs to assign greater weight to predictions from good predictors, and lesser weight to predictions from mediocre predictors.

\subsubsection{Analytical evaluation}
Consider a predictor whose task is to identify the set of genes $T$ that are known to be associated with certain diseases. The predictor comes up with a ranking of the genes with higher scores assigned to genes, it believes to have higher likelihood of influencing the diseases. We need to measure how good the prediction is. Our evaluation methodology uses area under ROC curve as a measure of performance of the predictor. To understand the evaluation methodology, we need to define two terms: sensitivity and specificity. Let $P_n$ denote a set of $n$ genes predicted by a predictor for a given phenotype.

\paragraph*{Sensitivity and Specificity.}
Sensitivity measures the ability of the predictor to identify genes in $T$:

$$\emph{Sensitivity} = \frac{|P_n \cap T|}{|T|}. $$

Let $U$ refer to the set of human genes not in T, i.e., $U = genes(Hs) - T$. Specificity measures the ability of the predictor to exclude genes not in $T$:

$$\emph{Specificity} = \frac{|(U - P_n) \cap (U - T)|}{|U-T|}.$$

\paragraph*{ROC, AUC and average AUC.}
The Receiver Operating Characteristics (ROC) curve is the plot of sensitivity vs (1-specificity) for varying $n$. The area under this ROC curve denoted by AUC is a measure of overall performance of the predictor. The performance of a predictor is evaluated using AUC for different human diseases and the average performance of the predictor is determined by the average of all the AUC values. This average AUC $\in [0,1]$ can be used as a performance measure used to compare the performance of different predictors.

For the purpose of wetlab experimentation, however, only the top, say fifty, recommendations produced by a predictor matter. So, ideally we would want to evaluate predictors based on how good their top fifty recommendations are, rather than evaluating them over the entire range of $n$. We can do this by examining a slice of the ROC curve formed by measuring the sensitivity and specificity the predictor achieves for an average disease at regular intervals between $n = 1$ and $n = 50$, an evaluation strategy we propose in \cite{vasukiNatarajan}.

\subsubsection{Enabling wetlab experiments}
After all analytical validation, the ultimate test of the predictions produced by the techniques, of course, will be done by biologists. Considering the recent success of analytical modeling in identifying gene-disease links \cite{McGaryOrthologousPhenotypes}, we expect to discover many new connections between genes and diseases.

Bioinformatics has proven to be of enormous importance in helping biologists cope with the deluge of data they now face during their explorations. Given predictions made with different levels of confidence, a natural question to ask is one of how many resources one can allocate in investigating a predicted gene-disease link in wet-lab experiments. Wet-lab experiments generally tend to be costly and time consuming, making this a question of great importance. This question, which comes under the ambit of Decision Theory, is one we plan to explore in the final phase of this project.

\subsection{Development of bioinformatics tools}
A long term objective of this project is to impact the art of discovering gene-phenotype associations in cases beyond the dataset described in Section \ref{section:dataset}. Our knowledge of gene-phenotype interactions is constantly expanding; this yields new information which can be exploited for making predictions about gene-disease associations many years down the line. For this kind of broad applicability and impact of the methods we develop as part of this project, the development of general software tools is essential.

Accordingly, we plan to develop and release software packages for use by biologists, with algorithms which will identify gene-phenotype associations using the input they are fed. Besides gene-phenotype link prediction software, we also plan to release ancillary tools we develop as part of this research, for example, software to visualize the various gene-phenotype networks.
