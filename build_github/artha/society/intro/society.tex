\documentclass[oneside, article]{memoir}

\usepackage{amsmath, amssymb}
\usepackage{graphicx, verbatim, listings, multirow, subfigure, paralist}
% \usepackage{algorithm, algorithmic}
% \usepackage[bottom]{footmisc}
\lstset{breaklines=true}
\setcounter{tocdepth}{3}

% Lets verbatim and verb environments automatically break lines.
\makeatletter
\def\@xobeysp{ }
\makeatother
% \lstset{breaklines=true,basicstyle=\ttfamily}

\input{../packagesMemoir}
% Use something like:
% % Use something like:
% % Use something like:
% \input{../../macros}

% groupings of objects.
\newcommand{\set}[1]{\left\{ #1 \right\}}
\newcommand{\seq}[1]{\left(#1\right)}
\newcommand{\ang}[1]{\langle#1\rangle}
\newcommand{\tuple}[1]{\left(#1\right)}
\newcommand{\size}[1]{\left| #1\right|}

% numerical shortcuts.
\newcommand{\abs}[1]{\left| #1\right|}
\newcommand{\floor}[1]{\left\lfloor #1 \right\rfloor}
\newcommand{\ceil}[1]{\left\lceil #1 \right\rceil}

% linear algebra shortcuts.
\newcommand{\change}{\Delta}
\newcommand{\norm}[1]{\left\| #1\right\|}
\newcommand{\dprod}[1]{\langle#1\rangle}
\newcommand{\linspan}[1]{\langle#1\rangle}
\newcommand{\conj}[1]{\overline{#1}}
\newcommand{\der}{\frac{d}{dx}}
\newcommand{\lap}{\Delta}
\newcommand{\kron}{\otimes}
\newcommand{\nperp}{\nvdash}

\newcommand{\mat}[1]{\left[ \begin{smallmatrix}#1 \end{smallmatrix} \right]}

% derivatives and limits
\newcommand{\partder}[2]{\frac{\partial #1}{\partial #2}}
\newcommand{\partdern}[3]{\frac{\partial^{#3} #1}{\partial #2^{#3}}}
\newcommand{\gradient}{\nabla}
\newcommand{\subgradient}{\partial}

% Arrows
\newcommand{\diverge}{\nearrow}
\newcommand{\notto}{\nrightarrow}
\newcommand{\up}{\uparrow}
\newcommand{\down}{\downarrow}
% gets and gives are defined!

% ordering operators
\newcommand{\oleq}{\preceq}
\newcommand{\ogeq}{\succeq}

% programming and logic operators
\newcommand{\dfn}{:=}
\newcommand{\assign}{:=}
\newcommand{\co}{\ co\ }
\newcommand{\en}{\ en\ }


% logic operators
\newcommand{\xor}{\oplus}
\newcommand{\Land}{\bigwedge}
\newcommand{\Lor}{\bigvee}
\newcommand{\finish}{$\Box$}
\newcommand{\contra}{\ensuremath{\Rightarrow \Leftarrow}}
\newcommand{\iseq}{\stackrel{_?}{=}}


% Set theory
\newcommand{\symdiff}{\ensuremath{\Delta}}
\newcommand{\union}{\ensuremath{\cup}}
\newcommand{\inters}{\ensuremath{\cap}}
\newcommand{\Union}{\ensuremath{\bigcup}}
\newcommand{\Inters}{\ensuremath{\bigcap}}
\newcommand{\nullSet}{\ensuremath{\phi}}


% graph theory
\newcommand{\nbd}{\Gamma}

% Script alphabets
% For reals, use \Re

% greek letters
\newcommand{\eps}{\ensuremath{\epsilon}}
\newcommand{\del}{\ensuremath{\delta}}
\newcommand{\ga}{\ensuremath{\alpha}}
\newcommand{\gb}{\ensuremath{\beta}}
\newcommand{\gd}{\ensuremath{\del}}
\newcommand{\gp}{\ensuremath{\pi}}
\newcommand{\gf}{\ensuremath{\phi}}
\newcommand{\gF}{\ensuremath{\Phi}}
\newcommand{\gl}{\ensuremath{\lambda}}
\newcommand{\gm}{\ensuremath{\mu}}
\newcommand{\gn}{\ensuremath{\nu}}
\newcommand{\gr}{\ensuremath{\rho}}
\newcommand{\gs}{\ensuremath{\sigma}}
\newcommand{\gt}{\ensuremath{\theta}}
\newcommand{\gx}{\ensuremath{\xi}}

\newcommand{\sw}{\ensuremath{\sigma}}
\newcommand{\SW}{\ensuremath{\Sigma}}
\newcommand{\ew}{\ensuremath{\lambda}}
\newcommand{\EW}{\ensuremath{\Lambda}}

\newcommand{\Del}{\ensuremath{\Delta}}
\newcommand{\gD}{\ensuremath{\Delta}}
\newcommand{\gG}{\ensuremath{\Gamma}}
\newcommand{\gO}{\ensuremath{\Omega}}
\newcommand{\gS}{\ensuremath{\Sigma}}

% Bold english letters.
\newcommand{\bA}{\ensuremath{\mathbf{A}}}
\newcommand{\bB}{\ensuremath{\mathbf{B}}}
\newcommand{\bC}{\ensuremath{\mathbf{C}}}
\newcommand{\bD}{\ensuremath{\mathbf{D}}}
\newcommand{\bE}{\ensuremath{\mathbf{E}}}
\newcommand{\bF}{\ensuremath{\mathbf{F}}}
\newcommand{\bG}{\ensuremath{\mathbf{G}}}
\newcommand{\bH}{\ensuremath{\mathbf{H}}}
\newcommand{\bI}{\ensuremath{\mathbf{I}}}
\newcommand{\bJ}{\ensuremath{\mathbf{J}}}
\newcommand{\bK}{\ensuremath{\mathbf{K}}}
\newcommand{\bL}{\ensuremath{\mathbf{L}}}
\newcommand{\bM}{\ensuremath{\mathbf{M}}}
\newcommand{\bN}{\ensuremath{\mathbf{N}}}
\newcommand{\bO}{\ensuremath{\mathbf{O}}}
\newcommand{\bP}{\ensuremath{\mathbf{P}}}
\newcommand{\bQ}{\ensuremath{\mathbf{Q}}}
\newcommand{\bR}{\ensuremath{\mathbf{R}}} % PROSPER defines \R
\newcommand{\bS}{\ensuremath{\mathbf{S}}}
\newcommand{\bT}{\ensuremath{\mathbf{T}}}
\newcommand{\bU}{\ensuremath{\mathbf{U}}}
\newcommand{\bV}{\ensuremath{\mathbf{V}}}
\newcommand{\bW}{\ensuremath{\mathbf{W}}}
\newcommand{\bX}{\ensuremath{\mathbf{X}}}
\newcommand{\bY}{\ensuremath{\mathbf{Y}}}
\newcommand{\bZ}{\ensuremath{\mathbf{Z}}}
\newcommand{\ba}{\ensuremath{\mathbf{a}}}
\newcommand{\bb}{\ensuremath{\mathbf{b}}}
\newcommand{\bc}{\ensuremath{\mathbf{c}}}
\newcommand{\bd}{\ensuremath{\mathbf{d}}}
\newcommand{\be}{\ensuremath{\mathbf{e}}}
\newcommand{\bbf}{\ensuremath{\mathbf{f}}}
\newcommand{\bg}{\ensuremath{\mathbf{g}}}
\newcommand{\bh}{\ensuremath{\mathbf{h}}}
\newcommand{\bk}{\ensuremath{\mathbf{k}}}
\newcommand{\bl}{\ensuremath{\mathbf{l}}}
\newcommand{\bm}{\ensuremath{\mathbf{m}}}
\newcommand{\bn}{\ensuremath{\mathbf{n}}}
\newcommand{\bp}{\ensuremath{\mathbf{p}}}
\newcommand{\bq}{\ensuremath{\mathbf{q}}}
\newcommand{\br}{\ensuremath{\mathbf{r}}}
\newcommand{\bs}{\ensuremath{\mathbf{s}}}  % TIPA defines \s and LaTeX \ss!
\newcommand{\bt}{\ensuremath{\mathbf{t}}}
\newcommand{\bu}{\ensuremath{\mathbf{u}}}
\newcommand{\bv}{\ensuremath{\mathbf{v}}}
\newcommand{\bw}{\ensuremath{\mathbf{w}}}
\newcommand{\bx}{\ensuremath{\mathbf{x}}}
\newcommand{\by}{\ensuremath{\mathbf{y}}}
\newcommand{\bz}{\ensuremath{\mathbf{z}}}
\newcommand{\0}{\ensuremath{\mathbf{0}}}
\newcommand{\1}{\ensuremath{\mathbf{1}}}


% Formatting shortcuts
\newcommand{\red}[1]{\textcolor{red}{#1}}
\newcommand{\blue}[1]{\textcolor{blue}{#1}}
\newcommand{\htext}[2]{\texorpdfstring{#1}{#2}}

% Statistics
\newcommand{\distr}{\ensuremath{\sim}}
\newcommand{\stddev}{\ensuremath{\sigma}}
\newcommand{\covmatrix}{\ensuremath{\Sigma}}
\newcommand{\mean}{\ensuremath{\mu}}
\newcommand{\param}{\ensuremath{\gt}}
\newcommand{\ftr}{\ensuremath{\phi}}

% General utility
\newcommand{\todo}[1]{\footnote{TODO: #1}}
\newcommand{\exclaim}[1]{{\textbf{\textit{#1}}}}
\newcommand{\tbc}{[\textbf{Incomplete}]}
\newcommand{\chk}{[\textbf{Check}]}
\newcommand{\oprob}{[\textbf{OP}]:}
\newcommand{\core}[1]{\textbf{Core Idea:}}
\newcommand{\why}{[\textbf{Find proof}]}
\newcommand{\opt}[1]{\textit{#1}}


\renewcommand{\~}{\htext{$\sim$}{~}}

\DeclareMathOperator*{\argmin}{arg\,min}
\DeclareMathOperator*{\argmax}{arg\,max}




% groupings of objects.
\newcommand{\set}[1]{\left\{ #1 \right\}}
\newcommand{\seq}[1]{\left(#1\right)}
\newcommand{\ang}[1]{\langle#1\rangle}
\newcommand{\tuple}[1]{\left(#1\right)}
\newcommand{\size}[1]{\left| #1\right|}

% numerical shortcuts.
\newcommand{\abs}[1]{\left| #1\right|}
\newcommand{\floor}[1]{\left\lfloor #1 \right\rfloor}
\newcommand{\ceil}[1]{\left\lceil #1 \right\rceil}

% linear algebra shortcuts.
\newcommand{\change}{\Delta}
\newcommand{\norm}[1]{\left\| #1\right\|}
\newcommand{\dprod}[1]{\langle#1\rangle}
\newcommand{\linspan}[1]{\langle#1\rangle}
\newcommand{\conj}[1]{\overline{#1}}
\newcommand{\der}{\frac{d}{dx}}
\newcommand{\lap}{\Delta}
\newcommand{\kron}{\otimes}
\newcommand{\nperp}{\nvdash}

\newcommand{\mat}[1]{\left[ \begin{smallmatrix}#1 \end{smallmatrix} \right]}

% derivatives and limits
\newcommand{\partder}[2]{\frac{\partial #1}{\partial #2}}
\newcommand{\partdern}[3]{\frac{\partial^{#3} #1}{\partial #2^{#3}}}
\newcommand{\gradient}{\nabla}
\newcommand{\subgradient}{\partial}

% Arrows
\newcommand{\diverge}{\nearrow}
\newcommand{\notto}{\nrightarrow}
\newcommand{\up}{\uparrow}
\newcommand{\down}{\downarrow}
% gets and gives are defined!

% ordering operators
\newcommand{\oleq}{\preceq}
\newcommand{\ogeq}{\succeq}

% programming and logic operators
\newcommand{\dfn}{:=}
\newcommand{\assign}{:=}
\newcommand{\co}{\ co\ }
\newcommand{\en}{\ en\ }


% logic operators
\newcommand{\xor}{\oplus}
\newcommand{\Land}{\bigwedge}
\newcommand{\Lor}{\bigvee}
\newcommand{\finish}{$\Box$}
\newcommand{\contra}{\ensuremath{\Rightarrow \Leftarrow}}
\newcommand{\iseq}{\stackrel{_?}{=}}


% Set theory
\newcommand{\symdiff}{\ensuremath{\Delta}}
\newcommand{\union}{\ensuremath{\cup}}
\newcommand{\inters}{\ensuremath{\cap}}
\newcommand{\Union}{\ensuremath{\bigcup}}
\newcommand{\Inters}{\ensuremath{\bigcap}}
\newcommand{\nullSet}{\ensuremath{\phi}}


% graph theory
\newcommand{\nbd}{\Gamma}

% Script alphabets
% For reals, use \Re

% greek letters
\newcommand{\eps}{\ensuremath{\epsilon}}
\newcommand{\del}{\ensuremath{\delta}}
\newcommand{\ga}{\ensuremath{\alpha}}
\newcommand{\gb}{\ensuremath{\beta}}
\newcommand{\gd}{\ensuremath{\del}}
\newcommand{\gp}{\ensuremath{\pi}}
\newcommand{\gf}{\ensuremath{\phi}}
\newcommand{\gF}{\ensuremath{\Phi}}
\newcommand{\gl}{\ensuremath{\lambda}}
\newcommand{\gm}{\ensuremath{\mu}}
\newcommand{\gn}{\ensuremath{\nu}}
\newcommand{\gr}{\ensuremath{\rho}}
\newcommand{\gs}{\ensuremath{\sigma}}
\newcommand{\gt}{\ensuremath{\theta}}
\newcommand{\gx}{\ensuremath{\xi}}

\newcommand{\sw}{\ensuremath{\sigma}}
\newcommand{\SW}{\ensuremath{\Sigma}}
\newcommand{\ew}{\ensuremath{\lambda}}
\newcommand{\EW}{\ensuremath{\Lambda}}

\newcommand{\Del}{\ensuremath{\Delta}}
\newcommand{\gD}{\ensuremath{\Delta}}
\newcommand{\gG}{\ensuremath{\Gamma}}
\newcommand{\gO}{\ensuremath{\Omega}}
\newcommand{\gS}{\ensuremath{\Sigma}}

% Bold english letters.
\newcommand{\bA}{\ensuremath{\mathbf{A}}}
\newcommand{\bB}{\ensuremath{\mathbf{B}}}
\newcommand{\bC}{\ensuremath{\mathbf{C}}}
\newcommand{\bD}{\ensuremath{\mathbf{D}}}
\newcommand{\bE}{\ensuremath{\mathbf{E}}}
\newcommand{\bF}{\ensuremath{\mathbf{F}}}
\newcommand{\bG}{\ensuremath{\mathbf{G}}}
\newcommand{\bH}{\ensuremath{\mathbf{H}}}
\newcommand{\bI}{\ensuremath{\mathbf{I}}}
\newcommand{\bJ}{\ensuremath{\mathbf{J}}}
\newcommand{\bK}{\ensuremath{\mathbf{K}}}
\newcommand{\bL}{\ensuremath{\mathbf{L}}}
\newcommand{\bM}{\ensuremath{\mathbf{M}}}
\newcommand{\bN}{\ensuremath{\mathbf{N}}}
\newcommand{\bO}{\ensuremath{\mathbf{O}}}
\newcommand{\bP}{\ensuremath{\mathbf{P}}}
\newcommand{\bQ}{\ensuremath{\mathbf{Q}}}
\newcommand{\bR}{\ensuremath{\mathbf{R}}} % PROSPER defines \R
\newcommand{\bS}{\ensuremath{\mathbf{S}}}
\newcommand{\bT}{\ensuremath{\mathbf{T}}}
\newcommand{\bU}{\ensuremath{\mathbf{U}}}
\newcommand{\bV}{\ensuremath{\mathbf{V}}}
\newcommand{\bW}{\ensuremath{\mathbf{W}}}
\newcommand{\bX}{\ensuremath{\mathbf{X}}}
\newcommand{\bY}{\ensuremath{\mathbf{Y}}}
\newcommand{\bZ}{\ensuremath{\mathbf{Z}}}
\newcommand{\ba}{\ensuremath{\mathbf{a}}}
\newcommand{\bb}{\ensuremath{\mathbf{b}}}
\newcommand{\bc}{\ensuremath{\mathbf{c}}}
\newcommand{\bd}{\ensuremath{\mathbf{d}}}
\newcommand{\be}{\ensuremath{\mathbf{e}}}
\newcommand{\bbf}{\ensuremath{\mathbf{f}}}
\newcommand{\bg}{\ensuremath{\mathbf{g}}}
\newcommand{\bh}{\ensuremath{\mathbf{h}}}
\newcommand{\bk}{\ensuremath{\mathbf{k}}}
\newcommand{\bl}{\ensuremath{\mathbf{l}}}
\newcommand{\bm}{\ensuremath{\mathbf{m}}}
\newcommand{\bn}{\ensuremath{\mathbf{n}}}
\newcommand{\bp}{\ensuremath{\mathbf{p}}}
\newcommand{\bq}{\ensuremath{\mathbf{q}}}
\newcommand{\br}{\ensuremath{\mathbf{r}}}
\newcommand{\bs}{\ensuremath{\mathbf{s}}}  % TIPA defines \s and LaTeX \ss!
\newcommand{\bt}{\ensuremath{\mathbf{t}}}
\newcommand{\bu}{\ensuremath{\mathbf{u}}}
\newcommand{\bv}{\ensuremath{\mathbf{v}}}
\newcommand{\bw}{\ensuremath{\mathbf{w}}}
\newcommand{\bx}{\ensuremath{\mathbf{x}}}
\newcommand{\by}{\ensuremath{\mathbf{y}}}
\newcommand{\bz}{\ensuremath{\mathbf{z}}}
\newcommand{\0}{\ensuremath{\mathbf{0}}}
\newcommand{\1}{\ensuremath{\mathbf{1}}}


% Formatting shortcuts
\newcommand{\red}[1]{\textcolor{red}{#1}}
\newcommand{\blue}[1]{\textcolor{blue}{#1}}
\newcommand{\htext}[2]{\texorpdfstring{#1}{#2}}

% Statistics
\newcommand{\distr}{\ensuremath{\sim}}
\newcommand{\stddev}{\ensuremath{\sigma}}
\newcommand{\covmatrix}{\ensuremath{\Sigma}}
\newcommand{\mean}{\ensuremath{\mu}}
\newcommand{\param}{\ensuremath{\gt}}
\newcommand{\ftr}{\ensuremath{\phi}}

% General utility
\newcommand{\todo}[1]{\footnote{TODO: #1}}
\newcommand{\exclaim}[1]{{\textbf{\textit{#1}}}}
\newcommand{\tbc}{[\textbf{Incomplete}]}
\newcommand{\chk}{[\textbf{Check}]}
\newcommand{\oprob}{[\textbf{OP}]:}
\newcommand{\core}[1]{\textbf{Core Idea:}}
\newcommand{\why}{[\textbf{Find proof}]}
\newcommand{\opt}[1]{\textit{#1}}


\renewcommand{\~}{\htext{$\sim$}{~}}

\DeclareMathOperator*{\argmin}{arg\,min}
\DeclareMathOperator*{\argmax}{arg\,max}




% groupings of objects.
\newcommand{\set}[1]{\left\{ #1 \right\}}
\newcommand{\seq}[1]{\left(#1\right)}
\newcommand{\ang}[1]{\langle#1\rangle}
\newcommand{\tuple}[1]{\left(#1\right)}
\newcommand{\size}[1]{\left| #1\right|}

% numerical shortcuts.
\newcommand{\abs}[1]{\left| #1\right|}
\newcommand{\floor}[1]{\left\lfloor #1 \right\rfloor}
\newcommand{\ceil}[1]{\left\lceil #1 \right\rceil}

% linear algebra shortcuts.
\newcommand{\change}{\Delta}
\newcommand{\norm}[1]{\left\| #1\right\|}
\newcommand{\dprod}[1]{\langle#1\rangle}
\newcommand{\linspan}[1]{\langle#1\rangle}
\newcommand{\conj}[1]{\overline{#1}}
\newcommand{\der}{\frac{d}{dx}}
\newcommand{\lap}{\Delta}
\newcommand{\kron}{\otimes}
\newcommand{\nperp}{\nvdash}

\newcommand{\mat}[1]{\left[ \begin{smallmatrix}#1 \end{smallmatrix} \right]}

% derivatives and limits
\newcommand{\partder}[2]{\frac{\partial #1}{\partial #2}}
\newcommand{\partdern}[3]{\frac{\partial^{#3} #1}{\partial #2^{#3}}}
\newcommand{\gradient}{\nabla}
\newcommand{\subgradient}{\partial}

% Arrows
\newcommand{\diverge}{\nearrow}
\newcommand{\notto}{\nrightarrow}
\newcommand{\up}{\uparrow}
\newcommand{\down}{\downarrow}
% gets and gives are defined!

% ordering operators
\newcommand{\oleq}{\preceq}
\newcommand{\ogeq}{\succeq}

% programming and logic operators
\newcommand{\dfn}{:=}
\newcommand{\assign}{:=}
\newcommand{\co}{\ co\ }
\newcommand{\en}{\ en\ }


% logic operators
\newcommand{\xor}{\oplus}
\newcommand{\Land}{\bigwedge}
\newcommand{\Lor}{\bigvee}
\newcommand{\finish}{$\Box$}
\newcommand{\contra}{\ensuremath{\Rightarrow \Leftarrow}}
\newcommand{\iseq}{\stackrel{_?}{=}}


% Set theory
\newcommand{\symdiff}{\ensuremath{\Delta}}
\newcommand{\union}{\ensuremath{\cup}}
\newcommand{\inters}{\ensuremath{\cap}}
\newcommand{\Union}{\ensuremath{\bigcup}}
\newcommand{\Inters}{\ensuremath{\bigcap}}
\newcommand{\nullSet}{\ensuremath{\phi}}


% graph theory
\newcommand{\nbd}{\Gamma}

% Script alphabets
% For reals, use \Re

% greek letters
\newcommand{\eps}{\ensuremath{\epsilon}}
\newcommand{\del}{\ensuremath{\delta}}
\newcommand{\ga}{\ensuremath{\alpha}}
\newcommand{\gb}{\ensuremath{\beta}}
\newcommand{\gd}{\ensuremath{\del}}
\newcommand{\gp}{\ensuremath{\pi}}
\newcommand{\gf}{\ensuremath{\phi}}
\newcommand{\gF}{\ensuremath{\Phi}}
\newcommand{\gl}{\ensuremath{\lambda}}
\newcommand{\gm}{\ensuremath{\mu}}
\newcommand{\gn}{\ensuremath{\nu}}
\newcommand{\gr}{\ensuremath{\rho}}
\newcommand{\gs}{\ensuremath{\sigma}}
\newcommand{\gt}{\ensuremath{\theta}}
\newcommand{\gx}{\ensuremath{\xi}}

\newcommand{\sw}{\ensuremath{\sigma}}
\newcommand{\SW}{\ensuremath{\Sigma}}
\newcommand{\ew}{\ensuremath{\lambda}}
\newcommand{\EW}{\ensuremath{\Lambda}}

\newcommand{\Del}{\ensuremath{\Delta}}
\newcommand{\gD}{\ensuremath{\Delta}}
\newcommand{\gG}{\ensuremath{\Gamma}}
\newcommand{\gO}{\ensuremath{\Omega}}
\newcommand{\gS}{\ensuremath{\Sigma}}

% Bold english letters.
\newcommand{\bA}{\ensuremath{\mathbf{A}}}
\newcommand{\bB}{\ensuremath{\mathbf{B}}}
\newcommand{\bC}{\ensuremath{\mathbf{C}}}
\newcommand{\bD}{\ensuremath{\mathbf{D}}}
\newcommand{\bE}{\ensuremath{\mathbf{E}}}
\newcommand{\bF}{\ensuremath{\mathbf{F}}}
\newcommand{\bG}{\ensuremath{\mathbf{G}}}
\newcommand{\bH}{\ensuremath{\mathbf{H}}}
\newcommand{\bI}{\ensuremath{\mathbf{I}}}
\newcommand{\bJ}{\ensuremath{\mathbf{J}}}
\newcommand{\bK}{\ensuremath{\mathbf{K}}}
\newcommand{\bL}{\ensuremath{\mathbf{L}}}
\newcommand{\bM}{\ensuremath{\mathbf{M}}}
\newcommand{\bN}{\ensuremath{\mathbf{N}}}
\newcommand{\bO}{\ensuremath{\mathbf{O}}}
\newcommand{\bP}{\ensuremath{\mathbf{P}}}
\newcommand{\bQ}{\ensuremath{\mathbf{Q}}}
\newcommand{\bR}{\ensuremath{\mathbf{R}}} % PROSPER defines \R
\newcommand{\bS}{\ensuremath{\mathbf{S}}}
\newcommand{\bT}{\ensuremath{\mathbf{T}}}
\newcommand{\bU}{\ensuremath{\mathbf{U}}}
\newcommand{\bV}{\ensuremath{\mathbf{V}}}
\newcommand{\bW}{\ensuremath{\mathbf{W}}}
\newcommand{\bX}{\ensuremath{\mathbf{X}}}
\newcommand{\bY}{\ensuremath{\mathbf{Y}}}
\newcommand{\bZ}{\ensuremath{\mathbf{Z}}}
\newcommand{\ba}{\ensuremath{\mathbf{a}}}
\newcommand{\bb}{\ensuremath{\mathbf{b}}}
\newcommand{\bc}{\ensuremath{\mathbf{c}}}
\newcommand{\bd}{\ensuremath{\mathbf{d}}}
\newcommand{\be}{\ensuremath{\mathbf{e}}}
\newcommand{\bbf}{\ensuremath{\mathbf{f}}}
\newcommand{\bg}{\ensuremath{\mathbf{g}}}
\newcommand{\bh}{\ensuremath{\mathbf{h}}}
\newcommand{\bk}{\ensuremath{\mathbf{k}}}
\newcommand{\bl}{\ensuremath{\mathbf{l}}}
\newcommand{\bm}{\ensuremath{\mathbf{m}}}
\newcommand{\bn}{\ensuremath{\mathbf{n}}}
\newcommand{\bp}{\ensuremath{\mathbf{p}}}
\newcommand{\bq}{\ensuremath{\mathbf{q}}}
\newcommand{\br}{\ensuremath{\mathbf{r}}}
\newcommand{\bs}{\ensuremath{\mathbf{s}}}  % TIPA defines \s and LaTeX \ss!
\newcommand{\bt}{\ensuremath{\mathbf{t}}}
\newcommand{\bu}{\ensuremath{\mathbf{u}}}
\newcommand{\bv}{\ensuremath{\mathbf{v}}}
\newcommand{\bw}{\ensuremath{\mathbf{w}}}
\newcommand{\bx}{\ensuremath{\mathbf{x}}}
\newcommand{\by}{\ensuremath{\mathbf{y}}}
\newcommand{\bz}{\ensuremath{\mathbf{z}}}
\newcommand{\0}{\ensuremath{\mathbf{0}}}
\newcommand{\1}{\ensuremath{\mathbf{1}}}


% Formatting shortcuts
\newcommand{\red}[1]{\textcolor{red}{#1}}
\newcommand{\blue}[1]{\textcolor{blue}{#1}}
\newcommand{\htext}[2]{\texorpdfstring{#1}{#2}}

% Statistics
\newcommand{\distr}{\ensuremath{\sim}}
\newcommand{\stddev}{\ensuremath{\sigma}}
\newcommand{\covmatrix}{\ensuremath{\Sigma}}
\newcommand{\mean}{\ensuremath{\mu}}
\newcommand{\param}{\ensuremath{\gt}}
\newcommand{\ftr}{\ensuremath{\phi}}

% General utility
\newcommand{\todo}[1]{\footnote{TODO: #1}}
\newcommand{\exclaim}[1]{{\textbf{\textit{#1}}}}
\newcommand{\tbc}{[\textbf{Incomplete}]}
\newcommand{\chk}{[\textbf{Check}]}
\newcommand{\oprob}{[\textbf{OP}]:}
\newcommand{\core}[1]{\textbf{Core Idea:}}
\newcommand{\why}{[\textbf{Find proof}]}
\newcommand{\opt}[1]{\textit{#1}}


\renewcommand{\~}{\htext{$\sim$}{~}}

\DeclareMathOperator*{\argmin}{arg\,min}
\DeclareMathOperator*{\argmax}{arg\,max}




%opening
\title{Groups of people: survey}
\author{vishvAs vAsuki}

\begin{document}
\maketitle

\part{Introduction}
Economics of various groups of people is considered in the economics survey.

\chapter{Culture/ values}
\section{Group psyche}
purastAt api jana-samudAyAH pravRRiddhAH \\
yE kAnchana niyamAn (ityuktE saMskRRitIM) anusRRitya chalanti.\\
EtEShAM jananI cha pAlinI sAmUhikAH chitta-vRRittayaH. 

\subsection{Behavioral role}
\subsubsection{Default patterns}
Individuals - even within a single culture - show great variation in their behavior - so cultural stereotypes are inadequate to model individual actions. Yet, culture helps form the default behavior, ways of looking at the world, morals - this is especially useful for adolescents, who tend to be unstable and highly susceptible to peer pressure.

\subsubsection{Support in daily life}
They provide various cultural institutions/ customs which may be used by individuals to solve practical problems - Eg:  positive stereotypes, finding a mate (perhaps by arranged marriage/ parental introductions), habits (clothing suited for the environment).

\subsection{Mechanism and transmission}
Cultural values and behaviors may be explicitly encoded in verse, poetry and stories. Apart from these, traditions (including festivals) are vessels by which culture is impressed upon individuals. Symbols of distinctiveness and identification of cultures serve to bring the stereotypes and values into action.

\subsection{Choice}
janAH sva-vRRindasya cha pUrvajAnAM itihAsasya anusAraM uchitaM saMskRRitiM varanti - chEtasA vA achEtasA.

All countries and cultures have their noble and ignoble aspects. sarvEbhyaH api EkaH saMskAraH cha dharmaH uchitaH iti na - varitaH saMskAraH dRRiDha-mUla-sthApanE uchitaH cha yOgyaH bhavitavyaH.

\subsubsection{Modifications}
ekAyAM saMskRRityAM dRRiDha-mUla-sthApanAt anantaraM anya-saMskRRitibhyaH bhadrAH kRRitavaH yOgya-mAtrAyAM svIkartuM sAdhyaM.

\section{Conflicts among cultures}
EtEShu samRRiddhi-prAptyai viparIta-paristhitInAM \\
virOdhE (prAyaH parasparaH) kalahAH vartantE Eva|\\
kasminchit api dishAyAM cha paristhitau EtEShAM \\
sva-sadasyAnAM samRRiddhi-vardhanE balaH asamAnaH Eva|

\subsection{Destruction}
kadAchit Ekasya saMskRRitEH jayaH cha anyasya \\
vinAshaH bhavati. kEchana janAH sa-prajA \\
vijayinIM saMskRRitiM pUrvaM saMyujanti, anyE vilambEna saMyujanti| \\
pUrva-saMyuktAnAM tu samRRiddhiH nUtana-saMyuktEbhyaH apEkShayA adhikA Eva bhavati|

\section{Persistence}
\subsection{Homogenization: limited}
There are some homogenizing tendencies in world culture due to greater economic and social interaction between different people: this is aka globalization. But, its homogenizing influence is limited: as foreign direct investment is limited, immigration and even internet traffic are limited.

\chapter{Crime}
Crime is when the social contract is broken.

\section{Execution}
Factors for successful crime : desire, ability, opportunity.

\section{Rate and statistics}
Crime rate in USA declined with the introduction of contraception and legalization of abortion! A causal relationship is hypothesized.

Crime rate is correlated with gender imbalance.

\section{Law enforcement}
\subsection{Forensics}
Police visit a crime scene, collect and record evidence. Then they speculate events which might account for the evidence - the speculation may be done by professional 'profilers'.

\subsection{Arrest}
\subsubsection{Entrapment}
The aim is sometimes to prevent, rather than punish crime after the fact. Entrapment can lead to confession of plans; but it can also act as a way by which police informants encourage people, who would otherwise not be competent or coherent to do so, to plan and very nearly commit a crime.

\subsubsection{Confession}
Sometimes the accused confesses to the crime and provides details. Sometimes, when placed under duress (eg: torture), even an innocent person produces a confession!

\subsection{Trial}
The prosecution makes charges against the defendent, who can choose to be zealously defended by a lawyer (potentially appointed by the government if he cannot afford one). The trial often involves cross examination of witnesses and experts, and exhibition of evidence by either side to support their case.

\subsubsection{Bail}
While the trial is going on, to ensure that the accused does not flee to avoid potential punishment, they are often jailed. Otherwise, they may be allowed greater freedom, often short of being able to leave the country or state, in exchange for a large sum of money deposited as guarantee with the court: this is bail.

Bail money is often provided for some fees by bail bondsmen, who inturn have the authority to use bounty hunters (in USA) to bring the accused to justice if they flee trial - causing them to loose money.

\subsubsection{Plea bargain}
To avoid the costs involved in prosecuting, or to prosecute more important targets, the prosecution may convince the defendant to admit his guilt or provide testimony in exchange for a lighter sentence.

\subsubsection{Jury vs Bench}
Trials can either be jury of peers guided by a judge or by a bench of judges. 

Jury decides whether the accused is guilty of the prosecution's charges.

\paragraph{Errors}
Jury trials have been abolished in India and Pakistan as they can be easily led and mislead by the public pressure.

Sometimes the prosecution and police, eager to close the case, coerce witnesses so much that they may lie just to end the unpleasantness - especially if they are guilty of some other crime themselves or if they are young and weak. Even in USA, many are wrongfully sentenced for serious crimes, like murder; and some are eventually exonerated after much struggle.

\paragraph{Sentencing}
Ultimately, the presiding judge decides the correctional action/ punishment. He is often forced by law to impose a minimum or maximum penalty. Harsh sentencing has led to too many people being incarcerated in USA.

\subsection{Incarceration}
Incarceration happens under varying conditions. The woes of incarceration can actually be higher - even inhuman - due to criminal abuse by inmates and guards. For example, prison rape is so common in USA, that prisoners often sought protection in exchange for sex Eg: Donny the punk.

\subsubsection{Parole}
After the minimum sentence has been served by the convict, he may be eligible to be forgiven and released. A parole board convenes to check if the prisoner is eligible - he is expected to be contrite and regretful - this is a problem if a person is wrongly convicted.

In USA, federal crimes are not eligible for parole.

\chapter{Organized crime}
Sometimes criminals form organizations with internal rules and hierarchies. Organization may vary from loose to tight. Big criminal organizations, based on their geographic origin and organization, can be classified into Mafia (Sicily), 'Ndranghetta (Calabria), Yakuza (Japan), Triad (China), Mexican cartels, Columbian cartels and others. Sometimes, as in the case of Yakuza, they are semi-open.

Some avoid particular endeavours - like drugs.

\section{Administration}
In order to guard against prosecution by maintaining plausible deniability, orders are often passed down through a strict hierarchy. Usually, core/ initiated members are required to belong to a particular ethnic group.

In case of American Mafia, the organization is: Godfather/ boss, underboss, consigliore, varios capos, soldiers under them, and finally uninitiated 'associates'.

\subsection{Recruitment}
\subsubsection{(De)Motivations}
Young adult males are often attracted to organized crime/ ganglife due to the easy money, respect of peers, feared status in the local community and a sense of belonging.

\section{Code}
Secrecy and discreetion is often important.

\chapter{Classes}
In nearly every culture, in different times, there have been different economic and social classes of people.

\section{Intelligentsia}
Since antiquity, some classes of people have had more time to devote to arts and sciences than the rest of the population. Furthermore, with the establishment of schools / traditions, they have often ensured development of arts and sciences across generations.

In many cultures, there has been a notable difference between classical (shAstrIya) and folk (janapada) traditions/ worldviews. Eg: dvijas among hindus, nobility and clergy in Europe, samurAi in Japan, mandarins and generals in China, navigators in polynesia, priests and scribes in Egypt, unenslaved people in ancient Greece.

\subsection{Negative aspects}
\subsubsection{Overly liberal attitude}
They do not accord high importance to purity.

Encourage futile self-exploration in habits.

From an internet discussion and from reading Anne Prolux's 'Brokeback mountain'; it appears that it is exploration that leads to the development and establishment of homo-eroticity. This affected an Iyer boy brought up in Kuwait and Canada, who sexually experimented too much.

\subsection{Spread of education}
purA sArvajanikaM dIrgha-shikShaNaM AsIt asAdhyaM - na kEvalaM bhArate, vidEshE api. ataH Eva janAH varNa-parIkShAM samyak na kurvantaH jAti-anusAraM shikShAM adadan. adhunA tu sArvajanikA shikShA sulabhA.

\chapter{Linguistics}
See language survey.

\chapter{Cities}
Humanity is in a process of rapid urbanization. Cities offer greater employment and economic opportunities more efficiently than rural areas.

\part{World-views}
\chapter{Introduction}
Religious world-views are described in another part.

\section{Traditions}
Traditions are vessels by which cultural values are maintained and transmitted across generations.

\chapter{Hedonism}
Hedonists tend seek (almost) immediate gratification - by chemicals (alcohol, drugs), pleasing the senses and stomach, sex, vandalism, even sadism/ hazing etc. - especially with other hedonist friends. This is seen in famed party schools of USA, this behavior institutionalized by several undergraduate fraternities.

\chapter{Materialism/ consumerism}
\tbc

\chapter{jApAnese culture}
\section{Society and values}
They were mainly a society of rice farmers, warriors/ samurai, monks. In old times, there was high social mobility. But it stopped for 400 years due to laws meant for pacification and obedience. They included outcastes called burakumin, who were relegated to 'unclean' and undesirable occupations.

They are an aging society.

\subsection{Conformity, collectivism}
They have been a culture with a high emphasis on conformity: it is important not to stand out - either in good or bad ways, unless the social role requires it.

People often think about what others think about them. Social honor is a big component of individual honor. In old days, some people were driven to commit suicide by hAra-kiri stoically as a matter of punishment or grief over shame.

\subsubsection{Individual sacrifice}
They have extraordinary tolerance of suffering without complaint (gamban). Eg: solidarity displayed after earthquakes and tsunamis without rioting.

During the war, kamikAze suicide attacks were partly motivated by this.

\subsubsection{Crime}
Common crime is low: unlocked bikes are rarely stolen, and 'lost and found' property (even money) is returned to the police. Yet, there is organized crime, in the form of Yakuza, who have existed since recent feudal times.

\subsection{dharma and saMskAras}
They mostly follow rituals from the shinTo dharma in life, and from the bauddha dharma in death. The theoretical framework of the bauddha dharma is deep. The bauddha schools include tAntrika (tendai and shingon), mAntra-vAda (nichiren) and dhyAna (soTo and rinzai/ lin-chi zen). All these schools have many monasteries and monks; they tend to be highly ritualistic (even in the case of dhyAna schools).

The bauddha dharma in japan emphasizes consciousness of tranientness of things.

\subsection{Self-mastery}
Due to influence of the bauddha dharma and deep arts like bushiDo, there is a high appreciation of self-mastery and discipline - which is attained by the few.

\subsection{Stickiness of traditional values}
They Americanized their tastes in clothes, music and even social attitudes- especially the urban population, that too, especially so in the years after their surrender in WW2. Traditionally, there was a heavy emphasis on quasi-vegetarianism; but under american influence, they started eating cow and horse meat.

Yet, there are strong revivalist threads Eg: Samurai movies.

They taken great pride and care in preserving their ancient arts - including martial arts (bushiDo, kenDo), music.

\section{Politics}
The empire of old became a highly feudal society under daimyos ruled by the shogun. After that, they became a constitutional monarchy which was often expansionist, but later renounced war for anything other than self-defense after the obliteration of Hiroshima and Nagasaki. Their neighbors, China and Korea, bear some grudge due to this.

The political landscape in the multiparty democracy is fractured.

\section{Economy}
The country rapidly industrialized after the Meiji restoration. Its current economic condition is perhaps considered in the economics survey.

\section{Xenopohbia}
\tbc

\section{Science and technology}
\tbc

\chapter{Buffalo plains native Americans}
\section{Economy}
They were buffalo hunters. They were essentially a stone-age tribe.

\subsection{Captives}
They also traded in captives. THey acquired horses from the Spanish territories. They raided to capture horses and captives from others.

They routinely adapted very young captives into their tribes to adjust for their high infant mortality and low fertility.

\section{War}
Their war strategy was mostly guerrilla warfare. Their battles were essentially well planned raids - initially for stealing horses, cattle and captives. In raids, all adult males were automatically killed. Young children were sometimes adapted. Women were enslaved.

Captives were brutally tortured in many creative ways - including cutting off toes etc.., burning, gang-rape. This was partly due to animal nature, and partly to weaken the resolve of the enemy people (mostly the latter in the case of white people who were claiming their territory). When the latter reason became prime, torture and mutilation became especially gruesome.

\section{Comanche}
Two main aspects of their life was the buffalo economy and battle.

\subsection{Military}
All Comanche males were warriors: it was a spartan culture. THey were skillful warriors.

\subsubsection{Successes and subdual}
They then subdued all other tribes, including the Apache. They very successfully repressed Spanish and French expansion into the plains. They were the reason the Mexicans let Anglo-saxon settlers settle in Texas. The Anglo-saxons eventually succeeded in beating the Comanche - initially by coordinated attacks, but ultimately by killing off the buffalo herds. The weakened, starving Comanche surrendered and lived reservation lives

\chapter{Polynesian/ micronesian culture}
This culture is spread throughout the pacific ocean - roughly bordered by Fiji, New Zeland, TahITi, Easter and Hawaii.

\section{Roles in society}
First there were fishermen, farmers. There were warriors/ raiders. The class of chiefs, who claimed descent from Gods, were said to carry 'mana'.

\subsection{Navigators}
Expert navigators who relied on natural cues (stars, clouds, flora/ fauna, swells) for navigating canoes to distant islands were highly regarded throughout polynesia. They joined a guild, went through years of apprenticeship, went through an initiation ceremony and were fairly secretive about their knowledge.

They commanded high respect - comparable to that offered to a chief.



\part{Non-scientific World-views}
\chapter{Introduction}
\section{Beliefs and evidence}
Religious world-views affect human thought and behavior. These world views often contain beliefs which are often based on hard-to-verify evidence which often contradicts science.

They often admit to logic and inference, but they often start with the assumption that things described in sacred lore of the religion is true.

\subsection{Reactions due to science}
Scientific theories, which are often supported by a huge mass of evidence, admit to revision when falsified and are examined for weaknesses systematically by thousands of brilliant humans, sometimes contradict, or obviate the myths constituting part of religious belief.

When faced with such a contradiction, what do the religious do?

\subsubsection{Rejection of contrary scientific beliefs}
One may assert that the scientific theories which contradict their beliefs are false. This is a very common reaction of people across religions - especially with those who are not educated in science, or who have not given much thought to the contradictions arising thence.

\subsubsection{Rejection of entire religion}
Another common reaction is to discard the entirety of religion - the unscientific beliefs along with noble traditions and values. This is especially true among well educated, moderately intelligent people who want to solve the contradiction they observe, but do not want to take the intellectually taxing route of critically examining and modifying/ interpreting their religion to be compatible with science.

\subsubsection{Rejection of unscientific beliefs only}
This reaction, though relatively rare, has been observed among scientists who are also Hindu. Rather than throw away 'the baby with the bath water', one tries to rejects religious beliefs contradicted by heavy evidence or obviated by simpler hypotheses; but consciously retains valuable traditions and values of their tradition.

\subsection{Negative consequences}
Personal consequences: Wrong beliefs can be detrimental to one's physical health due to refusal of scientific disease management.

Social consequences: Wrong beliefs being part of one's world-view retards human progress, can result in inter-religious violence etc.. For example: it was considered bad to be outdoors during the solar eclipse in India, and people were enjoined to fast during eclipses and bathe after eclipses. Also, islAmic and Christian zealots destroyed many native cultures.

\section{Beliefs: Attractive attributes}
\subsection{Prayer and hope}
The belief that their prayers and requests will be heard and may be acted upon if the deity deems it appropriate provides hope to people in difficult situations.

Further, prayer often involves the expression of compassion towards someone; and prayer may lead to the development of a more compassionate personality.

\subsection{Justice, continuity after death}
Belief in continuity of life after death comforts those whose beloved have died.

The promise of a reward or punishment after death comforts those who have had tough lives witnessing the injustice of the current life, in which the cruel often prosper.

\subsection{Explaining away mysteries}
Many things have puzzled humans - the mechanism of creation, what drives patterns in nature, the origins of our urges, existence of a supreme God. Religion often cooks up answers to these puzzles with great social authority and relieve people of these questions.

\subsubsection{Support for social rules, values}
Religion provides support and meaning for seemingly arbitrary or unfair social hierarchy and laws. Eg: Why should the Japanese emperor be treated as a deity by his subjects? Why should a hindU not touch paper or books (considered sarasvatI) with his feet?

\subsubsection{Life's purpose, meaning}
Religion, providing a reward at the end of a life as a result of actions, provides meaning and structure to life; and the religious may then feel some safety from feeling the pointlessness of life.

\subsection{Acceptance of vulnerability}
Religious beliefs - especially polytheistic and animist ones - allow practitioners to externalize strength, creativity and inspiration - perhaps into deities. Thus, they assume the role of media rather than the originators of good work.

\subsubsection{Increased effectiveness}
Having externalized desirable qualities, they are able to negotiate with these qualities and tolerate setbacks.

Being able to tolerate repeated setbacks by rationalizing them as being acts of deities, and being dedicated to serving their deities, the religious are able to achieve remarkable dedication in their work.

\section{Dealing with people with contrary beliefs}
In various world-views (religious or otherwise), beyond protecting one's own freedom, reactions to non-believers vary.

Degree 1: Allow them personal liberty, cooperate where there is agreement, contest where there is not.

Degree 2: Shun them, keep them away.

Degree 3: Kill, enslave, imprison or worse if they don't convert.

\chapter{bhArata-darshaNas}
\subsection{purANa}
This is a collection of stories (purANa), often involving magic, rishis (sages), dEvas and their rival asuras, the apsaras, holy animals, snakes, birds.

They include myths about creation, the structure and lifecycle of the universe. (brahmAnDa)

sthala puranas: The regular temple origin myths: Atma lingas, idols hidden in termite mounds with cows releasing milk over them, idols emerging from village lakes, idols as rewards to rigorous penance, site of battle with demon, site of penance.

kula puranas relate stories about origins of various jAtis (clans/ tribes).

Other parts of the Smriti include itihAsa (historic epics): Ramayana and Mahabharatha.

\subsubsection{Accuracy}
smRRiti has less authority than the vEdas - often they contradict each other, and they contradict modern scientific theories. However, they may contain some elements of truth - the germanic tribes claimed to have descended from a certain Manos (who may be identified with manu.)

\subsubsection{Importance}
In the purANas, there is probably a kernel of historic truth. Most of it, however, is mainly useful as moral lessons. The characters and their stories are important in the development of the hindu mind/ values.

\subsection{nAstika works}
bauddhas, jainas and sikhs have their own literature and commentaries. bauddha and jaina works include works analogous to the hindu smRRiti, sUtras and Agamas.

\section{Epistemology}
\tbc

\section{Premodern Metaphysics}
The notions of Atman, incarnation, reincarnation, Karma and Karma-phala ( which include notions of heavens and hells, varied states of being).

\subitem The nyaya school's contributions: epistemology, logic, purpose; Flaws. KanAda's Vaisheshika school; atomism. तयोः दीर्घं चित्तप्रज्ञाभ्यां अध्यायं।

\subitem The dualism of SAnkhya: Purusha (mystical consciousness) vs Prakrithi (matter).

Details about the human condition and purpose are described elsewhere.

\section{Social conventions of jAti and varNa}
\subsection{varNa and jAti}
\subsubsection{varNa}
The dharmashAstras by manu, apastamba etc.. classified people into different varNas (from vRRiNoti - covers) based on role/ occupation in society: brAhamaNas, kShatriyas, vaishyas, shUdras dedicated to learning and priesthood, war and ruling, business and farming, and general service. Besides these, relegated to the fringe occupations were those outside the varNa system, especially constituted by Adi-vAsIs/ forest dwellers.

\subsubsection{jAti}
Within each of these varNas there were subdivisions (mostly not elaborated on in religious scripture). These were mostly-endogamous, often associated with various sub-occupations. So, a varNa could be seen as a collection of jAtis.


\subsection{Division, specialization}
\subsubsection{Distribution of power}
THe varNa system ensured that not all kinds of power were concentrated within a single elite. Though there was some overlap, the vaishyas focused on wealth, the kShatriyas with ruling and war, and brAhmaNas with religion and learning. The shUdras rose to replace dvijas who fell, becoming honorary kShatrIyas, vaishyas and even brAhmaNas (due to manu's laws).

This diversification of power brought with it some robustness to error and aggression from inimical world-views like Islam: hindus were essentially a 5-headed snake.

\subsubsection{Refinement}
Lifelong (even multiple generation) refinement and specialization in narrow occupations occurred due to the varNa system. For example, Manu proscribed direct agriculture by brAhmaNas.

\subsection{Ritual and spiritual purity, hierarchy}
Based partially on economic and political prowess, but mostly on religious education, regular austerities, and adherence to notions of dietary and physical purity required by ritual manuals (including Agamas) and darshaNas, various jAtis formed a social hierarchy with brAhmaNas and kShatriyas being at the top, vaishyas below them, and finally shUdras at the bottom. The first three varNas are called dvijas, vaidika education being authorized for them by the shAstras.

\subsubsection{aspRRishas}
Adherence to notions of occupational, dietary and physical purity has also motivated certain jAtis being labeled unclean outcastes by hindus within varNas, jainas and bauddhas. They were required to live separately away from main settlements.

The bad treatment they recieved was comparable to cagots in France, Baekjong in Korea and burakumin in Japan.

\subsubsection{Enforcement}
The varNa hierarchy was enforced by social disapproval and punishment. Eg: dharmashAstras recommended punishment for a shUdra who insulted a dvija. 

\subsubsection{Opposition by dArshaNikas}
There has been opposition by key personalities (rAmAnuja, rAmAnanda, basavaNNa etc.. within the Astika hindu-darshaNas and buddha, mahAvIra, chArvAka etc.. within the nAstika-darshaNas) and texts (mahAbhArata -especially bhagavat-gIta) to various aspects of the varNa based social conventions, including excessive focus on ritual purity by the brAhmaNas and others, excessive importance placed on family and saMskAras compared to examination of guNas in determining the varNa.

\subsection{Determination of varNa}
Identification of a person's varNa factored in both the individual's qualities, and the family culture / milieu. A person's varNa was determined at a young age - often by birth, but sometimes by examination of the individual by a master.

\subsubsection{Fluidity and Heridity}
An individual's family background played a big role in the intellectual influences a person was surrounded by. See comments on the intelligentsia elsewhere.

Originally, the varNas were very fluid. As described in the texts, a family/ lineage's varNa could be changed up or down based on observing the pattern of behavior or conduct for a a certain number of generations.

Fluidity, however, decreased with time - particularly after the musalmAn invasion; by which time it became ossified - strengthening the jAtis. The correlation between jAti and varNa suitability noted earlier was reinforced and became, to an extant, self-perpetuating.

\subsubsection{Scriptural examples}
jātau vyāsastu kaivartyāḥ śvapākyāstu parāśaraḥ |
bahvo'nyepi vipratvā prāptā ye pūrvā advijāḥ ||
The sage Vyasa was born of a fisherwoman, and Sage Parasara was born of a Chandala, many were those who though not previously brahmins attained brahminical status. (M.B. Aranya-parva 312. 106.)

Following is a quote from P Jain.

"vAlmiki was initially a hunter.  According to Rig Veda (IX.112.3), the poet refers to his diverse parentage: "I am a reciter of hymns, my father is a physician and my mother grinds corn with stones. We desire to obtain wealth in various actions." Sage Aitareya, author of Aitareya Upanisad, was born of a sudra woman. Vasishtha, son of a prostitute, was established as a brahmin and Rig Veda book VII is attributed to him. In Chandogya Upanisad, the honesty of satyakäma establishes his brahminhood, even though his ancestry is unknown as he is the son of a maidservant. vishvAmitra, born in a ksatriya family becomes a sage, and hence a brahmin, based on his asceticism. The priest Vidathin Bhärdväja became a ksatriya as soon as he was adopted by King Bharata and his descendents were the well-known Bharata ksatriyas. Janaka, a ksatriya by birth, attained the rank of a brahmin by virtue of his ripe wisdom and saintly character and is considered a rajarishi (king-sage). vidura, a brahmin visionary, who gave religious and moral instructions to King Dhrtarashtra, was born to a woman servant of the palace. His varna as a brahmin was determined on the basis of his wisdom and knowledge of scriptures. The Kauravas and Pandavas were the descendants of Satyavati, a fisher-woman, and Vyäsa, a brahmin. In spite of this mixed heredity, the Kauravas and Pandavas were known as ksatriyas on the basis of their occupation. Ajamidha and Puramidha were admitted to the status of the brahmin class, and even composed Vedic hymns. Yaska, in his Nirukta, tells us that of two brothers, Santanu and Devapi, one becomes a ksatriya king and the other a brahmin priest. Kavasa, the son of the slave girl Ilusa, becomes a brahmin priest. The Bhagavata Purana tells of the elevation of the ksatriya clan named Dhastru to brahminhood. In the later Vedic times, Chandragupta Maurya, originally from the Muria tribe, goes on to become the famous Mauryan emperor of Magadha. Similarly, his descendant, King Asoka, was the son of a maidservant. The Sanskrit poet and author, Kalidasa is also not known to be a brahmin by birth. His works are considered among the most important Sanskrit works. In the medieval period, saint Thiruvalluvar, author of 'Thirukural' was a weaver. Other saints such as Kabir, Sura Dasa, Ram Dasa and Tukaram came from the sudra class also. Many of the great visionaries in modern India were not brahmins by birth but can be regarded as brahmins by their life-styles and teachings: Mahätmä Gändhi, Swämi Vivekänada, Sri Aurobindo, Maharishi Mahesh Yogi, Swämi Chinmayänanda etc."

Brhadaranyaka Upanisad I. 4. 11-15. These passages describe the successive creation of the four varnas, in contrast to their simultaneous creation in the Purusa Sukta. Just as in the case of Manu where all of humanity is traced to a single parent, here all of humanity is traced to a single homogeneous class, to begin with.

Visnu Purana VIII. 138-140. According to this account when the Eden-like existence ceased: "At this juncture the perfect mind-born sons of Brahma, of different dispositions, who had formerly existed in the Satya age, were reproduced in the Treta as brahmins, ksatriyas, vaisyas, sudras, and destructive men."

\subsection{Modern social role of jAti, varNa}
\subsubsection{Role in identity formation}
The ancient role of jAti in identity formation continues in dvija families who strongly hold on to their saMskAras and niyamas. But, mostly, varNa has been rendered insignificant in social interactions; while jAtis have continued as ethnicities.

The reservation policy of bhArata-gaNarAjya has kept jAti a part of the identity of all people in general.

A huge fraction of people born in dvija jAtis have fallen to western subverstion described in a different section. Many brAhmaNas have thence abandoned their saMskAras. brAhmaNa families in south India have taken to hiding information about their jAti in their names.

\subsubsection{Local variations}
Outside India, the jAti plays no role in the social regard one receives. In urban India, jAti plays a very small role in the social regard one receives. In rural India, jAti plays a big, significant role in the social regard one receives - but this is decreasing by the day, and is much smaller compared to the past.

\subsubsection{Reservation: Job and educational}
The bhArata-gaNarAjya is  theoretically (and to some extant by design as a democracy) committed to the welfare of people of all jAtis; and is mindful of the need for broad-based economic and intellectual prosperity in the face of current lower educational and economic status of the shUdra and AdivAsI jAtis (who constitute roughly 2/3 of the population). So, fairly naturally, it has reserved a huge fraction of educational opportunities in the predominant government university system and of government employment opportunities for people from such jAtis.

In doing so, it has it its way kept jAti as a significant divisive factor in the formation of individuals' identities: one is reminded of one's jAti whenever one tries to access educational or employment opportunities for oneself or for one's family.

There may be better ways of effecting broad economic and intellectual prosperity (eg: rather than being based on jAti, reservation based on economic and educational status of the family), but the status-quo is unlikely to change - disadvantaged people from the forward jAtis are perhaps too few to affect political interests, while people from jAtis enjoying reservation will strive to maintain the status quo and to maintain jAti as part of their political and cultural identity/ consciousness so that they will not need to share the benefits of reservation with disadvantaged people of the forward castes.

While jAti-based reservation may be sub-optimal, it still is somewhat useful and its existance is an unsurprising consequence of the importance of jAti in cultural identity across the country in 1950 when the constitution was created.

\section{Traditions (saMskAras)}
shODasha-saMskAras (16 saMskAras) are the rituals which must be undergone during the life of a person, starting from conception. Topics: Ahnika/ daily rituals are to be observed daily. Occasional rites include: shrAddha. maDi / ritual purity.

\subsection{Festivals}
The festivals, the stories behind the festivals, the associated customs, the Lunar and Solar calendars, Ekadasi.

\section{Human condition and efforts}
puruShArtha (goals of a human life) are said to include dharma (social utility), artha (anything which can be accrued, including education and wealth), kAma. Many branches include mOkSha/ liberation.

Purva Mimamsa: their orthopraxy and verbal Dharma, orthodoxy, anti-asceticism and anti-mysticism. Clear polytheism. No "liberation".

\subsection{Pursuit of excellence}
Various jAtis and varNas, specializing in their different social roles, were expected and encouraged to attain excellence.

\subsubsection{guru-shiShya-parampara}
Students in many disciplines were expected to undergo long apprenticeships starting from childhood in their chosen vocation. During this time, they were expected to be dedicated to the art, and simultaneously earn their keep by being servants of their masters (who could be quite harsh - atleast in the beginning).

\subsubsection{Exaltation of scholarship}
The classical example of Tirumalai KrishNamAcharya; The fine examples of the ancient mathematicians, Srinivasa Ramanujan, CV Raman, Subrahmaniam Chandrashekar.

Wise seers/ RRishis were universally honored.

\section{Art}
\section{Superstitions and their effects}
\subitem astrology, NAdi.
\subitem Yantras, curses, posessions by ghosts.
\subitem Sati, child marriage, lack of high female education.
\subitem The thugees.
\subitem The despicable Yogabhrashtas/ Bhogis.

\subitem Fatalism.

\section{Science and technology}
The contributions of ancient Indian mathematicians (0, negative numbers, decimal number system, trigonometric functions, formal grammar, infinite series for PI, trigonometric functions, lIlavathi.).

The contributions of modern mathematicians and scientists of the community.

Fascinating lectures about bhArata's scientific contributions - beginning with AryabhaTIya (which includes wonders like calculation of the diameter and declination of the earth, even the height of its atmosphere), proceeding to rasOpaniShat and other metallurgy (which describes 23 tin alloys of various colors recently patented by Americans, introduction of zinc / malbAr lead to the west, sword-iron making), civil engineering (remarkably strong brick, cement, ceramic composition), surgery (surgical instruments and procedures copied to the west, meticulous training), silk textiles! 

\htext{Link}{http://www.youtube.com/watch?v=dqmlhG397g0&feature=related}.

\subsection{Poetic presentation}
1 The preoccupation with poetic/ word beauty is interesting - even in communicating numbers. The number of words/ concepts with each natural number in the bhUta-sa~NkhyA (described in the lectures) seems to make communicating (or even contemplating) numbers so poetic! The kaTapayAdi number system also seems to be designed for poetic composition.

\subsection{Role of proofs}
2 I was under the impression that the bhAratas were lazy in giving proofs for their theorems, but it appears that proofs usually appear in the bhAShyas (commentaries).

\section{brAhmaNa-saMskArasya vaishiShTyaM}
\subsection{Positive aspects}
\tbc To be edited

\begin{itemize}
\item The values imbibed by the members of the community in general. These include the following:
\subitem Exaltation of scholarship.
\subitem Introspection, strong consciousness and a deep understanding of oneself. (Prajnaa)
\subitem Importance/ permanence of family.
\subitem Hygene, chastity and disgust at intoxication.
\subitem Desire to question and seek the truth. Close familiarity and the resultant immunity from irrational ideas. Arguably, a desire to advance man's knowledge of and power over nature and mathematics.
\subitem The habit of introspection.
\subitem Deprecation of money relative to learning.

\item Useful institutions such as arranged marriage.

\item The positive stereotypes associated with the community. An understanding of the differences in values and support for one's objectives among various communities, and the ability to choose and exploit one's membership in the right communities.

\item The symbols of distinctiveness and identification which serve to bring the sterotypes and values into action.
\end{itemize}


\subsection{The Nerd culture}
Being an orthodox brAhmaNa means being a nerd and belonging to a tradition of nerds. Think about it: A nerd is someone who derives pleasure mainly by propitiating his intellect; plus he is sincere. Pradoxical as it sounds, brAhmaNas were a society of nerds, a group of people who delighted in the world of ideas. As evidence, of course are all esoteric rituals from sandhyasvandanam down (including parisinchanam), recitation of the Vedas, performance of homas governed by strict rules, the great corpus of commentaries, philosophical works, poems, the inquiry and introspection found therein. Samskrutam, the highly inflected language in which it is hard not to make a mistake is verily a language of nerds.

Relevant fact: On visiting the tamil matrimony site in 2008, it was apparent that both among males and females, persons doing PhD were overwhelmingly likely to be Brahmins (Iyers/ Iyengars).

\section{itihAsa}
The history of India, relative to the world. The history of south India.

\subsection{Aryan invasion}
Linguistic and textual evidence suggests that people from the northwest carried the prototypical vaidika culture with them, perhaps amalgamating influences from the indus valley civilization in the process.

\subsection{Failures in development}
The failures of the Hindu civilization, overemphasis on mysticism and religion, the caste discrimination in education, not travelling to foreign centers of learning, not learning from foreign civilization, not valuing skepticism. The criticisms of Charvaka and Al Beruni.

\subsection{Interaction with islAm}
\subsubsection{Treatment of hindu people}
hindu prisoners were often given a choice between conversion (most often followed by slavery) and death. Eg: The poet muhammad iqbAl (writer of 'sAre jahA.n se achCHA') was the grandson of a kAshmIri paNDita who was alleged to be siphoning off the kAshmIri governor's money and forced to convert at sword point.

The saMskRRita translation of SL Bhairappa's AvaraNaM tells me the gory details, citing excellent references, about trade in eununchs during the various Sultanates that ruled India! These precious slaves were created
by cutting off the balls of prisoners of war and children of peasants unable to pay taxes.

Upon defeat, women dreading use as slaves by the turuSHkas sometimes preferred death and underwent jIvahara (jouhAr).

It was at Jai Singh’s insistence that the hated jaziya tax, imposed on the Hindu population by Aurangzeb (1679), was finally abolished by the Emperor Muhammad Shah in 1720. In 1728 Jai Singh prevailed on him to also withdraw the pilgrimage tax on Hindus at Gaya.

\subsubsection{Destruction of temple, universities}
Many turuShka kings took their titles and roles as ghAzis seriously, and consciously, repeatedly worked to destroy hindu culture, its temples and universities. Eg: the library and university at Nalanda, the ruined temples of vijayanagara, haLEbIDu. Aurangzeb was particularly zealous: Only two years after Jai Singh's demise Aurangzeb passed an order (1669) calling for the demolition of Hindu temples in the Mughal provinces.

``The wicked mlecchas pollute the religion of the Hindus every day. They break the images of the gods into pieces and throw away the articles of worship. They throw into fire Bhagavata Purana and other holy scriptures. They forcibly take away a conch shell and bell of the brahmanas, and lick the sandal paints of their bodies. They urinate like dogs on the sacred Tulasi plant and deliberately pass faeces in the Hindu temples. They would throw water from their mouths on the Hindus engaged in worship, and harass the Hindu saints as if they were so many lunatics let large.`` —I. Nagara Advaita-prakasa.

\subsubsection{Reconstructions}
Whenever hindu kings regained dominion, they restored or built fresh new temples the turuShkAs destroyed. Eg: somanAtha temple was rebuilt repeatedly, the vishvanAtha temple in kAshi (by the rAjaputra prince. then by ahilyAbAi, with gold added by raNajIt singh).

The vaishyas kept financing recontructions and repairs. Eg: mArvADis in the case of kAShi.

\subsubsection{Military/ political resistance}
\paragraph{kShatriyas}
Often overlooked is the fact that some battle or the other was waged for almost every month against the admittedly powerful armies of "Peace" (islAm) for 1000 years, led by heroes such as bhoja, prithvirAja chavana, hammIra siMha, rANa sagrAma siMha, hEmachandra vikramAditya, maharANa pratAp, kRRiShNadevarAya, shivAji, some pEshvas, banda bahadur, raNajIt singh etc.; which is why India did not crumble like many other civilizations - from Persia to Indonesia!

\paragraph{vaishyas}
vaishyAs sometimes voluntarily helped finance such resistance: eg: ba(h)man shAh financing and participating in mahArANA pratAp's fight against akbar.

\paragraph{brAhmaNas}
brAhmaNas repeatedly provided inspiration and plotted for the restoration and protection of hindu kingdoms from yavanas. Eg: chANakya behind the mauryas, vidyAraNya behind vijayanagara, ??? behind shivAji.

Besides, they often served as ministers (diwAn / peshkar) to turuShka kings and governors. In some cases, they were useless - often making profit; but in many cases, they plotted to ensure greater religious freedom - and even the overthrow of their turuShka masters. Eg: hEmachandra vikramAditya in dilli, rAmadAsa, madaNNa and akaNNa in gOlkoNDa.

\paragraph{shUdras}
Besides participating in business and wars at low levels, they replenished the ranks of kShatriyas who fell in battle. They were repeatedly raised to the ranks of honorary kShatriyas - often with invention of fictitious lineages by brAhmaNas. 

\subsubsection{Philosophical resistance}
A few converted musalmAns have reverted to the hindu dharma. But, these were precious few - they are listed in \htext{Wikipedia}{http://en.wikipedia.org/wiki/List_of_converts_to_Hinduism#From_Abrahamic_religions}.

jalAluddin akbar, in his later years, became highly tolerant of hindus - his rAjaputra wives were allowed to keep hindu idols, his din-e-ilAhi religion allowed that there are several paths to a good life.

dAra-shukOh, elder brother of the ghAzi awrangzEb, was in contact with hindu sAdhus, taught hindu thought - he was instrumental in recording and translating upaniShads to fArsi.

AshIsh khAn, son of the master musician ali akbar khAn, stated in 2006 that his family were never officially converted into Islam, but hid under the guise of being muslims.

\subsubsection{Influence on islAm}
The sufi branch of islAm was heavily influenced by bhAratIya ideas - eg: in ideals like fanA (ego-death), meditative exercises, trances. Yet, sufis (including those of the chisti order) assisted ghAzis in their war on hindus.

\section{Western subversion}
\subsection{Education and media}
The british colonial government's education which looked upon hindu saMskAras from the eyes of outsiders (initially aided strongly by missionaries, and later by Marxists) has led to a strong subversion of the hindu dharma - going beyond mere correction of flaws to disparaging even the noble traditions in the intellects of Hindus. The media, filled with people educated in this manner, has reinforced this subversion.

\subsection{Development argument}
One argument made in favor of western education and lifestyle is that it has brought development and prosperity to a poor people who were antagonistic to each other. However, countries such as Thailand, Taiwan, Japan and South Korea have copied beneficial ideas such as science and the industrial revolution, and have reformed their societies and economy mostly from within - with little imposition from outside.

\subsection{Exploiting divisions}
The western subversion is to a great extant aided by disenchanted group/s indigenous to India whose interests they supported greatly, even though strong hindu movements have supported the same interests. These groups include dalitas (treated as aspRRishas under old conventions), musalmAns, Christians. The jainas, sikhs, bauddhas and pArsis, apart from brief exceptions, have not joined this effort.

People in general have lately been poorly educated in the noble history and accomplishments of their fertile culture (hitherto dominated by dvija interests), but have received education in science, exposed to western values and have been rendered guilt-ridden about the jAti-hierarchy which denied humane and respectful treatment to shUdras and AdivAsIs. This has resulted in dvijas abandoning their saMskAras.

\subsection{Current strength}
Although most of India's intellectual elite is Hindu, the great majority of them are Hindu haters who are ashamed to identify themselves as Hindu.

\subsection{Resistance: old institutions}
maThas (monasteries), temples, utsavas have offered continued access to some part of the the dharma's ethos and ideas. They have thus been a potent force in the dvijas with strong saMskAras and rural people resisting the subversion.

\subsection{Resistance: spiritual movements}
The great patriot vivEkAnanda worked to revive hindu svAbhimAna and conscience through the rAmakRRiShNa mission. His presentation of the hindu dharma and saMskAras found great appreciation abroad; and there fore became 'cool' in bhArata. Arya-samAja, yOgAnanda's 'self realization fellowship', bhakti movements such as that of gauDIya vaiShNavas through ISKCON, mahEsha yOgi's transcendental meditation movement, ravisha\~Nkara's 'Art of living' continued and strengthened this resistance.

Non-hindu movements springing from bhAratIya darshaNas such as various bauddha traditions of various bauddha countries, sikh movments reinforced this to some extant.

yOga gaining popularity in the west - through the efforts of people from shivAnanda, bikrama and kRRiShNamAchArya's gurukulas (BKS iye\~NgAr, paTThAbhi jois), has again subverted the westerd subversion of the dharma.

\subsection{Resistance: cultural movements}
Inspired in part by vIra sAvarkar, rAShTRIya svayaMsEvaka sangha (RSS) pioneered hindu revivalism - they have tried to cultivate hindu youth strong in both body and intellect by starting a grassroots/ bottom-up movement across bhArata in the form of shAkhAs in different places.

saMskRRita-bhAratI, another grassroots movement, is working towards the revival of sanskrit, the language of liturgy in bhAratIya darshaNas and of classical culture. saMskRRita has great potential to function in bhAratIya revival in a way analogous to the role of the 'master switch' in a house's electrical circuitry.

More traditionalist movements, such as the vishva hindu parishat (VHP), have worked to organize hindu leaders.

All these movements have spread to the bhAratIya diaspora abroad.

\subsection{Resistence: media, internet and videos}
Videos of the epics rAmAyaNa and mahAbhArata (and later some purANas) broadcast on national television reacquainted bhAratIya children with the dharma, and reminded their elders of it. Books by rAjagOpAlachAri, amar chitra kathA comics did the same.

Videos about historical figures such as the marATha empire, chANakya, gAndhi reminded people of past successful hindu resistance to foreign rule.

The internet has allowed people to connect with people who care about the dharma even when they are far away from it. \tbc

\chapter{abrAhamic religions}
\section{Metaphysics}
\subsection{Jealous omnipotent God}
Judaism and cognates in the middle-east later gave rise to chirstianity and islAm. Fundamental is the belief in the accuracy of the old-testament in the bible and a omnipotent omniscient god called yahweh/ allAH who is jealously eager to be the sole recipient of all worship.

\subsubsection{Creation}
There is a tale about creation in 7 days and nights; one person speculated that he did this to relieve himself of loneliness.

\subsection{His deals}
He is said to have chosen and blessed the tribe of israel, and made agreements with various individuals such as Abraham and Moses to ensure that he is glorified and worshiped in exchange for fertility, prosperity and comfort after death.

\subsubsection{Intervention in daily-life}
God is said to answer prayers/ requests, and intervene in everyday life.

\subsubsection{Heaven and hell}
Good deeds are said to be rewarded with admittance to heaven, where a luxurious life awaits, while bad deeds are rewarded with admittance to hell, where there is much suffering. Besides this, Christianity may admit to the existence of a limbo.

The fear of being sent to hell causes guilt and suffering in many dying people - especially because that transition is irreversible.

\subsection{Devil}
Just as God is said to intervene in daily life, a certain entity called Satan or shaitAn - a fallen angel (God's servant with superpowers) - is said to do the same, in opposition to God. Why his existence is tolerated by a supposedly all-powerful God is obscure: perhaps on account of granting free will? \chk

People tend to blame those who do not follow the laws specified in their books as being inspired by this entity - by direct communication or by emotional manipulation (something God would not do because of his gift of free will).

\section{Basic laws}
Amongst the fundamental 10 commandments (which include some sensible social laws against stealing, murder of fellow-believers, adultery) revealed by this God is an injunction against idol-worship and worship of other Gods, which makes them intolerant towards eastern religions.

\section{islAm}
This derivative developed in the Arabian desert.

\subsection{prophet and saints}
It has the added affirmation that Muhammad is the last prophet of God, and holds his life up as an ideal for believers/ muslims in general.

\subsection{Social conventions}
\subsubsection{Treatment of unbelievers}
The Banu Qurayza massacre being an excellent example: decimation of all males who did not accept that he is the prophet (that is most of them), enslavement of the women and children (of which he took a 1/5th share, being the 'government'). Subsequent muslim ghAzis, especially before the 20th century, just followed his example.

\subsubsection{Role of women}
Women and slaves were treated as property and their freedom curtailed, even though it was acknowledged that they have a soul. \tbc


\subsection{Self-control}
The laws common to abrAhamic religions induce some self-control - yet even devout musalmAns lie 'for higher purposes'. Many forms of self-discipline are encouraged - abstinence from alcohol, pork and animals not ritually slaughtered.

\subsubsection{Admittance of lust}
Muhammad's lust is displayed in his courting of Rayhana (who spurned him and was enslaved too). Thus, there is greater allowance for lust and incivility than in bhAratIya darshaNas.

\subsection{Muhammad's example: criticism}
Is this the epitome, are these the qualities of either a role model or a compassionate, wise person who is in cohorts with God? Is he someone worthy of being held dear? 

\subsubsection{Appreciation from others}
There are social (leave alone political) dimensions to any dharma - it cannot just be a means to attain mOkSha/ liberation of some sort. Might saints such as rAmakRRiShNa, in taking just the devotional aspect of Islam (thus effectively hinduizing it, as has happened to some extant to the sUfi branch), have only been talking about that part of the religion - therefore just finding what he wanted to find in it?

\subsubsection{saMbhrAnta-yOgI}
Muhammad did not have a qualified guru. So, it is very possible that he could have misinterpreted the extraordinary visions and experiences he might have. He might have been blind to great flaws in himself and his thoughts. For example, he was eager to be accepted as the messaiah by the jews, of which his anti-idolatory campaign at mecca need not have been the cause, but the effect.

\subsection{History}
\tbc

\section{Christianity}
\subsection{Savior and saints}
Christianity imposes additional belief and focus about Jesus Christ as the savior of humanity.

Acceptance of Jesus as the savior is claimed by many Christians as the only route to heaven.


\end{document}
