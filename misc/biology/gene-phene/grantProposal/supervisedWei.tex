Another way of estimating new proximities between genes and phenes is to extend the above graph proximity methods into a supervised framework. For instance, if we adopt a linear regression model, a linear function by combining various path counting features (extracted from gene-phene bipartite graphs based on other species) can be used to approximate the target gene-phene bipartite graph. The model parameters, i.e., weights for different path counting features, can be obtained adaptively through linear regression by fitting the data. More sophisticated models such as generalized linear models can be applied by extending the linear function to the nonlinear case. For example, we can transform the output of the linear model through a logistic function. This will lead to the logistic regression model, which can capture more complex intrinsic properties of the data than the linear regression model. 

Multiple gene-phene interactions in species other than humans can be naturally treated as multiple sources of information characterizing various connections between genes and phenes. It would be interesting to find out whether gene-phene interactions in multiple species can collectively predict potential gene-phene connections in human diseases. With multiple proximity graphs, we have a much richer choice of graph topological features (hybrid path counting features) by allowing cross routes between graphs. Let us take an example for better understanding. Note that multiple gene-gene connections can be formed based on gene-phene networks from multiple species. In the context of \emph{multigraphs}\cite{harary94}, between any two genes there can be an edge from source $A$ and an edge from source $B$. Both $A$ and $B$ are in the form of bipartite graphs. By traversing edges in both $A$ and $B$, we can construct hybrid path counting features across multiple sources in addition to pure path counting features within just one source. To control the model complexity due to the increase of the number of path counting features, we can enforce some sparsity constraints on the feature weights through L1 regularization\cite{lasso}. This will lead to the Lasso or the grouped Lasso problem (following the hierarchical structure formed by multiple sources)\cite{jenatton09,bach08,zhao09}. In the spirit of selectively combining hybrid path counting features from multiple sources, the PI's group has sucessfully applied the idea to the link prediction problem in social network analysis, which yields promising results on real world application data\cite{berkantSupervised}.  
