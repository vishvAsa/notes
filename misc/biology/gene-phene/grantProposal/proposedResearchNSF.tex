\subsection{Approaches based on unbalanced classification}
One can embed gene-disease pairs in a feature space, as described in \cite{basilicoHofmann} for example, and then try to find a separator which is good at separating known gene-disease pairs (ones in the $P_1$ matrix, the positive samples) from gene-phene pairs with no known associations (zeros in the $P_1$ matrix, the unlabelled samples). One can then identify predicted gene-disease pairs based on their position in the feature space relative to this separator. However, a good separator must take into account the fact that we are completely certain that the positive samples are correctly labelled; whereas we know that even though the unlabelled samples are most likely to be negative, they include a few positive samples. One can therefore view the problem of identifying gene-disease links as an unbalanced classification problem.

So, rather than using the usual maximum margin classifier (SVM), one can consider a variant which penalizes misclassification of positive samples in the training set far more than penalizing the classification of unlabelled points as positive. In the same spirit, one can consider another variant of the SVM which does not penalize upto $k$ classifications of unabelled points as positive. Even though this problem is non-convex, early experiments show that alternating minimization based heuristics work well. Such application of one-class classification to the problem of link prediction would also be explored as part of the proposed research.

\subsection{Other approaches}
The ranking problem has recently attracted much attention in recent years. Rank aggregation algorithms \cite{rankAggregation} promise to be useful in combining predictions from multiple predictors in identifying gene-disease links.