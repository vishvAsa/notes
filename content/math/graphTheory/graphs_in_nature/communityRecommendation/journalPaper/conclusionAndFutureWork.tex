\subsection{Summary and Conclusion}
In this paper, we have tackled the affiliation recommendation problem, where the task is to recommend new affiliations to users, given the current state of the friendship and affiliation networks. We show that information from the friendship network can indeed be fruitfully exploited in making affiliation recommendations. This auxiliary source of information was hitherto not used in making community recommendations.

Using a simple way of combining these networks, we suggested two ways of modeling the networks for the purpose of making affiliation recommendations (Section~\ref{Models}). The first of these approached the problem from the graph proximity viewpoint, whereas the second modelled the interactions of users and groups in the two networks using latent factors derived from optimizing towards a joint matrix factorization objective. We studied the algorithms suggested by these models on real world networks (Section~\ref{Experimental Evaluation}). We motivated and proposed a way of evaluating recommenders, by measuring how good the top 50 recommendations are, and demonstrated the importance of choosing the right evaluation strategy. Algorithms suggested by the graph proximity model turn out to be the most effective, based on experiments on large real world data sets. We also introduced scalable versions of these algorithms, and demonstrated their performance. These results show that the application of techniques from social network link prediction in affiliation and item recommendation is a promising one.

\subsection{Future Work}
There is the intriguing possibility of using an affiliation network for link prediction in the friendship network. Discovering techniques and models which do this effectively seems to be a challenging research avenue. Our early experiments at doing this indicate that this is a much harder problem. The reasons for this are not yet clear, and this question seems fertile for further exploration.

Within the ambit of the affiliation recommendation problem itself, one may research the ways of fruitfully using even more sources of information. For example, \cite{GoogleCCF} use information from textual description of communities along with the affiliation network to make affiliation recommendations. It might be useful to consider the social network together with this auxiliary information. Also, predictors based on latent factors model and the graph proximity model may be suited for different types of users, and creating a meta-predictor which combines predictions from both classes of predictors is another attractive research direction. Finally,  from the perspective of making scalable recommendations, and considering the relative effectiveness of the common subspace approach to approximate $\SS$ and $\A$, we may benefit from a clustered version of the truncated \textsf{Katz} measure when it is based on the common subspace approximations of the two networks. 