\documentclass[10pt]{amsart}
\usepackage{amssymb}
\usepackage{graphics}
\usepackage{verbatim}

% Use something like:
% % Use something like:
% % Use something like:
% \input{../../macros}

% groupings of objects.
\newcommand{\set}[1]{\ensuremath{\left\{ #1 \right\}}}
\newcommand{\seq}[1]{\ensuremath{\left(#1\right)}}
\newcommand{\ang}[1]{\ensuremath{\langle#1\rangle}}
\newcommand{\tuple}[1]{\ensuremath{\left(#1\right)}}
\newcommand{\size}[1]{\ensuremath{\left| #1\right|}}

% numerical shortcuts.
\newcommand{\abs}[1]{\ensuremath{\left| #1\right|}}
\newcommand{\floor}[1]{\ensuremath{\left\lfloor #1 \right\rfloor}}
\newcommand{\ceil}[1]{\ensuremath{\left\lceil #1 \right\rceil}}

% linear algebra shortcuts.
\newcommand{\change}{\ensuremath{\Delta}}
\newcommand{\norm}[1]{\ensuremath{\left\| #1\right\|}}
\newcommand{\dprod}[1]{\ensuremath{\langle#1\rangle}}
\newcommand{\linspan}[1]{\ensuremath{\langle#1\rangle}}
\newcommand{\conj}[1]{\ensuremath{\overline{#1}}}
\newcommand{\der}{\ensuremath{\frac{d}{dx}}}
\newcommand{\lap}{\ensuremath{\Delta}}
\newcommand{\kron}{\ensuremath{\otimes}}
\newcommand{\nperp}{\ensuremath{\nvdash}}

\newcommand{\mat}[1]{\left[ \begin{smallmatrix}#1 \end{smallmatrix} \right]}

% derivatives and limits
\newcommand{\partder}[2]{\ensuremath{\frac{\partial #1}{\partial #2}}}
\newcommand{\partdern}[3]{\ensuremath{\frac{\partial^{#3} #1}{\partial #2^{#3}}}}
\newcommand{\gradient}{\ensuremath{\nabla}}
\newcommand{\subdifferential}{\ensuremath{\partial}}

% Arrows
\newcommand{\diverge}{\ensuremath{\nearrow}}
\newcommand{\notto}{\ensuremath{\nrightarrow}}
\newcommand{\up}{\ensuremath{\uparrow}}
\newcommand{\down}{\ensuremath{\downarrow}}
% gets and gives are defined!

% ordering operators
\newcommand{\oleq}{\preceq}
\newcommand{\ogeq}{\succeq}

% programming and logic operators
\newcommand{\dfn}{:=}
\newcommand{\assign}{:=}
\newcommand{\co}{\ co\ }
\newcommand{\en}{\ en\ }


% logic operators
\newcommand{\xor}{\ensuremath{\oplus}}
\newcommand{\Land}{\ensuremath{\bigwedge}}
\newcommand{\Lor}{\ensuremath{\bigvee}}
\newcommand{\finish}{\ensuremath{\Box}}
\newcommand{\contra}{\ensuremath{\Rightarrow \Leftarrow}}
\newcommand{\iseq}{\ensuremath{\stackrel{_?}{=}}}


% Set theory
\newcommand{\symdiff}{\ensuremath{\Delta}}
\newcommand{\union}{\ensuremath{\cup}}
\newcommand{\inters}{\ensuremath{\cap}}
\newcommand{\Union}{\ensuremath{\bigcup}}
\newcommand{\Inters}{\ensuremath{\bigcap}}
\newcommand{\nullSet}{\ensuremath{\phi}}


% graph theory
\newcommand{\nbd}{\Gamma}

% Script alphabets
% For reals, use \Re

% greek letters
\newcommand{\eps}{\ensuremath{\epsilon}}
\newcommand{\del}{\ensuremath{\delta}}
\newcommand{\ga}{\ensuremath{\alpha}}
\newcommand{\gb}{\ensuremath{\beta}}
\newcommand{\gd}{\ensuremath{\del}}
\newcommand{\gp}{\ensuremath{\pi}}
\newcommand{\gf}{\ensuremath{\phi}}
\newcommand{\gh}{\ensuremath{\eta}}
\newcommand{\gF}{\ensuremath{\Phi}}
\newcommand{\gl}{\ensuremath{\lambda}}
\newcommand{\gm}{\ensuremath{\mu}}
\newcommand{\gn}{\ensuremath{\nu}}
\newcommand{\gr}{\ensuremath{\rho}}
\newcommand{\gs}{\ensuremath{\sigma}}
\newcommand{\gt}{\ensuremath{\theta}}
\newcommand{\gx}{\ensuremath{\xi}}

\newcommand{\sw}{\ensuremath{\sigma}}
\newcommand{\SW}{\ensuremath{\Sigma}}
\newcommand{\ew}{\ensuremath{\lambda}}
\newcommand{\EW}{\ensuremath{\Lambda}}

\newcommand{\Del}{\ensuremath{\Delta}}
\newcommand{\gD}{\ensuremath{\Delta}}
\newcommand{\gG}{\ensuremath{\Gamma}}
\newcommand{\gO}{\ensuremath{\Omega}}
\newcommand{\gS}{\ensuremath{\Sigma}}

% Bold english letters.
\newcommand{\bA}{\ensuremath{\mathbf{A}}}
\newcommand{\bB}{\ensuremath{\mathbf{B}}}
\newcommand{\bC}{\ensuremath{\mathbf{C}}}
\newcommand{\bD}{\ensuremath{\mathbf{D}}}
\newcommand{\bE}{\ensuremath{\mathbf{E}}}
\newcommand{\bF}{\ensuremath{\mathbf{F}}}
\newcommand{\bG}{\ensuremath{\mathbf{G}}}
\newcommand{\bH}{\ensuremath{\mathbf{H}}}
\newcommand{\bI}{\ensuremath{\mathbf{I}}}
\newcommand{\bJ}{\ensuremath{\mathbf{J}}}
\newcommand{\bK}{\ensuremath{\mathbf{K}}}
\newcommand{\bL}{\ensuremath{\mathbf{L}}}
\newcommand{\bM}{\ensuremath{\mathbf{M}}}
\newcommand{\bN}{\ensuremath{\mathbf{N}}}
\newcommand{\bO}{\ensuremath{\mathbf{O}}}
\newcommand{\bP}{\ensuremath{\mathbf{P}}}
\newcommand{\bQ}{\ensuremath{\mathbf{Q}}}
\newcommand{\bR}{\ensuremath{\mathbf{R}}} % PROSPER defines \R
\newcommand{\bS}{\ensuremath{\mathbf{S}}}
\newcommand{\bT}{\ensuremath{\mathbf{T}}}
\newcommand{\bU}{\ensuremath{\mathbf{U}}}
\newcommand{\bV}{\ensuremath{\mathbf{V}}}
\newcommand{\bW}{\ensuremath{\mathbf{W}}}
\newcommand{\bX}{\ensuremath{\mathbf{X}}}
\newcommand{\bY}{\ensuremath{\mathbf{Y}}}
\newcommand{\bZ}{\ensuremath{\mathbf{Z}}}
\newcommand{\bba}{\ensuremath{\mathbf{a}}}
\newcommand{\bbb}{\ensuremath{\mathbf{b}}}
\newcommand{\bbc}{\ensuremath{\mathbf{c}}}
\newcommand{\bbd}{\ensuremath{\mathbf{d}}}
\newcommand{\bbe}{\ensuremath{\mathbf{e}}}
\newcommand{\bbf}{\ensuremath{\mathbf{f}}}
\newcommand{\bbg}{\ensuremath{\mathbf{g}}}
\newcommand{\bbh}{\ensuremath{\mathbf{h}}}
\newcommand{\bbk}{\ensuremath{\mathbf{k}}}
\newcommand{\bbl}{\ensuremath{\mathbf{l}}}
\newcommand{\bbm}{\ensuremath{\mathbf{m}}}
\newcommand{\bbn}{\ensuremath{\mathbf{n}}}
\newcommand{\bbp}{\ensuremath{\mathbf{p}}}
\newcommand{\bbq}{\ensuremath{\mathbf{q}}}
\newcommand{\bbr}{\ensuremath{\mathbf{r}}}
\newcommand{\bbs}{\ensuremath{\mathbf{s}}}  % TIPA defines \s and LaTeX \ss!
\newcommand{\bbt}{\ensuremath{\mathbf{t}}}
\newcommand{\bbu}{\ensuremath{\mathbf{u}}}
\newcommand{\bbv}{\ensuremath{\mathbf{v}}}
\newcommand{\bbw}{\ensuremath{\mathbf{w}}}
\newcommand{\bbx}{\ensuremath{\mathbf{x}}}
\newcommand{\bby}{\ensuremath{\mathbf{y}}}
\newcommand{\bbz}{\ensuremath{\mathbf{z}}}
\newcommand{\0}{\ensuremath{\mathbf{0}}}
\newcommand{\1}{\ensuremath{\mathbf{1}}}


% Formatting shortcuts
\newcommand{\red}[1]{\textcolor{red}{#1}}
\newcommand{\green}[1]{\textcolor{green}{#1}}
\newcommand{\magenta}[1]{\textcolor{magenta}{#1}}
\newcommand{\orange}[1]{\textcolor{orange}{#1}}
\newcommand{\gray}[1]{\textcolor{gray}{#1}}
\newcommand{\blue}[1]{\textcolor{blue}{#1}}
\newcommand{\htext}[2]{\texorpdfstring{#1}{#2}}

% Statistics
\newcommand{\distr}{\ensuremath{\sim}}
\newcommand{\stddev}{\ensuremath{\sigma}}
\newcommand{\covmatrix}{\ensuremath{\Sigma}}
\newcommand{\mean}{\ensuremath{\mu}}
\newcommand{\param}{\ensuremath{\gt}}
\newcommand{\ftr}{\ensuremath{\phi}}

% General utility
\newcommand{\todo}[1]{\textbf{[TODO]}] \footnote{TODO: #1}}
\newcommand{\exclaim}[1]{{\textbf{\textit{#1}}}}
\newcommand{\tbc}{[\textbf{Incomplete}]}
\newcommand{\chk}{[\textbf{Check}]}
\newcommand{\oprob}{[\textbf{OP}]:}
\newcommand{\core}[1]{\textbf{Core Idea:}}
\newcommand{\why}{[\textbf{Find proof}]}
\newcommand{\opt}[1]{\textit{#1}}


\renewcommand{\~}{\htext{$\sim$}{~}}

\DeclareMathOperator*{\argmin}{arg\,min}
\DeclareMathOperator*{\argmax}{arg\,max}

% Items pertaining to the headed list.
\newcommand{\headeditem}[1]{\item\textbf{#1}}
\newcommand{\headedsubitem}[1]{\subitem\textbf{#1}}



% groupings of objects.
\newcommand{\set}[1]{\ensuremath{\left\{ #1 \right\}}}
\newcommand{\seq}[1]{\ensuremath{\left(#1\right)}}
\newcommand{\ang}[1]{\ensuremath{\langle#1\rangle}}
\newcommand{\tuple}[1]{\ensuremath{\left(#1\right)}}
\newcommand{\size}[1]{\ensuremath{\left| #1\right|}}

% numerical shortcuts.
\newcommand{\abs}[1]{\ensuremath{\left| #1\right|}}
\newcommand{\floor}[1]{\ensuremath{\left\lfloor #1 \right\rfloor}}
\newcommand{\ceil}[1]{\ensuremath{\left\lceil #1 \right\rceil}}

% linear algebra shortcuts.
\newcommand{\change}{\ensuremath{\Delta}}
\newcommand{\norm}[1]{\ensuremath{\left\| #1\right\|}}
\newcommand{\dprod}[1]{\ensuremath{\langle#1\rangle}}
\newcommand{\linspan}[1]{\ensuremath{\langle#1\rangle}}
\newcommand{\conj}[1]{\ensuremath{\overline{#1}}}
\newcommand{\der}{\ensuremath{\frac{d}{dx}}}
\newcommand{\lap}{\ensuremath{\Delta}}
\newcommand{\kron}{\ensuremath{\otimes}}
\newcommand{\nperp}{\ensuremath{\nvdash}}

\newcommand{\mat}[1]{\left[ \begin{smallmatrix}#1 \end{smallmatrix} \right]}

% derivatives and limits
\newcommand{\partder}[2]{\ensuremath{\frac{\partial #1}{\partial #2}}}
\newcommand{\partdern}[3]{\ensuremath{\frac{\partial^{#3} #1}{\partial #2^{#3}}}}
\newcommand{\gradient}{\ensuremath{\nabla}}
\newcommand{\subdifferential}{\ensuremath{\partial}}

% Arrows
\newcommand{\diverge}{\ensuremath{\nearrow}}
\newcommand{\notto}{\ensuremath{\nrightarrow}}
\newcommand{\up}{\ensuremath{\uparrow}}
\newcommand{\down}{\ensuremath{\downarrow}}
% gets and gives are defined!

% ordering operators
\newcommand{\oleq}{\preceq}
\newcommand{\ogeq}{\succeq}

% programming and logic operators
\newcommand{\dfn}{:=}
\newcommand{\assign}{:=}
\newcommand{\co}{\ co\ }
\newcommand{\en}{\ en\ }


% logic operators
\newcommand{\xor}{\ensuremath{\oplus}}
\newcommand{\Land}{\ensuremath{\bigwedge}}
\newcommand{\Lor}{\ensuremath{\bigvee}}
\newcommand{\finish}{\ensuremath{\Box}}
\newcommand{\contra}{\ensuremath{\Rightarrow \Leftarrow}}
\newcommand{\iseq}{\ensuremath{\stackrel{_?}{=}}}


% Set theory
\newcommand{\symdiff}{\ensuremath{\Delta}}
\newcommand{\union}{\ensuremath{\cup}}
\newcommand{\inters}{\ensuremath{\cap}}
\newcommand{\Union}{\ensuremath{\bigcup}}
\newcommand{\Inters}{\ensuremath{\bigcap}}
\newcommand{\nullSet}{\ensuremath{\phi}}


% graph theory
\newcommand{\nbd}{\Gamma}

% Script alphabets
% For reals, use \Re

% greek letters
\newcommand{\eps}{\ensuremath{\epsilon}}
\newcommand{\del}{\ensuremath{\delta}}
\newcommand{\ga}{\ensuremath{\alpha}}
\newcommand{\gb}{\ensuremath{\beta}}
\newcommand{\gd}{\ensuremath{\del}}
\newcommand{\gp}{\ensuremath{\pi}}
\newcommand{\gf}{\ensuremath{\phi}}
\newcommand{\gh}{\ensuremath{\eta}}
\newcommand{\gF}{\ensuremath{\Phi}}
\newcommand{\gl}{\ensuremath{\lambda}}
\newcommand{\gm}{\ensuremath{\mu}}
\newcommand{\gn}{\ensuremath{\nu}}
\newcommand{\gr}{\ensuremath{\rho}}
\newcommand{\gs}{\ensuremath{\sigma}}
\newcommand{\gt}{\ensuremath{\theta}}
\newcommand{\gx}{\ensuremath{\xi}}

\newcommand{\sw}{\ensuremath{\sigma}}
\newcommand{\SW}{\ensuremath{\Sigma}}
\newcommand{\ew}{\ensuremath{\lambda}}
\newcommand{\EW}{\ensuremath{\Lambda}}

\newcommand{\Del}{\ensuremath{\Delta}}
\newcommand{\gD}{\ensuremath{\Delta}}
\newcommand{\gG}{\ensuremath{\Gamma}}
\newcommand{\gO}{\ensuremath{\Omega}}
\newcommand{\gS}{\ensuremath{\Sigma}}

% Bold english letters.
\newcommand{\bA}{\ensuremath{\mathbf{A}}}
\newcommand{\bB}{\ensuremath{\mathbf{B}}}
\newcommand{\bC}{\ensuremath{\mathbf{C}}}
\newcommand{\bD}{\ensuremath{\mathbf{D}}}
\newcommand{\bE}{\ensuremath{\mathbf{E}}}
\newcommand{\bF}{\ensuremath{\mathbf{F}}}
\newcommand{\bG}{\ensuremath{\mathbf{G}}}
\newcommand{\bH}{\ensuremath{\mathbf{H}}}
\newcommand{\bI}{\ensuremath{\mathbf{I}}}
\newcommand{\bJ}{\ensuremath{\mathbf{J}}}
\newcommand{\bK}{\ensuremath{\mathbf{K}}}
\newcommand{\bL}{\ensuremath{\mathbf{L}}}
\newcommand{\bM}{\ensuremath{\mathbf{M}}}
\newcommand{\bN}{\ensuremath{\mathbf{N}}}
\newcommand{\bO}{\ensuremath{\mathbf{O}}}
\newcommand{\bP}{\ensuremath{\mathbf{P}}}
\newcommand{\bQ}{\ensuremath{\mathbf{Q}}}
\newcommand{\bR}{\ensuremath{\mathbf{R}}} % PROSPER defines \R
\newcommand{\bS}{\ensuremath{\mathbf{S}}}
\newcommand{\bT}{\ensuremath{\mathbf{T}}}
\newcommand{\bU}{\ensuremath{\mathbf{U}}}
\newcommand{\bV}{\ensuremath{\mathbf{V}}}
\newcommand{\bW}{\ensuremath{\mathbf{W}}}
\newcommand{\bX}{\ensuremath{\mathbf{X}}}
\newcommand{\bY}{\ensuremath{\mathbf{Y}}}
\newcommand{\bZ}{\ensuremath{\mathbf{Z}}}
\newcommand{\bba}{\ensuremath{\mathbf{a}}}
\newcommand{\bbb}{\ensuremath{\mathbf{b}}}
\newcommand{\bbc}{\ensuremath{\mathbf{c}}}
\newcommand{\bbd}{\ensuremath{\mathbf{d}}}
\newcommand{\bbe}{\ensuremath{\mathbf{e}}}
\newcommand{\bbf}{\ensuremath{\mathbf{f}}}
\newcommand{\bbg}{\ensuremath{\mathbf{g}}}
\newcommand{\bbh}{\ensuremath{\mathbf{h}}}
\newcommand{\bbk}{\ensuremath{\mathbf{k}}}
\newcommand{\bbl}{\ensuremath{\mathbf{l}}}
\newcommand{\bbm}{\ensuremath{\mathbf{m}}}
\newcommand{\bbn}{\ensuremath{\mathbf{n}}}
\newcommand{\bbp}{\ensuremath{\mathbf{p}}}
\newcommand{\bbq}{\ensuremath{\mathbf{q}}}
\newcommand{\bbr}{\ensuremath{\mathbf{r}}}
\newcommand{\bbs}{\ensuremath{\mathbf{s}}}  % TIPA defines \s and LaTeX \ss!
\newcommand{\bbt}{\ensuremath{\mathbf{t}}}
\newcommand{\bbu}{\ensuremath{\mathbf{u}}}
\newcommand{\bbv}{\ensuremath{\mathbf{v}}}
\newcommand{\bbw}{\ensuremath{\mathbf{w}}}
\newcommand{\bbx}{\ensuremath{\mathbf{x}}}
\newcommand{\bby}{\ensuremath{\mathbf{y}}}
\newcommand{\bbz}{\ensuremath{\mathbf{z}}}
\newcommand{\0}{\ensuremath{\mathbf{0}}}
\newcommand{\1}{\ensuremath{\mathbf{1}}}


% Formatting shortcuts
\newcommand{\red}[1]{\textcolor{red}{#1}}
\newcommand{\green}[1]{\textcolor{green}{#1}}
\newcommand{\magenta}[1]{\textcolor{magenta}{#1}}
\newcommand{\orange}[1]{\textcolor{orange}{#1}}
\newcommand{\gray}[1]{\textcolor{gray}{#1}}
\newcommand{\blue}[1]{\textcolor{blue}{#1}}
\newcommand{\htext}[2]{\texorpdfstring{#1}{#2}}

% Statistics
\newcommand{\distr}{\ensuremath{\sim}}
\newcommand{\stddev}{\ensuremath{\sigma}}
\newcommand{\covmatrix}{\ensuremath{\Sigma}}
\newcommand{\mean}{\ensuremath{\mu}}
\newcommand{\param}{\ensuremath{\gt}}
\newcommand{\ftr}{\ensuremath{\phi}}

% General utility
\newcommand{\todo}[1]{\textbf{[TODO]}] \footnote{TODO: #1}}
\newcommand{\exclaim}[1]{{\textbf{\textit{#1}}}}
\newcommand{\tbc}{[\textbf{Incomplete}]}
\newcommand{\chk}{[\textbf{Check}]}
\newcommand{\oprob}{[\textbf{OP}]:}
\newcommand{\core}[1]{\textbf{Core Idea:}}
\newcommand{\why}{[\textbf{Find proof}]}
\newcommand{\opt}[1]{\textit{#1}}


\renewcommand{\~}{\htext{$\sim$}{~}}

\DeclareMathOperator*{\argmin}{arg\,min}
\DeclareMathOperator*{\argmax}{arg\,max}

% Items pertaining to the headed list.
\newcommand{\headeditem}[1]{\item\textbf{#1}}
\newcommand{\headedsubitem}[1]{\subitem\textbf{#1}}



% groupings of objects.
\newcommand{\set}[1]{\ensuremath{\left\{ #1 \right\}}}
\newcommand{\seq}[1]{\ensuremath{\left(#1\right)}}
\newcommand{\ang}[1]{\ensuremath{\langle#1\rangle}}
\newcommand{\tuple}[1]{\ensuremath{\left(#1\right)}}
\newcommand{\size}[1]{\ensuremath{\left| #1\right|}}

% numerical shortcuts.
\newcommand{\abs}[1]{\ensuremath{\left| #1\right|}}
\newcommand{\floor}[1]{\ensuremath{\left\lfloor #1 \right\rfloor}}
\newcommand{\ceil}[1]{\ensuremath{\left\lceil #1 \right\rceil}}

% linear algebra shortcuts.
\newcommand{\change}{\ensuremath{\Delta}}
\newcommand{\norm}[1]{\ensuremath{\left\| #1\right\|}}
\newcommand{\dprod}[1]{\ensuremath{\langle#1\rangle}}
\newcommand{\linspan}[1]{\ensuremath{\langle#1\rangle}}
\newcommand{\conj}[1]{\ensuremath{\overline{#1}}}
\newcommand{\der}{\ensuremath{\frac{d}{dx}}}
\newcommand{\lap}{\ensuremath{\Delta}}
\newcommand{\kron}{\ensuremath{\otimes}}
\newcommand{\nperp}{\ensuremath{\nvdash}}

\newcommand{\mat}[1]{\left[ \begin{smallmatrix}#1 \end{smallmatrix} \right]}

% derivatives and limits
\newcommand{\partder}[2]{\ensuremath{\frac{\partial #1}{\partial #2}}}
\newcommand{\partdern}[3]{\ensuremath{\frac{\partial^{#3} #1}{\partial #2^{#3}}}}
\newcommand{\gradient}{\ensuremath{\nabla}}
\newcommand{\subdifferential}{\ensuremath{\partial}}

% Arrows
\newcommand{\diverge}{\ensuremath{\nearrow}}
\newcommand{\notto}{\ensuremath{\nrightarrow}}
\newcommand{\up}{\ensuremath{\uparrow}}
\newcommand{\down}{\ensuremath{\downarrow}}
% gets and gives are defined!

% ordering operators
\newcommand{\oleq}{\preceq}
\newcommand{\ogeq}{\succeq}

% programming and logic operators
\newcommand{\dfn}{:=}
\newcommand{\assign}{:=}
\newcommand{\co}{\ co\ }
\newcommand{\en}{\ en\ }


% logic operators
\newcommand{\xor}{\ensuremath{\oplus}}
\newcommand{\Land}{\ensuremath{\bigwedge}}
\newcommand{\Lor}{\ensuremath{\bigvee}}
\newcommand{\finish}{\ensuremath{\Box}}
\newcommand{\contra}{\ensuremath{\Rightarrow \Leftarrow}}
\newcommand{\iseq}{\ensuremath{\stackrel{_?}{=}}}


% Set theory
\newcommand{\symdiff}{\ensuremath{\Delta}}
\newcommand{\union}{\ensuremath{\cup}}
\newcommand{\inters}{\ensuremath{\cap}}
\newcommand{\Union}{\ensuremath{\bigcup}}
\newcommand{\Inters}{\ensuremath{\bigcap}}
\newcommand{\nullSet}{\ensuremath{\phi}}


% graph theory
\newcommand{\nbd}{\Gamma}

% Script alphabets
% For reals, use \Re

% greek letters
\newcommand{\eps}{\ensuremath{\epsilon}}
\newcommand{\del}{\ensuremath{\delta}}
\newcommand{\ga}{\ensuremath{\alpha}}
\newcommand{\gb}{\ensuremath{\beta}}
\newcommand{\gd}{\ensuremath{\del}}
\newcommand{\gp}{\ensuremath{\pi}}
\newcommand{\gf}{\ensuremath{\phi}}
\newcommand{\gh}{\ensuremath{\eta}}
\newcommand{\gF}{\ensuremath{\Phi}}
\newcommand{\gl}{\ensuremath{\lambda}}
\newcommand{\gm}{\ensuremath{\mu}}
\newcommand{\gn}{\ensuremath{\nu}}
\newcommand{\gr}{\ensuremath{\rho}}
\newcommand{\gs}{\ensuremath{\sigma}}
\newcommand{\gt}{\ensuremath{\theta}}
\newcommand{\gx}{\ensuremath{\xi}}

\newcommand{\sw}{\ensuremath{\sigma}}
\newcommand{\SW}{\ensuremath{\Sigma}}
\newcommand{\ew}{\ensuremath{\lambda}}
\newcommand{\EW}{\ensuremath{\Lambda}}

\newcommand{\Del}{\ensuremath{\Delta}}
\newcommand{\gD}{\ensuremath{\Delta}}
\newcommand{\gG}{\ensuremath{\Gamma}}
\newcommand{\gO}{\ensuremath{\Omega}}
\newcommand{\gS}{\ensuremath{\Sigma}}

% Bold english letters.
\newcommand{\bA}{\ensuremath{\mathbf{A}}}
\newcommand{\bB}{\ensuremath{\mathbf{B}}}
\newcommand{\bC}{\ensuremath{\mathbf{C}}}
\newcommand{\bD}{\ensuremath{\mathbf{D}}}
\newcommand{\bE}{\ensuremath{\mathbf{E}}}
\newcommand{\bF}{\ensuremath{\mathbf{F}}}
\newcommand{\bG}{\ensuremath{\mathbf{G}}}
\newcommand{\bH}{\ensuremath{\mathbf{H}}}
\newcommand{\bI}{\ensuremath{\mathbf{I}}}
\newcommand{\bJ}{\ensuremath{\mathbf{J}}}
\newcommand{\bK}{\ensuremath{\mathbf{K}}}
\newcommand{\bL}{\ensuremath{\mathbf{L}}}
\newcommand{\bM}{\ensuremath{\mathbf{M}}}
\newcommand{\bN}{\ensuremath{\mathbf{N}}}
\newcommand{\bO}{\ensuremath{\mathbf{O}}}
\newcommand{\bP}{\ensuremath{\mathbf{P}}}
\newcommand{\bQ}{\ensuremath{\mathbf{Q}}}
\newcommand{\bR}{\ensuremath{\mathbf{R}}} % PROSPER defines \R
\newcommand{\bS}{\ensuremath{\mathbf{S}}}
\newcommand{\bT}{\ensuremath{\mathbf{T}}}
\newcommand{\bU}{\ensuremath{\mathbf{U}}}
\newcommand{\bV}{\ensuremath{\mathbf{V}}}
\newcommand{\bW}{\ensuremath{\mathbf{W}}}
\newcommand{\bX}{\ensuremath{\mathbf{X}}}
\newcommand{\bY}{\ensuremath{\mathbf{Y}}}
\newcommand{\bZ}{\ensuremath{\mathbf{Z}}}
\newcommand{\bba}{\ensuremath{\mathbf{a}}}
\newcommand{\bbb}{\ensuremath{\mathbf{b}}}
\newcommand{\bbc}{\ensuremath{\mathbf{c}}}
\newcommand{\bbd}{\ensuremath{\mathbf{d}}}
\newcommand{\bbe}{\ensuremath{\mathbf{e}}}
\newcommand{\bbf}{\ensuremath{\mathbf{f}}}
\newcommand{\bbg}{\ensuremath{\mathbf{g}}}
\newcommand{\bbh}{\ensuremath{\mathbf{h}}}
\newcommand{\bbk}{\ensuremath{\mathbf{k}}}
\newcommand{\bbl}{\ensuremath{\mathbf{l}}}
\newcommand{\bbm}{\ensuremath{\mathbf{m}}}
\newcommand{\bbn}{\ensuremath{\mathbf{n}}}
\newcommand{\bbp}{\ensuremath{\mathbf{p}}}
\newcommand{\bbq}{\ensuremath{\mathbf{q}}}
\newcommand{\bbr}{\ensuremath{\mathbf{r}}}
\newcommand{\bbs}{\ensuremath{\mathbf{s}}}  % TIPA defines \s and LaTeX \ss!
\newcommand{\bbt}{\ensuremath{\mathbf{t}}}
\newcommand{\bbu}{\ensuremath{\mathbf{u}}}
\newcommand{\bbv}{\ensuremath{\mathbf{v}}}
\newcommand{\bbw}{\ensuremath{\mathbf{w}}}
\newcommand{\bbx}{\ensuremath{\mathbf{x}}}
\newcommand{\bby}{\ensuremath{\mathbf{y}}}
\newcommand{\bbz}{\ensuremath{\mathbf{z}}}
\newcommand{\0}{\ensuremath{\mathbf{0}}}
\newcommand{\1}{\ensuremath{\mathbf{1}}}


% Formatting shortcuts
\newcommand{\red}[1]{\textcolor{red}{#1}}
\newcommand{\green}[1]{\textcolor{green}{#1}}
\newcommand{\magenta}[1]{\textcolor{magenta}{#1}}
\newcommand{\orange}[1]{\textcolor{orange}{#1}}
\newcommand{\gray}[1]{\textcolor{gray}{#1}}
\newcommand{\blue}[1]{\textcolor{blue}{#1}}
\newcommand{\htext}[2]{\texorpdfstring{#1}{#2}}

% Statistics
\newcommand{\distr}{\ensuremath{\sim}}
\newcommand{\stddev}{\ensuremath{\sigma}}
\newcommand{\covmatrix}{\ensuremath{\Sigma}}
\newcommand{\mean}{\ensuremath{\mu}}
\newcommand{\param}{\ensuremath{\gt}}
\newcommand{\ftr}{\ensuremath{\phi}}

% General utility
\newcommand{\todo}[1]{\textbf{[TODO]}] \footnote{TODO: #1}}
\newcommand{\exclaim}[1]{{\textbf{\textit{#1}}}}
\newcommand{\tbc}{[\textbf{Incomplete}]}
\newcommand{\chk}{[\textbf{Check}]}
\newcommand{\oprob}{[\textbf{OP}]:}
\newcommand{\core}[1]{\textbf{Core Idea:}}
\newcommand{\why}{[\textbf{Find proof}]}
\newcommand{\opt}[1]{\textit{#1}}


\renewcommand{\~}{\htext{$\sim$}{~}}

\DeclareMathOperator*{\argmin}{arg\,min}
\DeclareMathOperator*{\argmax}{arg\,max}

% Items pertaining to the headed list.
\newcommand{\headeditem}[1]{\item\textbf{#1}}
\newcommand{\headedsubitem}[1]{\subitem\textbf{#1}}



\newtheorem{thm}{Theorem}[subsection]
\newtheorem{cor}[thm]{Corollary}
\newtheorem{lem}[thm]{Lemma}

\theoremstyle{remark}
\newtheorem{defn}[thm]{Definition}
\newtheorem*{notation}{Notation}
\newtheorem{alg}[thm]{Algorithm}
\newtheorem{rem}[thm]{Remark}

%opening
\title{Linear Algebra: Answer to Homework 4}
\author{vishvAs vAsuki}

\begin{document}

\maketitle

\section{8.2}
Write a MATLAB function [Q,R] = mgs(A) which returns the reduced QR factorization. Output variables are Q and R.

Code was emailed to the TA.

\verbatiminput{../hwCode/mgs.m}

\section{10.2}
\subsection{a}
A is $m \times n, m\geq  n$.

Write [W, R] = house(A). Use Householder reflections which computes the implicit representation of a full QR factorization of A. W is $m\times n$, lower triangular, its columns are the vectors $v_{k}$ defining the successive Householder reflections. R is a $n\times n$ triangular matrix.

Code was emailed to the TA.

\verbatiminput{../hwCode/house.m}

\subsection{b}
Write a MATLAB function Q = formQ(W) that takes the matrix W produced by house as input and generates a corresponding $m \times m$ orthogonal matrix Q.

Code was emailed to the TA.

\verbatiminput{../hwCode/formQ.m}


\section{11.3}
Very long matlab question comparing mgs, qr, house.

Let m = 50, n = 12. Use linspace, to define t = m-vector of linearly spaced grid points from 0 to 1. Use vander and fliplr to define A to be matrix associated with least squares fitting on thi sgrid by polynomial of degree n-1. b = cos(4t). Print to 16 digit precision the least squares coefficient vector x by six methods.

\subsection{a} Solving normal equations using Matlab's $\backslash$.

\begin{verbatim}
   1.000000009828539
  -0.000003058556934
  -7.999881599423152
  -0.001794968011434
  10.680787070393082
  -0.065291258825856
  -5.500576191037059
  -0.344542637686066 :Inaccurate: compare with SVD result.
   2.012711257119547 :Inaccurate: compare with SVD result.
  -0.229531584456158 :Inaccurate: compare with SVD result.
  -0.275744751507555
   0.070224085955164
\end{verbatim}


\subsection{b} Using QR factorization by mgs.
\begin{verbatim}
   0.999999996532329
   0.000000389043255
  -8.000011138708601
   0.000118975453461
  10.666096448792814
   0.001163513676147
  -5.689404852048028
   0.001960173714906
   1.602552673895843
   0.072894805227406 :Inaccurate: compare with SVD result.
  -0.402066671638750 :Inaccurate: compare with SVD result.
   0.093052068368706
\end{verbatim}

\subsection{c} Using QR factorization by house.
\begin{verbatim}
   1.000000000996605
  -0.000000422742926
  -7.999981235688280
  -0.000318763197465
  10.669430795645004
  -0.013820286887481
  -5.647075630351267
  -0.075316019060198
   1.693606957535560
   0.006032112954680
  -0.374241705130035
   0.088040576364041
\end{verbatim}

\subsection{d} Using QR factorization by qr.
\begin{verbatim}
   1.000000000996609
  -0.000000422742968
  -7.999981235687960
  -0.000318763203417
  10.669430795705138
  -0.013820287197629
  -5.647075629418464
  -0.075316020815308
   1.693606959644216
   0.006032111376332
  -0.374241704457822
   0.088040576239507
\end{verbatim}

\subsection{e} $x=A\backslash b$ in Matlab.
\begin{verbatim}
   1.000000000996608
  -0.000000422743228
  -7.999981235679865
  -0.000318763301264
  10.669430796332966
  -0.013820289609699
  -5.647075623539531
  -0.075316030120297
   1.693606969166215
   0.006032105309351
  -0.374241702274479
   0.088040575901462
\end{verbatim}

\subsection{f} Matlab's svd.
\begin{verbatim}
   1.000000000996608
  -0.000000422742965
  -7.999981235688125
  -0.000318763200829
  10.669430795684868
  -0.013820287102762
  -5.647075629701436
  -0.075316020265539
   1.693606958952933
   0.006032111917519
  -0.374241704697128
   0.088040576285090
\end{verbatim}

\subsection{g} 
The numbers which appear very wrong ($error>0.1$) are marked above. Normal equations exhibit instability, as does the algorithm which uses mgs (to a lesser extant).

\section{10.1}
\subsection{a} Eigenvalues of a Hoseholder reflector F. Give geometric argument as well as algebraic proof.

\begin{thm}
 Let F be a $m \times m$ Householder reflector. The eigenvalues of F are +1 and -1.
\end{thm}
\begin{rem}
 Geometric argument: F reflects all vectors in $C^{n}$ accross a (m-1) dimensional hyperplane, P. For vectors x inside P, F has no effect as $Fx = x$. This eigenvalue has multiplicity n-1, as the corresponding eigenspace is m-1 dimensional. For all vectors passing through the 1 dimensional vector space orthogonal to P, $Fx = -x$. So, -1 is the only other eigenvalue. It has multiplicity 1.
\end{rem}
\begin{proof}
The proof follows from the following lemmata. As 1 is an eigenvalue of $n \times n$ F with multiplicity n-1 and -1 is another eigenvalue with multiplicity atleast 1, there are no other eigenvalues.
\end{proof}

\begin{lem}
 1 is an eigenvalue of $F = (I-2vv^{*})x$. If F is $n\times n$, 1 has multiplicity n-1.
\end{lem}
\begin{proof}
Fx = x whenever $v^{*}x$ = 0, where k is a constant:
\begin{eqnarray} 
(I-2vv^{*})x &=& x -2vv^{*}x\\
&=& x \texttt{ as $v^{*}x$ = 0}\\
\end{eqnarray}
If F is $n\times n$, the multiples of v or kv can cover only a space of 1 dimension. So, the dimension of the space perpendicular to v is n-1. Hence, the eigenspace is n-1 dimensional, and the multiplicity of the eigenvalue 1 is n-1.
\end{proof}

\begin{lem}
 -1 is an eigenvector of $F = (I-2vv^{*})x$.
\end{lem}
\begin{proof}
Fx = -x whenever x = kv, where k is a constant:
\begin{eqnarray} 
(I-2vv^{*})kv &=& kv -2kvv^{*}v\\
&=& kv -2kv\\
&=& -kv\\
\end{eqnarray}
So, -1 is an eigenvalue.
\end{proof}


\subsection{b}
\begin{thm}
 The determinant of a m*m Hoseholder reflector F is -1.
\end{thm}
\begin{proof}
 As proved in the previous theorem, the eigenvalues of F are 1 (with multiplicity m-1) and -1. As the determinant of a matrix is equal to the determinant of its Eigenvalue matrix, det(F)=-1.
\end{proof}

\subsection{c}
\begin{thm}
 Singular Values of a m*m Hoseholder reflector F are all 1.
\end{thm}
\begin{rem}
 Geometric argument: The Householder reflector maps vectors in the unit sphere to other vectors of equal length in the unit sphere.
\end{rem}
\begin{proof}
Proof by construction of SVD.

$Let F = I - 2vv^{*}$.

We have shown in an earlier lemma (regarding eigenvalues of F) that Fx = x for all $x \perp v$, and that such x span a n-1 dimensional space. One can use the Gram schmidt method to find the orthogonal basis $F_{1}$, which is $m\times m$, but has rank (m-1), for such a space.

We have shown in an earlier lemma that Fx= -x when x=kv , where k is a constant. So, we observe that F(-v) = v. We note that -v $\perp$ column space $C(F_{1})$.

So, we can construct the matrix $U=[F_{1}\ (-v)]$, and $V = [F_{1}\ v]$. Then, we see that FU=VI. Hence, we have the SVD of F, and all singular values of F are 1.
\end{proof}


\section{10.4}
s= sin t, c = cos t for some t. 

F:
\begin{verbatim}
-c s
 s c
\end{verbatim}
det(F) = -1. F is a reflector, a special case of a Householder reflector in $C^{2}$.


J:
\begin{verbatim}
 c s
-s c
\end{verbatim}
det(J) = 1. J is a rotator. Called 'Givens rotation'.

\subsection{a}
\begin{rem}
Geometric effects of left multiplication by F on the plane $R^{2}$:
Upon solving the equation Fx = x in order to find the reflecting line, we find $x^{*}$ = k[$\frac{1-c}{s}$ 1], where k is any scalar constant. Thus, F reflects every vector in $R^{2}$ accross the space (line) spanned by this vector.
\end{rem}

\begin{rem}
 Geometric effects of left multiplications by J on the plane $R^{2}$: Consider a vector $x^{*} = [\norm{x}\cos A \   \norm{x}\sin A]$. We see that $(Jx)^{*} = \norm{x}[\cos(A-t)\ \sin (A-t)]$. So, J rotates vectors clockwise if t is positive, and counterclockwise if t is negative.
\end{rem}

\subsection{b}
Make an algorithm for QR factorization that is analogous to Alg 10 (Triangularizing by introducing zeros), but based on Givens rotations instead of Householder reflections.

\begin{alg}\textbf{findJ}\\
Input: Vector x in $R^{2}$.\\
Output: Givens rotation matrix J which takes x to $\norm{x}e_{1}$.\\
\begin{itemize}
\item Let $c = \frac{x_{1}}{\norm{x}}$.
\item $s=\frac{x_{2}}{\norm{x}}$.
\item Make matrix J with s and c.
\end{itemize}
\end{alg}

\begin{alg} \textbf{makeR}\\
for k=1 to n\\
for i = m-1 to k\\
\begin{itemize}
\item $x = A_{i:i+1,k}$.
\item Find the corresponding $J_{k,i}$ using findJ.
\item $A_{i:i+1,k:n} = J_{k,i} A_{i:i+1,k:n}$.
\end{itemize}
\end{alg}

\subsection{c}
\begin{thm}
 \textbf{makeR} involves 6 flops per entry operated on rather than 4; asymptotic operation count is 50\% greater than work for householder orthogonalization.
\end{thm}
\begin{proof}
Take arbitrary $A_{i,j}$, such that it is not in the first row of A. Suppose that the index in the outermost loop is $k<j$. Now, $A_{i,j}$ is involved in two multiplications: $A_{i:i+1,j}=J_{k,i}A_{i:i+1,j}$ and $A_{i-1:i,j}=J_{k,i-1}A_{i:i,j}$. Each of these matrix multiplications involve 2 multiplications and 1 addition per entry. These operations dominate the others when m and n are large. So, asymptotically, 6 flops are required per entry, compared to 4 flops per entry required in Householder orthogonalization. Hence, 50\% more work is required.

\begin{rem}
findJ itself requires 6 flops, but this is compensated by the fact that $A_{i:i+1,k} = J_{k,i}A_{i:i+1,k}$ can be direcly assigned to $(\norm{A_{i:i+1,k}}, 0)$, without requiring further calculation.
\end{rem}
\end{proof}

% \bibliographystyle{plain}
% \bibliography{../linAlg}


\end{document}
