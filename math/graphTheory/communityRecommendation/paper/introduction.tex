There has been an explosion in the number of online social networks and their active members. This wealth of information in the social networks has driven prolific work on the analysis of the networks, understanding the processes that explain the evolution of the networks, modeling the spread of behavior through the networks, predicting their future state and so on.

Users of a social network tend to affiliate with communities. In some social networks, the groups are identified more by the preferences of the members of the social network than by direct declaration: e.g. the genre of movies that a set of customers tend to patronize more often in Netflix. Online social networks like Facebook, Orkut and Live Journal are more interesting examples because the affiliation networks here are explicitly established by the members of the network. Thus, two networks exist simultaneously: the friendship network among users, and the affiliation network between users and groups.

\paragraph*{The problem}
Group formation and evolution in social networks\cite{GroupFormation}, and co-evolution of social and affiliation networks\cite{Coevolution} have been recently studied. One of the interesting challenges in social network analysis is the affiliation recommendation problem, where the task is to recommend communities to users. The fact that the social and affiliation networks ``co-evolve'' suggests that a better solution to the affiliation recommendation problem can be obtained if the friendship network is considered along with the affiliation network. This problem setting has applications beyond community recommendation. Affiliations, for example, can be interpreted in general as a user's taste for an item. Neither is it limited to social networks. For example, in biology, the friendship network can correspond to a network among genes, whereas the affiliation network can correspond to a network between genes and traits/ diseases, and the affiliation recommendation problem can be viewed as one of identifying genes affecting the expression of a disease.

\paragraph*{Contributions}
We consider how one can model the interplay between users and communities in both the networks simultaneously. An ideal unifying model would not only explain the current state of the networks, but also help in predicting future relationships among the nodes. Using a simple way of combining these networks, we suggest and explore two ways of modeling the networks for the purpose of making affiliation recommendations. The graph proximity model is based on estimating the affinity between a user and a community by considering their proximity as nodes in a combined graph, while the latent factors model is based on the proposition that community affiliations arise from interactions of \textit{user} and \textit{group factors}. Each of these network models suggests affiliation recommendation algorithms.

We evaluate these algorithms on social networks from Orkut consisting of 9,123 users and 75,546 communities, and Youtube consisting of 16,575 users and 21,326 communities. We propose a way of evaluating affiliation recommendations, by measuring how good the top 50 recommendations per user are, and demonstrate the importance of designing the right evaluation strategy. Of the algorithms proposed, those suggested by the graph proximity model turn out to be the most effective and efficient. This use of link prediction techniques for the purpose of affiliation recommendation is, to our knowledge, novel. We show that information in the friendship network can be used effectively for affiliation recommendation. We also observe that the benefit we derive from the social network in affiliation recommendation is strongly contingent on how the problem is modeled and what algorithms are used.

\paragraph*{Overview}
We now provide a brief overview of the organization of the paper. In Section \ref{Models}, we consider a network formed by merging the friendship and affiliation networks and introduce two models of the behaviour of nodes in this network: the graph proximity model, and the latent factors model, and explore recommendation algorithms that arise from these models. In Section \ref{Related work}, we consider how the proposed models and algorithms relate to prior work. Then, in Section \ref{Experimental Evaluation}, we describe our chosen evaluation strategy, and using experiments on real world networks, we evaluate the effectiveness of various algorithms in affiliation recommendation in practical scenarios. Finally, in Section \ref{Conclusion and Future Work}, we conclude with a summary of our findings, and a discussion of future lines of research in this direction.

