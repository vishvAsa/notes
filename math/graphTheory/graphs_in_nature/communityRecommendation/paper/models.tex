In this section, we first establish the notation used, and then pose the affiliation recommendation problem as a \emph{ranking} problem. Then we describe a natural way of combining the friendship and affiliation networks into a single graph. In the subsections that follow, we describe affiliation recommendation approaches based on the graph proximity model and those based on the latent factors model.

\paragraph*{Notation}
We use $N_u$ to denote the total number of users in the affiliation network and $N_g$ to denote the total number of communities in the affiliation network. $\A \in \mathbb{R}^{N_u \times N_g}$ denotes the user $\times$ group adjacency matrix of affiliation network $\calA$. $\SS \in \mathbb{R}^{N_u \times N_u}$ denotes the user $\times$ user adjacency matrix of friendship network $\calS$. Other notation will be introduced as needed.

\paragraph*{As a ranking problem}
The task is to recommend affiliations to a given user. This problem can be posed as a problem of ranking various affiliations in the order of the user's interest in joining them. The methods we describe here to solve this problem rely on assigning scores to various affiliations in order to rank them. The task of an affiliation recommendation algorithm can be viewed as one of generating an $N_u \times N_g$ score matrix. 

\subsection{Prediction on the Combined Graph}
Consider the adjacency matrices $\A$ and $\SS$. For now, we assume $\SS$ to be symmetric (or equivalently $\calS$ to be undirected), although a non-symmetric extension to our model can be easily obtained. Clearly, $\SS$ corresponds to an undirected graph among users and $\begin{bmatrix}
 0 & \A \\
\A^T & 0
\end{bmatrix}$
corresponds to an undirected bipartite graph between users and groups. Despite the heterogeneity of the two types of ``links", it is rather natural to consider a graph $\calC$ between all the users and groups, with the combined adjacency matrix
\[\C =
\begin{bmatrix}
\lambda \SS & \A \\
\A^T & 0
\end{bmatrix},
\label{e:C}
\]
where the heterogeneity of two types of links is reduced to a single parameter $\gl \geq 0$, that controls the ratio of the weight of friendship to the weight of group membership. Clearly when $\gl = 0$, the user-user friendship ceases to play a role in this joint graph, and it simply degenerates to the bipartite affiliation graph.

As in the case of prediction with a single graph, the prediction on $\calC$ can be carried out from the following two perspectives:
\begin{itemize}
\item {\bf Graph Proximity Model:} We assume that the entire graph is known, and that the prediction of new or unobserved links is based on an estimated proximity in $\C$ between nodes.
\item {\bf Latent Factors Model:} We model the graph as a matrix, some of whose zero entries are actually ones, while modelling all entries as the product of latent factors.
\end{itemize}
We will elaborate on the two perspectives in the following two subsections.

\subsubsection{Graph Proximity Model}
\label{Graph Proximity Model}
As described earlier, the affiliation network can be modelled by a graph. The graph proximity model assumes that the probability of there arising a link between two nodes in a graph is based on an estimated proximity between the two nodes. The proximity of two vertices can be calculated as the weighted sum of the number of paths connecting the two with varying lengths. This is the underlying mechanism of many link prediction models in the context of social network analysis. Consider the widely-used \textsf{Katz} measure\cite{KleinbergLinkPred} on the friendship network $\calS$. The proximity is given by
\[
\textsf{Katz}(\SS; \beta) = \beta \SS + \beta^2 \SS^2 + \beta^3 \SS^3 + \cdots.
\]
where the weights of paths decay exponentially with the length.

A simple extension of \textsf{Katz} to the bipartite graph $\calA$ is
\begin{eqnarray}
\textsf{Katz}(\A; \beta) \hspace{-6pt} &= & \hspace{-6pt}(\beta \A\A^T + \beta^2 (\A\A^T)^2 + \beta^3 (\A\A^T)^3 + \cdots) \A \nonumber \\
\hspace{-6pt} &= & \hspace{-6pt} \beta \A\A^T\A + \beta^2 (\A\A^T)^2\A + \cdots.
\label{e:extKatz}
\end{eqnarray}
where in the co-occurrence matrix $\A\A^T$, two users $i$ and $j$ are considered connected if $i$ and $j$ belong to at least one same group, i.e. $(\A\A^T)_{ij} > 0$. In this case, we consider the paths of the following types
\[
\text{user $i \, \xrightarrow{\A\A^T}\, j \,\cdots \xrightarrow{\A\A^T}\, k \,  \xrightarrow{\A}$ to group $n$}
\]
The intuition in considering $\A\A^T\A$ is if user $i$ shares some community with user $j$, it is likely that $i$ will join some other community $j$ belongs to. The higher order terms can be interpreted in a similar way.

Consider the following Katz-measure proximity matrix on the combined graph $\calC$
\[
\mathsf{katz}(\calC; \beta) = \beta \C + \beta^2 \C^2 + \beta^3 \C^3 + \cdots.
\]
The user-community block of the above matrix is simply
\begin{equation}
\mathsf{katz}(\calC; \beta)_{12} = \beta \A + \beta^2 \lambda \SS \A +  \beta^3(\lambda^2 \SS^2 \A + \A \A^T \A) + \cdots
\label{e:combKatz}
\end{equation}
Clearly (\ref{e:combKatz}) generalizes the normal Katz on the bipartite graph $\A$
by also considering some paths like
\[
\text{user $i \xrightarrow{\SS}j\xrightarrow{\A}$ group $n$ (in $\C^2$)}
\]
and
\[
\text{user $i \xrightarrow{\SS}j\xrightarrow{\A\A^T}k \xrightarrow{\A}j$ group $n$ (in $\C^4$)}.
\]
$\mathsf{katz}(\calC; \beta)_{12}$ can then be used as the score matrix.

Since the computation of the Katz matrices as described above is expensive, we consider a truncated Katz matrix, $$tKatz(T, k, \gb) = \sum_{i=1}^{k} \gb^i T^i.$$ For computing the Katz matrices for the graphs described by $\A$ or $\C$, a conservative estimate of the computational cost is $O(N_u \times nnz)$, where $nnz$ is the number of non-zeros in $(\A\A^T)^k$.

\subsubsection{Latent Factors Model}
\label{Latent Factors Model}
We now describe the latent factors model, consider the optimization problem it tries to solve, and examine some of its properties. In this model, the zeros in the adjacency matrix of the affiliation network, $\A$ may be viewed as being unobserved entries with a huge prior belief in favor of them being actually zero. Every user $i$ and community $j$ are assumed to have a low dimensional representations $u_i,\ g_j$. The affinity of a user $i$ to a community $j$ is assumed to correspond to $u_i^{T} g_j$. In other words, users and communities with a high inner product are assumed to be connected to each other; that is to say: $$\A_{i, j} \approx u_i^T g_j.$$

In matrix form, we express this as
$$\A \approx \U^T \G,$$
where $\U \in \bbR^{d \times N_u}$ is the matrix of user factors and $\G \in \bbR^{d \times N_g}$ is the matrix of group factors. Note that $\text{rank}(\U) \leq d,\ \text{rank}(\G) \leq d,\ d \ll N_u, N_g$.

To get factors which account for both the observed entries in $\A$ as well as the interactions in $\SS$ we consider the following generalized combined network,
\begin{equation}
\C'(\lambda, \D) =
\begin{bmatrix}
\lambda \SS & \A \\
\A^T & \D
\end{bmatrix},
\label{e:generalizedCombined}
\end{equation}
where $\D$ is the \emph{derived} similarity between groups which is \emph{not} observed. Note that $\C = \C'(\lambda, 0)$. Also, we approximate $\C'$ as an interaction of user and group factors of the individual graphs. i.e.
\[\C'(\lambda, \D) \approx
\begin{bmatrix}
\U^T \\
\G^T
\end{bmatrix}
\begin{bmatrix}
\U & \G
\end{bmatrix}.
\]
The same set of user factors drive both the friendship and affiliation network creations. We therefore want to find the user factors $\U$ and the group factors $\G$ such that the reconstruction error on the observed entries in $\C'$,
\begin{equation}
\|\U^T \U -\lambda \SS \|^2 + 2\|\U^T \G -\A \|^2 + \|\G^T\G - \D\|^2
\label{e:objective}
\end{equation}
is minimized.

\paragraph*{Role of $\lambda$}
Intuitively, in equation (\ref{e:objective}), $\gl$ controls the contribution of $\SS$ in deciding the user factors. The larger the $\lambda$, the learned factor model describes the friendship network $\calS$ better, and correspondingly the affiliation network is described less well.

\paragraph*{Choice of D}
The derived similarity between the communities $\D$ in the combined matrix $\C'$ can be approximated using proximity between communities in the graph corresponding to $\C'$. It follows that a potential choice is $\D = \A^T \A$, which is simply the number of users which any two communities share. One may also consider weighing the contribution of $\D$ with $\gl$, which is the weight factor learnt for $\SS$. We will see later that the experiments suggest that the choice of $\D$ is not very important, and that the information from $\D$, when derived from A, is redundant.

\subsection{Solution to the optimization problem}
The analytical solution to the minimizing $\set{\U, \G}$ of the objective function in (\ref{e:objective}) is given by the SVD of $\C'$. Specifically,
\begin{eqnarray}
 \textbf{argmin}_{\U, \G} \{\|\U^T \U -\lambda \SS \|^2 + 2\|\U^T \G -\A \|^2 + \nonumber\\ \|\G^T\G - \D\|^2 \} =  \text{SVD}(\C', d)
\end{eqnarray}
where SVD($\C', d$) denotes the rank-$d$ singular value decomposition of matrix $\C'$, i.e, $\C' \approx \U \bSigma \G^T$, where
$\bSigma$ is the $d \times d$ matrix of singular values, $\U$ is the matrix of $d$ leading left singular vectors, and $\G$ is the matrix of $d$ leading right singular vectors. We can interpret the low rank approximation $\U \bSigma \G^T$ as the score matrix.
