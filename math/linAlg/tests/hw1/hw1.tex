\documentclass[10pt]{amsart}
\usepackage{amssymb}



\newtheorem{thm}{Theorem}[subsection]
\newtheorem{cor}[thm]{Corollary}
\newtheorem{lem}[thm]{Lemma}

\theoremstyle{remark}
\newtheorem{defn}[thm]{Definition}
\newtheorem*{notation}{Notation}
\newtheorem{alg}[thm]{Algorithm}
\newtheorem{rem}[thm]{Remark}

%opening
\title{Linear Algebra: Answer to Homework 1}
\author{vishvAs vAsuki}

\begin{document}

\maketitle

\section{Question}
2.1 Show that if a matrix A is both triangular and unitary, it is diagonal.

\subsection{Answer}

\begin{notation}
$a_{i,j}$ indicates the element of A at row i and column j. $a_{i}$ indicates the ith column of A.
\end{notation}

\begin{rem}
Note that A, being triangular, is square. Let A be a $m \times m$ matrix.
\end{rem}

\begin{lem}
If A is both upper triangular and unitary, then $a_{1, j} = 0$ for all $j \neq 1$. So, for all $j \neq 1$, $a_{j, 1} = a_{1, j} = 0$. That is, the non diagonal elements in the first column and row are 0.
\end{lem}
\begin{proof}
Consider any $j \neq 1$. As A is unitary, $A^{*}A=I$.

So, $a_{j}^{*}a_{1} = I_{j,1} = 0$.

So, $\sum_{i=1}^{m} a_{i, j}a_{i, 1} = 0$.

But, as A is upper triangular, for all $i > 1, a_{i, 1} = 0$.

So, $a_{1, j}a_{1, 1} = 0$. This holds even if $a_{1, 1} \neq 0$. Hence, $a_{1, j}=0$.
\end{proof}

\begin{lem}
Suppose that A is both upper triangular and unitary. Suppose that if $i \neq j$ and either $i \leq l$, or if $j \leq l$, $a_{i,j} = 0$. That is, the non diagonal elements in the first l columns and rows are 0. Then, all non-diagonal elements in the first l+1 columns and rows are 0.
\end{lem}
\begin{proof}
Consider any $j > l+1$. As A is unitary, $A^{*}A=I$.

So, $a_{j}^{*}a_{l+1} = I_{j,l+1} = 0$.

So, $\sum_{i=1}^{m} a_{i, j}a_{i, l+1} = 0$.

But, as A is upper triangular, for all $i > l+1, a_{i, l+1} = 0$. Also, from our assumption, for all $i \leq l, a_{i, l+1} = 0$.

So, $a_{l+1, j}a_{l+1, l+1} = 0$. This holds even if $a_{l+1, l+1} \neq 0$. Hence, $a_{l+1, j}=0$. Thence, we have the result.
\end{proof}

\begin{thm}
If a matrix A is both upper triangular and unitary, it is diagonal.
\end{thm}
\begin{proof}
\textbf{Base Case}: Suppose that A is both upper triangular and unitary. In our lemmas, we have already shown that all non diagonal elements of the first column and row will be 0.

\textbf{Induction}: Suppose that the non diagonal elements in the first l columns and rows are 0. Then, we have shown in our lemmas that all non-diagonal elements in the first l+1 columns and rows are 0.

\textbf{Conclusion}: Then, by the principle of mathematical induction, all non-diagonal elements in all of the first m columns and rows are 0.
\end{proof}

\begin{thm}
If a matrix A is both lower triangular and unitary, it is diagonal.
\end{thm}
\begin{proof}
As A is unitary, $AA^{*} = I$. As A is lower triangular, $A^{*}$ is both upper triangular and unitary, because of which we may apply the theorem we proved for such matreces to conclude that $A^{*}$ is diagonal. Note that A is diagonal if and only if $A^{*}$ is diagonal. Hence, A is diagonal.
\end{proof}


\section{Question}
2.2 The Pythagorean theorem asserts that for a set of n orthogonal vectors $\{x_{i}\}$, $||\sum_{i=1}^{n} x_{i}||^{2} = \sum_{i=1}^{n}|| x_{i}||^{2}$.

\subsubsection{1.}
Prove this for case n=2 by an explicit computation of $||x_{1}+x_{2}||^{2}$.

\subsubsection{2.}
Show that this computation also establishes the general case by induction.

\subsection{Answer}
\subsubsection{1.}
\begin{thm}
For a set of n=2 orthogonal vectors $\{x_{i}\}$, \\
$||\sum_{i=1}^{n} x_{i}||^{2} = \sum_{i=1}^{n}|| x_{i}||^{2}$.
\end{thm}
\begin{proof}
\begin{eqnarray}
||x_{1}+x_{2}||^{2} &=& (x_{1}+x_{2})^{*}(x_{1}+x_{2})\\
&=& (x_{1}^{*}+x_{2}^{*})(x_{1}+x_{2})\\
&=& x_{1}^{*}x_{1}+x_{2}^{*}x_{1} + x_{1}^{*}x_{2}+x_{2}^{*}x_{2}\\
&=& x_{1}^{*}x_{1} +x_{2}^{*}x_{2} \texttt{: Due to orthogonality.}\\
&=& \sum_{i=1}^{2}||x_{i}||^{2}
\end{eqnarray} 
\end{proof}

\subsubsection{2.}
\begin{thm}
For a set of n orthogonal vectors $\{x_{i}\}$, \\
$||\sum_{i=1}^{n} x_{i}||^{2} = \sum_{i=1}^{n}|| x_{i}||^{2}$.
\end{thm}
\begin{proof}
\textbf{Base case}: For n=2, the statement has been proved to be true.

\textbf{Induction}: Suppose that it is true for n. Then, consider a set of n+1 orthogonal vectors $\{x_{i}\}$. Then:
\begin{eqnarray}
||\sum_{i=1}^{n+1} x_{i}||^{2} &=& ||\sum_{i=1}^{n} x_{i} + x_{n+1}||^{2} \\
&=& || \sum_{i=1}^{n} x_{i}||^{2} + ||x_{n+1}||^{2} \texttt{ : Using theorem for n=2}\\
&=& \sum_{i=1}^{n+1} || x_{i}||^{2} \texttt{ : Applying inductive hypothesis.}\\
\end{eqnarray}

Thus, by the principle of mathematical induction, we have the result.
\end{proof}

\section{Question}
2.6 If u and v are m-vectors, the matrix $A=I+uv^{*}$ is known as a rank-one perturbation of the identity. Show that if A is nonsingular, then its inverse has the form $A^{-1} = I + auv$ for some scalar a, and give an expression for a. For what u and v is A singular? If it is singular, what is null(A)?

\subsection{Answer}
\begin{thm}
If $A=I+uv^{*}$ is nonsingular, then its inverse has the form $A^{-1} = I + auv^{*}$ for some scalar a.
\end{thm}
\begin{proof}
\begin{eqnarray}
A(I+auv^{*}) &=& (I+uv^{*})(I+auv^{*})\\
&=& I + uv^{*} + auv^{*} + uv^{*}auv^{*}\\
&=& I + uv^{*} + auv^{*} + auv^{*}uv^{*}\\
&=& I + uv^{*} + uv^{*} a(1 + v^{*}u)\\
&=& I \texttt{ If $a = \frac{-1}{(1 + v^{*}u)}$}\\
\end{eqnarray}
\end{proof}

\begin{rem}
We have found above, the expression $a = \frac{-1}{(1 + v^{*}u)}$.
\end{rem}

\begin{cor}
a fails to exist if $v^{*}u = -1$. For all other cases, $A^{-1}$ may be found using the value of a found above.
\end{cor}

\begin{thm}
For values of v and u where $v^{*}u = -1$, A is singular. Null(A) is 1-dimensional, and has u as its basis.
\end{thm}
\begin{proof}
\begin{eqnarray}
Ax &=& (I+uv^{*})x \\
&=& x+ uv^{*}x \\
Au &=& u+ uv^{*}u \\
\end{eqnarray}
Hence, Au = 0 if $v^{*}u = -1$. u is assumed to be non zero. So, A is singular in such a case. We have noted in the corollary above that in all other cases, A is non-singular. Hence, Null(A) is 1-dimensional, and has u as its basis.
\end{proof}

% \bibliographystyle{plain}
% \bibliography{../linAlg}


\end{document}
